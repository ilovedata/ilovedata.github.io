% Options for packages loaded elsewhere
\PassOptionsToPackage{unicode}{hyperref}
\PassOptionsToPackage{hyphens}{url}
%
\documentclass[
]{book}
\usepackage{lmodern}
\usepackage{amsmath}
\usepackage{ifxetex,ifluatex}
\ifnum 0\ifxetex 1\fi\ifluatex 1\fi=0 % if pdftex
  \usepackage[T1]{fontenc}
  \usepackage[utf8]{inputenc}
  \usepackage{textcomp} % provide euro and other symbols
  \usepackage{amssymb}
\else % if luatex or xetex
  \usepackage{unicode-math}
  \defaultfontfeatures{Scale=MatchLowercase}
  \defaultfontfeatures[\rmfamily]{Ligatures=TeX,Scale=1}
\fi
% Use upquote if available, for straight quotes in verbatim environments
\IfFileExists{upquote.sty}{\usepackage{upquote}}{}
\IfFileExists{microtype.sty}{% use microtype if available
  \usepackage[]{microtype}
  \UseMicrotypeSet[protrusion]{basicmath} % disable protrusion for tt fonts
}{}
\makeatletter
\@ifundefined{KOMAClassName}{% if non-KOMA class
  \IfFileExists{parskip.sty}{%
    \usepackage{parskip}
  }{% else
    \setlength{\parindent}{0pt}
    \setlength{\parskip}{6pt plus 2pt minus 1pt}}
}{% if KOMA class
  \KOMAoptions{parskip=half}}
\makeatother
\usepackage{xcolor}
\IfFileExists{xurl.sty}{\usepackage{xurl}}{} % add URL line breaks if available
\IfFileExists{bookmark.sty}{\usepackage{bookmark}}{\usepackage{hyperref}}
\hypersetup{
  pdftitle={회귀모형 R 실습},
  pdfauthor={서울시립대학교 통계학과 이용희},
  hidelinks,
  pdfcreator={LaTeX via pandoc}}
\urlstyle{same} % disable monospaced font for URLs
\usepackage{color}
\usepackage{fancyvrb}
\newcommand{\VerbBar}{|}
\newcommand{\VERB}{\Verb[commandchars=\\\{\}]}
\DefineVerbatimEnvironment{Highlighting}{Verbatim}{commandchars=\\\{\}}
% Add ',fontsize=\small' for more characters per line
\usepackage{framed}
\definecolor{shadecolor}{RGB}{248,248,248}
\newenvironment{Shaded}{\begin{snugshade}}{\end{snugshade}}
\newcommand{\AlertTok}[1]{\textcolor[rgb]{0.94,0.16,0.16}{#1}}
\newcommand{\AnnotationTok}[1]{\textcolor[rgb]{0.56,0.35,0.01}{\textbf{\textit{#1}}}}
\newcommand{\AttributeTok}[1]{\textcolor[rgb]{0.77,0.63,0.00}{#1}}
\newcommand{\BaseNTok}[1]{\textcolor[rgb]{0.00,0.00,0.81}{#1}}
\newcommand{\BuiltInTok}[1]{#1}
\newcommand{\CharTok}[1]{\textcolor[rgb]{0.31,0.60,0.02}{#1}}
\newcommand{\CommentTok}[1]{\textcolor[rgb]{0.56,0.35,0.01}{\textit{#1}}}
\newcommand{\CommentVarTok}[1]{\textcolor[rgb]{0.56,0.35,0.01}{\textbf{\textit{#1}}}}
\newcommand{\ConstantTok}[1]{\textcolor[rgb]{0.00,0.00,0.00}{#1}}
\newcommand{\ControlFlowTok}[1]{\textcolor[rgb]{0.13,0.29,0.53}{\textbf{#1}}}
\newcommand{\DataTypeTok}[1]{\textcolor[rgb]{0.13,0.29,0.53}{#1}}
\newcommand{\DecValTok}[1]{\textcolor[rgb]{0.00,0.00,0.81}{#1}}
\newcommand{\DocumentationTok}[1]{\textcolor[rgb]{0.56,0.35,0.01}{\textbf{\textit{#1}}}}
\newcommand{\ErrorTok}[1]{\textcolor[rgb]{0.64,0.00,0.00}{\textbf{#1}}}
\newcommand{\ExtensionTok}[1]{#1}
\newcommand{\FloatTok}[1]{\textcolor[rgb]{0.00,0.00,0.81}{#1}}
\newcommand{\FunctionTok}[1]{\textcolor[rgb]{0.00,0.00,0.00}{#1}}
\newcommand{\ImportTok}[1]{#1}
\newcommand{\InformationTok}[1]{\textcolor[rgb]{0.56,0.35,0.01}{\textbf{\textit{#1}}}}
\newcommand{\KeywordTok}[1]{\textcolor[rgb]{0.13,0.29,0.53}{\textbf{#1}}}
\newcommand{\NormalTok}[1]{#1}
\newcommand{\OperatorTok}[1]{\textcolor[rgb]{0.81,0.36,0.00}{\textbf{#1}}}
\newcommand{\OtherTok}[1]{\textcolor[rgb]{0.56,0.35,0.01}{#1}}
\newcommand{\PreprocessorTok}[1]{\textcolor[rgb]{0.56,0.35,0.01}{\textit{#1}}}
\newcommand{\RegionMarkerTok}[1]{#1}
\newcommand{\SpecialCharTok}[1]{\textcolor[rgb]{0.00,0.00,0.00}{#1}}
\newcommand{\SpecialStringTok}[1]{\textcolor[rgb]{0.31,0.60,0.02}{#1}}
\newcommand{\StringTok}[1]{\textcolor[rgb]{0.31,0.60,0.02}{#1}}
\newcommand{\VariableTok}[1]{\textcolor[rgb]{0.00,0.00,0.00}{#1}}
\newcommand{\VerbatimStringTok}[1]{\textcolor[rgb]{0.31,0.60,0.02}{#1}}
\newcommand{\WarningTok}[1]{\textcolor[rgb]{0.56,0.35,0.01}{\textbf{\textit{#1}}}}
\usepackage{longtable,booktabs}
\usepackage{calc} % for calculating minipage widths
% Correct order of tables after \paragraph or \subparagraph
\usepackage{etoolbox}
\makeatletter
\patchcmd\longtable{\par}{\if@noskipsec\mbox{}\fi\par}{}{}
\makeatother
% Allow footnotes in longtable head/foot
\IfFileExists{footnotehyper.sty}{\usepackage{footnotehyper}}{\usepackage{footnote}}
\makesavenoteenv{longtable}
\usepackage{graphicx}
\makeatletter
\def\maxwidth{\ifdim\Gin@nat@width>\linewidth\linewidth\else\Gin@nat@width\fi}
\def\maxheight{\ifdim\Gin@nat@height>\textheight\textheight\else\Gin@nat@height\fi}
\makeatother
% Scale images if necessary, so that they will not overflow the page
% margins by default, and it is still possible to overwrite the defaults
% using explicit options in \includegraphics[width, height, ...]{}
\setkeys{Gin}{width=\maxwidth,height=\maxheight,keepaspectratio}
% Set default figure placement to htbp
\makeatletter
\def\fps@figure{htbp}
\makeatother
\setlength{\emergencystretch}{3em} % prevent overfull lines
\providecommand{\tightlist}{%
  \setlength{\itemsep}{0pt}\setlength{\parskip}{0pt}}
\setcounter{secnumdepth}{5}
%----- my options----------------
\usepackage[hangul]{kotex}
\usepackage{bm}
\usepackage{fullpage}

\newcommand{\pardiff}[2]{\frac{\partial #1}{\partial #2 }}
\newcommand{\pardiffl}[2]{{\partial #1}/{\partial #2 }}
\newcommand{\pardiffd}[2]{\frac{\partial^2 #1}{\partial #2^t \partial #2 }}
\newcommand{\pardiffdd}[3]{\frac{\partial^2 #1}{\partial #2 \partial #3 }}
\newcommand{\norm}[1]{\left\lVert#1\right\rVert}
\newcommand{\hatmat}{\bm X ({\bm X}^t {\bm X} )^{-1} {\bm X}^t}
\newcommand{\hatmatt}[1]{\bm X_{#1} ({\bm X}_{#1}^t {\bm X}_{#1})^{-1} {\bm X}_{#1}^t}

%--------- from bookdown.org --------------

\usepackage{booktabs}


\usepackage{framed,color}
\definecolor{shadecolor}{RGB}{248,248,248}

\renewcommand{\textfraction}{0.05}
\renewcommand{\topfraction}{0.8}
\renewcommand{\bottomfraction}{0.8}
\renewcommand{\floatpagefraction}{0.75}

\renewenvironment{quote}{\begin{VF}}{\end{VF}}
\let\oldhref\href
\renewcommand{\href}[2]{#2\footnote{\url{#1}}}

\makeatletter
\newenvironment{kframe}{%
\medskip{}
\setlength{\fboxsep}{.8em}
 \def\at@end@of@kframe{}%
 \ifinner\ifhmode%
  \def\at@end@of@kframe{\end{minipage}}%
  \begin{minipage}{\columnwidth}%
 \fi\fi%
 \def\FrameCommand##1{\hskip\@totalleftmargin \hskip-\fboxsep
 \colorbox{shadecolor}{##1}\hskip-\fboxsep
     % There is no \\@totalrightmargin, so:
     \hskip-\linewidth \hskip-\@totalleftmargin \hskip\columnwidth}%
 \MakeFramed {\advance\hsize-\width
   \@totalleftmargin\z@ \linewidth\hsize
   \@setminipage}}%
 {\par\unskip\endMakeFramed%
 \at@end@of@kframe}
\makeatother

\makeatletter

\@ifundefined{Shaded}{
}{\renewenvironment{Shaded}{\begin{kframe}}{\end{kframe}}}
\makeatother

\newenvironment{rmdblock}[1]
  {
  \begin{itemize}
  \renewcommand{\labelitemi}{
    \raisebox{-.7\height}[0pt][0pt]{
      {\setkeys{Gin}{width=3em,keepaspectratio}\includegraphics{images/#1}}
    }
  }
  \setlength{\fboxsep}{1em}
  \begin{kframe}
  \item
  }
  {
  \end{kframe}
  \end{itemize}
  }
  
\newenvironment{rmdnote}
  {\begin{rmdblock}{note}}
  {\end{rmdblock}}
  
\newenvironment{rmdcaution}
  {\begin{rmdblock}{caution}}
  {\end{rmdblock}}
  
\newenvironment{rmdimportant}
  {\begin{rmdblock}{important}}
  {\end{rmdblock}}
  
\newenvironment{rmdtip}
  {\begin{rmdblock}{tip}}
  {\end{rmdblock}}
  
\newenvironment{rmdwarning}
  {\begin{rmdblock}{warning}}
  {\end{rmdblock}}
  


\usepackage{makeidx}
\makeindex

\urlstyle{tt}

\usepackage{amsthm}
\makeatletter
 \def\thm@space@setup{%
   \thm@preskip=8pt plus 2pt minus 4pt
   \thm@postskip=\thm@preskip
}
\makeatother

\frontmatter
\usepackage{booktabs}
\usepackage{longtable}
\usepackage{array}
\usepackage{multirow}
\usepackage{wrapfig}
\usepackage{float}
\usepackage{colortbl}
\usepackage{pdflscape}
\usepackage{tabu}
\usepackage{threeparttable}
\usepackage{threeparttablex}
\usepackage[normalem]{ulem}
\usepackage{makecell}
\usepackage{xcolor}
\ifluatex
  \usepackage{selnolig}  % disable illegal ligatures
\fi
\usepackage[]{natbib}
\bibliographystyle{apalike}

\title{회귀모형 R 실습}
\author{서울시립대학교 통계학과 이용희}
\date{2021-04-27}

\begin{document}
\maketitle

{
\setcounter{tocdepth}{1}
\tableofcontents
}
\hypertarget{preface}{%
\chapter*{Preface}\label{preface}}


이 책은 일반 선형모형에 대한 R 프로그램과 결과에 대하여 설명합니다.

다음과 같은 R 패키지가 필요합니다.

\begin{verbatim}
library(ggplot2)
library(dplyr)
library(tidyr)
library(kableExtra)
library(regbook)
library(ellipse)
library(car)
library(MASS)
library(Matrix)
\end{verbatim}

\begin{rmdimportant}
이 책에서 사용된 기호, 표기법, 프로그램의 규칙과 쓰임은 다음과 같습니다.

\begin{itemize}
\tightlist
\item
  스칼라(scalar)와 일변량 확률변수는 일반적으로 보통 글씨체의 소문자로 표기한다. 특별한 이유가 있는 경우 대문자로 표시할 것이다.
\item
  벡터, 행렬, 다변량 확률벡터는 굵은 글씨체로 표기한다.
\item
  통계 프로그램은 \texttt{R}을 이용하였다. 각 예제에 사용된 \texttt{R} 프로그램은 코드 상자를 열면 나타난다.
\item
  통계 프로그램은 \texttt{R}에 대한 기초는 저자의 홈페이지에 있는 \href{https://ilovedata.github.io/computing/}{안내 사이트}에서 먼저 학습할 것을 권장한다.
\end{itemize}
\end{rmdimportant}

\hypertarget{chap02}{%
\chapter{단순회귀 예제}\label{chap02}}

\hypertarget{mammal-uxc790uxb8cc}{%
\section{MAMMAL 자료}\label{mammal-uxc790uxb8cc}}

이제 교재 109 페이지(연습문제 2.25) 에서 소개된 포유류의 뇌의 무게와 몸무게에 대한 자료 데이터프레임 \texttt{Mammal}를 사용할 수 있다.

\begin{Shaded}
\begin{Highlighting}[]
\FunctionTok{head}\NormalTok{(mammal)}
\end{Highlighting}
\end{Shaded}

\begin{verbatim}
##              brain    body
## Arctic fox  44.500   3.385
## Owl monkey  15.499   0.480
## Beaver       8.100   1.350
## Cow        423.012 464.983
## Gray wolf  119.498  36.328
## Goat       114.996  27.660
\end{verbatim}

\begin{Shaded}
\begin{Highlighting}[]
\FunctionTok{plot}\NormalTok{(brain}\SpecialCharTok{\textasciitilde{}}\NormalTok{body, }\AttributeTok{data=}\NormalTok{mammal) }
\end{Highlighting}
\end{Shaded}

\includegraphics{lmpractice_files/figure-latex/unnamed-chunk-4-1.pdf}

\hypertarget{uxbcc0uxc218uxc758-uxbcc0uxd658}{%
\section{변수의 변환}\label{uxbcc0uxc218uxc758-uxbcc0uxd658}}

데이터프레임 \texttt{Mammal}의 두 두 변수를 \texttt{log10()} 함수를 이용하여 변환하고 새로운 변수를 만들자.

\begin{Shaded}
\begin{Highlighting}[]
\NormalTok{mammal}\SpecialCharTok{$}\NormalTok{lbrain }\OtherTok{\textless{}{-}} \FunctionTok{log10}\NormalTok{(mammal}\SpecialCharTok{$}\NormalTok{brain)}
\NormalTok{mammal}\SpecialCharTok{$}\NormalTok{lbody }\OtherTok{\textless{}{-}} \FunctionTok{log10}\NormalTok{(mammal}\SpecialCharTok{$}\NormalTok{body)}
\FunctionTok{head}\NormalTok{(mammal)}
\end{Highlighting}
\end{Shaded}

\begin{verbatim}
##              brain    body   lbrain      lbody
## Arctic fox  44.500   3.385 1.648360  0.5295587
## Owl monkey  15.499   0.480 1.190304 -0.3187588
## Beaver       8.100   1.350 0.908485  0.1303338
## Cow        423.012 464.983 2.626353  2.6674371
## Gray wolf  119.498  36.328 2.077361  1.5602415
## Goat       114.996  27.660 2.060683  1.4418522
\end{verbatim}

\begin{Shaded}
\begin{Highlighting}[]
\FunctionTok{plot}\NormalTok{(lbrain}\SpecialCharTok{\textasciitilde{}}\NormalTok{lbody, }\AttributeTok{data=}\NormalTok{mammal)}
\end{Highlighting}
\end{Shaded}

\includegraphics{lmpractice_files/figure-latex/unnamed-chunk-5-1.pdf}

\hypertarget{uxd2b9uxbcc4uxd55c-uxc790uxb8ccuxb97c-uxcc3euxae30}{%
\section{특별한 자료를 찾기}\label{uxd2b9uxbcc4uxd55c-uxc790uxb8ccuxb97c-uxcc3euxae30}}

자료에서 최대값과 최소값을 찾고 그 위치를 알아보는 방법은 여러 가지가 있다.

일단 산점도를 그린 후에 마우스를 이용하여 자료의 특성을 알아낼 수 있는 방법이 있다.
이러한 방법은 \texttt{plot()}으로 산범도를 그린 후에 \texttt{identify()}함수를 이용하면 마우스를 이용하여 동물의 이름을 볼수 있다.

\begin{Shaded}
\begin{Highlighting}[]
\FunctionTok{plot}\NormalTok{(lbrain}\SpecialCharTok{\textasciitilde{}}\NormalTok{lbody, }\AttributeTok{data=}\NormalTok{mammal)}
\FunctionTok{with}\NormalTok{(mammal, }\FunctionTok{identify}\NormalTok{(lbody, lbrain, }\AttributeTok{labels =} \FunctionTok{rownames}\NormalTok{(mammal)))}
\end{Highlighting}
\end{Shaded}

\includegraphics{lmpractice_files/figure-latex/unnamed-chunk-6-1.pdf}

\begin{verbatim}
## integer(0)
\end{verbatim}

데이터프레임 \texttt{mammal}에 있는 각 동물의 이름은 \texttt{rownames()} 함수를 통하여 알 수 있다.

\begin{Shaded}
\begin{Highlighting}[]
\FunctionTok{rownames}\NormalTok{(mammal)}
\end{Highlighting}
\end{Shaded}

\begin{verbatim}
##  [1] "Arctic fox"               "Owl monkey"               "Beaver"                   "Cow"                      "Gray wolf"                "Goat"                     "Roe deer"                 "Guinea pig"               "Vervet\""                 "Chinchilla"              
## [11] "Ground squirrel"          "Arctic ground squirrel"   "African giant pouched ra" "Lesser short-tailed shre" "Star-nosed mole"          "Nine-banded armadillo"    "Tree hyrax"               "N. American opossum"      "Asian elephant"           "Big brown bat"           
## [21] "Donkey"                   "Horse"                    "European hedgehog"        "Patas monkey"             "Cat"                      "Galago"                   "Genet"                    "Giraffe"                  "Gorilla"                  "Gray seal"               
## [31] "Rock hyrax1"              "Human"                    "African elephant"         "Water opossum"            "Rhesus monkey"            "Kangaroo"                 "Yellow-bellied marmot"    "Golden hamster"           "Mouse"                    "Little brown bat"        
## [41] "Slow loris"               "Okapi"                    "Rabbit"                   "Sheep"                    "Jaguar"                   "Chimpanzee"               "Baboon"                   "Desert hedgehog"          "Giant armadillo"          "Rock hyrax2"             
## [51] "Raccoon"                  "Rat"                      "E. American mole"         "Mole rat"                 "Musk shrew"               "Pig"                      "Echidna"                  "Brazilian tapir"          "Tenrec"                   "Phalanger"               
## [61] "Tree shrew"               "Red fox"
\end{verbatim}

\hypertarget{uxc790uxb8ccuxc758-uxc815uxb82c}{%
\section{자료의 정렬}\label{uxc790uxb8ccuxc758-uxc815uxb82c}}

\begin{itemize}
\item
  벡터에 있는 자료들을 크기순으로 정렬하고 싶다면 함수 \texttt{sort()}를 사용한다. 내림차순 정렬이 기본이고 내림차순으로 정렬하려면 \texttt{sort(x,\ decreasing\ =\ TRUE)}로 사용한다.
\item
  또한 벡터에 있는자료가 정렬된 순서(기본은 내림차순)를 구하고 싶으면 함수 \texttt{order()}를 사용한다. 내림차순의 순서를 구하고 싶으면 \texttt{order(x,\ decreasing\ =\ TRUE)}를 사용한다.
\end{itemize}

\begin{Shaded}
\begin{Highlighting}[]
\NormalTok{mammal}\SpecialCharTok{$}\NormalTok{body}
\end{Highlighting}
\end{Shaded}

\begin{verbatim}
##  [1]    3.385    0.480    1.350  464.983   36.328   27.660   14.831    1.040    4.190    0.425    0.101    0.920    1.000    0.005    0.060    3.500    2.000    1.700 2547.070    0.023  187.092  521.026    0.785   10.000    3.300    0.200    1.410  529.006  206.996   85.004    0.750   61.998
## [33] 6654.180    3.500    6.800   34.998    4.050    0.120    0.023    0.010    1.400  250.010    2.500   55.501  100.003   52.159   10.550    0.550   59.997    3.600    4.288    0.280    0.075    0.122    0.048  192.001    3.000  160.004    0.900    1.620    0.104    4.235
\end{verbatim}

\begin{Shaded}
\begin{Highlighting}[]
\FunctionTok{sort}\NormalTok{(mammal}\SpecialCharTok{$}\NormalTok{body)}
\end{Highlighting}
\end{Shaded}

\begin{verbatim}
##  [1]    0.005    0.010    0.023    0.023    0.048    0.060    0.075    0.101    0.104    0.120    0.122    0.200    0.280    0.425    0.480    0.550    0.750    0.785    0.900    0.920    1.000    1.040    1.350    1.400    1.410    1.620    1.700    2.000    2.500    3.000    3.300    3.385
## [33]    3.500    3.500    3.600    4.050    4.190    4.235    4.288    6.800   10.000   10.550   14.831   27.660   34.998   36.328   52.159   55.501   59.997   61.998   85.004  100.003  160.004  187.092  192.001  206.996  250.010  464.983  521.026  529.006 2547.070 6654.180
\end{verbatim}

\begin{Shaded}
\begin{Highlighting}[]
\FunctionTok{order}\NormalTok{(mammal}\SpecialCharTok{$}\NormalTok{body)}
\end{Highlighting}
\end{Shaded}

\begin{verbatim}
##  [1] 14 40 20 39 55 15 53 11 61 38 54 26 52 10  2 48 31 23 59 12 13  8  3 41 27 60 18 17 43 57 25  1 16 34 50 37  9 62 51 35 24 47  7  6 36  5 46 44 49 32 30 45 58 21 56 29 42  4 22 28 19 33
\end{verbatim}

자료의 최대값과 최소값을 구하는 함수는 \texttt{max()}와 \texttt{min()}이다.

\begin{Shaded}
\begin{Highlighting}[]
\FunctionTok{max}\NormalTok{(mammal}\SpecialCharTok{$}\NormalTok{lbrain)}
\end{Highlighting}
\end{Shaded}

\begin{verbatim}
## [1] 3.756778
\end{verbatim}

\begin{Shaded}
\begin{Highlighting}[]
\FunctionTok{min}\NormalTok{(mammal}\SpecialCharTok{$}\NormalTok{lbrain)}
\end{Highlighting}
\end{Shaded}

\begin{verbatim}
## [1] -0.853872
\end{verbatim}

자료의 최대값과 최소값의 순서을 구하는 함수는 \texttt{which.max()}와 \texttt{which.min()}이다.
이러한 함수를 통해서 구해진 순서의 자료에 대한 변수를 모두 볼 수 있다.

\begin{Shaded}
\begin{Highlighting}[]
\FunctionTok{which.max}\NormalTok{(mammal}\SpecialCharTok{$}\NormalTok{body)}
\end{Highlighting}
\end{Shaded}

\begin{verbatim}
## [1] 33
\end{verbatim}

\begin{Shaded}
\begin{Highlighting}[]
\NormalTok{mammal[}\FunctionTok{which.max}\NormalTok{(mammal}\SpecialCharTok{$}\NormalTok{body), ]}
\end{Highlighting}
\end{Shaded}

\begin{verbatim}
##                    brain    body   lbrain    lbody
## African elephant 5711.86 6654.18 3.756778 3.823095
\end{verbatim}

\begin{Shaded}
\begin{Highlighting}[]
\FunctionTok{which.min}\NormalTok{(mammal}\SpecialCharTok{$}\NormalTok{body)}
\end{Highlighting}
\end{Shaded}

\begin{verbatim}
## [1] 14
\end{verbatim}

\begin{Shaded}
\begin{Highlighting}[]
\NormalTok{mammal[}\FunctionTok{which.min}\NormalTok{(mammal}\SpecialCharTok{$}\NormalTok{body), ]}
\end{Highlighting}
\end{Shaded}

\begin{verbatim}
##                          brain  body    lbrain    lbody
## Lesser short-tailed shre  0.14 0.005 -0.853872 -2.30103
\end{verbatim}

\hypertarget{uxb2e8uxc21cuxd68cuxadc0uxbaa8uxd615uxc758-uxc801uxd569}{%
\section{단순회귀모형의 적합}\label{uxb2e8uxc21cuxd68cuxadc0uxbaa8uxd615uxc758-uxc801uxd569}}

다음과 같은 단순선형모형을 고려하자
\[ y_i = \beta_0 + \beta_1  x_i + \epsilon_i,~~ i=1,2,\dots,n \]

데이터프레임 \texttt{mammal}에서 로그변환된 몸무게 \texttt{lbody}을 독립변수 \(x\)로 하고 로그변환된 뇌무게를 \texttt{lbrain}을 종속변수 \(y\)로 하는 선형회귀직선의 절편과 기울기를 다음과 같이 함수 \texttt{lm()}을 이용하여 추정할 수 있다.

\begin{Shaded}
\begin{Highlighting}[]
\NormalTok{mammal.lm }\OtherTok{\textless{}{-}} \FunctionTok{lm}\NormalTok{(lbrain}\SpecialCharTok{\textasciitilde{}}\NormalTok{lbody, }\AttributeTok{data=}\NormalTok{mammal)}
\FunctionTok{summary}\NormalTok{(mammal.lm)}
\end{Highlighting}
\end{Shaded}

\begin{verbatim}
## 
## Call:
## lm(formula = lbrain ~ lbody, data = mammal)
## 
## Residuals:
##      Min       1Q   Median       3Q      Max 
## -0.74503 -0.21380 -0.02676  0.18934  0.84615 
## 
## Coefficients:
##             Estimate Std. Error t value Pr(>|t|)    
## (Intercept)  0.92713    0.04171   22.23   <2e-16 ***
## lbody        0.75169    0.02846   26.41   <2e-16 ***
## ---
## Signif. codes:  0 '***' 0.001 '**' 0.01 '*' 0.05 '.' 0.1 ' ' 1
## 
## Residual standard error: 0.3015 on 60 degrees of freedom
## Multiple R-squared:  0.9208, Adjusted R-squared:  0.9195 
## F-statistic: 697.4 on 1 and 60 DF,  p-value: < 2.2e-16
\end{verbatim}

\hypertarget{uxc0b0uxc810uxb3c4uxc5d0uxc11c-uxd2b9uxc815-uxc790uxb8ccuxc758-uxd45cuxc2dc}{%
\section{산점도에서 특정 자료의 표시}\label{uxc0b0uxc810uxb3c4uxc5d0uxc11c-uxd2b9uxc815-uxc790uxb8ccuxc758-uxd45cuxc2dc}}

산점도에 인간 \texttt{human} 자료 \((x_i, y_i)\)를 표시하고 싶으면 다음과 같은 \texttt{R} 코드를 사용할 수 있다. 산점도에 문자를 표시하는 함수 \texttt{text()}를 사용한다.

\begin{verbatim}
text(x,y, labels="A", cex=1.0, pos=1 )
\end{verbatim}

먼저 사람(\texttt{Human})에 대한 자료만 선택한다.

\begin{Shaded}
\begin{Highlighting}[]
\NormalTok{pickhuman }\OtherTok{\textless{}{-}} \FunctionTok{rownames}\NormalTok{(mammal) }\SpecialCharTok{==} \StringTok{"Human"}  
\NormalTok{pickhuman}
\end{Highlighting}
\end{Shaded}

\begin{verbatim}
##  [1] FALSE FALSE FALSE FALSE FALSE FALSE FALSE FALSE FALSE FALSE FALSE FALSE FALSE FALSE FALSE FALSE FALSE FALSE FALSE FALSE FALSE FALSE FALSE FALSE FALSE FALSE FALSE FALSE FALSE FALSE FALSE  TRUE FALSE FALSE FALSE FALSE FALSE FALSE FALSE FALSE FALSE FALSE FALSE FALSE FALSE FALSE FALSE FALSE FALSE
## [50] FALSE FALSE FALSE FALSE FALSE FALSE FALSE FALSE FALSE FALSE FALSE FALSE FALSE
\end{verbatim}

\begin{Shaded}
\begin{Highlighting}[]
\NormalTok{dat1 }\OtherTok{\textless{}{-}}\NormalTok{ mammal[pickhuman,]}
\NormalTok{dat1}
\end{Highlighting}
\end{Shaded}

\begin{verbatim}
##         brain   body   lbrain    lbody
## Human 1320.02 61.998 3.120581 1.792378
\end{verbatim}

함수 \texttt{points()} 는 지정된 좌표\((x,y)\)에 기호를 표시하며 기호의 종류는 \texttt{pch=}를 이용하여 숫자로 기호의 종류를 지정한다. 예를 들어 \texttt{pch=3}는 \texttt{+} 를 나타낸다.

함수\texttt{text()}에서 지정된 좌표\((x,y)\)에 문자로 표시를 하며 \texttt{labels=}는 산점도에 표시할 문자열을 지정하고 \texttt{cex=}은 문자의 크기, \texttt{pos=}은 표시할 위치를 지정한다.

\begin{Shaded}
\begin{Highlighting}[]
\FunctionTok{plot}\NormalTok{(lbrain}\SpecialCharTok{\textasciitilde{}}\NormalTok{lbody, }\AttributeTok{data=}\NormalTok{mammal)}
\FunctionTok{with}\NormalTok{(dat1, }\FunctionTok{points}\NormalTok{(lbody,lbrain, }\AttributeTok{pch=}\DecValTok{3}\NormalTok{))}
\FunctionTok{with}\NormalTok{(dat1, }\FunctionTok{text}\NormalTok{(lbody,lbrain, }\AttributeTok{labels =}\FunctionTok{rownames}\NormalTok{(dat1), }\AttributeTok{cex=}\FloatTok{1.2}\NormalTok{, }\AttributeTok{pos =} \DecValTok{4}\NormalTok{))}
\end{Highlighting}
\end{Shaded}

\includegraphics{lmpractice_files/figure-latex/unnamed-chunk-13-1.pdf}

위의 코드를 이용하면 다음과 같이 모든 자료의 이름을 표시할 수 있지만 읽기 힘든 그림이다.

\begin{Shaded}
\begin{Highlighting}[]
\FunctionTok{plot}\NormalTok{(lbrain}\SpecialCharTok{\textasciitilde{}}\NormalTok{lbody, }\AttributeTok{data=}\NormalTok{mammal)}
\FunctionTok{with}\NormalTok{(mammal, }\FunctionTok{text}\NormalTok{(lbody,lbrain, }\AttributeTok{labels =}\FunctionTok{rownames}\NormalTok{(mammal), }\AttributeTok{cex=}\FloatTok{0.6}\NormalTok{, }\AttributeTok{pos =} \DecValTok{1}\NormalTok{))}
\end{Highlighting}
\end{Shaded}

\includegraphics{lmpractice_files/figure-latex/unnamed-chunk-14-1.pdf}

\mainmatter

\hypertarget{chapter03}{%
\chapter{중회귀분석}\label{chapter03}}

예제 3.3에 나온 중고차 가격자료를 이용한 R 실습입니다.
\#\#

\hypertarget{uxc911uxace0uxcc28-uxc790uxb8cc}{%
\section{중고차 자료}\label{uxc911uxace0uxcc28-uxc790uxb8cc}}

\begin{Shaded}
\begin{Highlighting}[]
\FunctionTok{head}\NormalTok{(usedcars)}
\end{Highlighting}
\end{Shaded}

\begin{verbatim}
##   price year mileage   cc automatic
## 1   790   78  133462 1998         1
## 2  1380   39   33000 2000         1
## 3   270  109  120000 1800         0
## 4  1190   20   69727 1999         1
## 5   590   70  112000 2000         0
## 6  1120   58   39106 1998         1
\end{verbatim}

\hypertarget{uxc0b0uxc810uxb3c4-uxd589uxb82c}{%
\section{산점도 행렬}\label{uxc0b0uxc810uxb3c4-uxd589uxb82c}}

\begin{Shaded}
\begin{Highlighting}[]
\FunctionTok{pairs}\NormalTok{(usedcars)}
\end{Highlighting}
\end{Shaded}

\includegraphics{lmpractice_files/figure-latex/unnamed-chunk-16-1.pdf}

\hypertarget{uxc911uxd68cuxadc0-uxbaa8uxd615uxc758-uxc801uxd569}{%
\section{중회귀 모형의 적합}\label{uxc911uxd68cuxadc0-uxbaa8uxd615uxc758-uxc801uxd569}}

\begin{Shaded}
\begin{Highlighting}[]
\NormalTok{fit0 }\OtherTok{\textless{}{-}} \FunctionTok{lm}\NormalTok{(price }\SpecialCharTok{\textasciitilde{}}\NormalTok{ year }\SpecialCharTok{+}\NormalTok{ mileage }\SpecialCharTok{+}\NormalTok{ cc }\SpecialCharTok{+}\NormalTok{ automatic, usedcars)}
\end{Highlighting}
\end{Shaded}

계획행렬은 다음과 같이 구할 수 있다.

\begin{Shaded}
\begin{Highlighting}[]
\FunctionTok{model.matrix}\NormalTok{(fit0)}
\end{Highlighting}
\end{Shaded}

\begin{verbatim}
##    (Intercept) year mileage   cc automatic
## 1            1   78  133462 1998         1
## 2            1   39   33000 2000         1
## 3            1  109  120000 1800         0
## 4            1   20   69727 1999         1
## 5            1   70  112000 2000         0
## 6            1   58   39106 1998         1
## 7            1   53   95935 1800         1
## 8            1   68  120000 1800         0
## 9            1   15   20215 1798         1
## 10           1   96  140000 1800         0
## 11           1   63   68924 1998         1
## 12           1   82   90000 2000         0
## 13           1   76   81279 1998         0
## 14           1   17   24070 1798         1
## 15           1   38   40000 2000         0
## 16           1   46   56887 1832         1
## 17           1   95   91216 1997         1
## 18           1   37   48680 1998         1
## 19           1   68    8000 2000         0
## 20           1   41   60634 1835         1
## 21           1   69  114131 1998         1
## 22           1   71   75000 1800         0
## 23           1   99  124417 1998         1
## 24           1  129  130000 1800         0
## 25           1   57   77559 1997         1
## 26           1  107   75216 1838         1
## 27           1   45   52000 2000         0
## 28           1   80   58000 2000         1
## 29           1  113  134500 1800         0
## 30           1   41   80000 2000         0
## attr(,"assign")
## [1] 0 1 2 3 4
\end{verbatim}

\texttt{fit0} 에 저장된 결과를 다음과 같이 함수 \texttt{str}을 이용하여 볼 수 있다.

\begin{Shaded}
\begin{Highlighting}[]
\FunctionTok{str}\NormalTok{(fit0)}
\end{Highlighting}
\end{Shaded}

\begin{verbatim}
## List of 12
##  $ coefficients : Named num [1:5] 525.28696 -5.79964 -0.00226 0.38879 165.31263
##   ..- attr(*, "names")= chr [1:5] "(Intercept)" "year" "mileage" "cc" ...
##  $ residuals    : Named num [1:30] 76.98 212.69 -51.4 -4.01 -53.45 ...
##   ..- attr(*, "names")= chr [1:30] "1" "2" "3" "4" ...
##  $ effects      : Named num [1:30] -4407 -1434 -369 -229 419 ...
##   ..- attr(*, "names")= chr [1:30] "(Intercept)" "year" "mileage" "cc" ...
##  $ rank         : int 5
##  $ fitted.values: Named num [1:30] 713 1167 321 1194 643 ...
##   ..- attr(*, "names")= chr [1:30] "1" "2" "3" "4" ...
##  $ assign       : int [1:5] 0 1 2 3 4
##  $ qr           :List of 5
##   ..$ qr   : num [1:30, 1:5] -5.477 0.183 0.183 0.183 0.183 ...
##   .. ..- attr(*, "dimnames")=List of 2
##   .. .. ..$ : chr [1:30] "1" "2" "3" "4" ...
##   .. .. ..$ : chr [1:5] "(Intercept)" "year" "mileage" "cc" ...
##   .. ..- attr(*, "assign")= int [1:5] 0 1 2 3 4
##   ..$ qraux: num [1:5] 1.18 1.18 1.08 1.03 1.26
##   ..$ pivot: int [1:5] 1 2 3 4 5
##   ..$ tol  : num 1e-07
##   ..$ rank : int 5
##   ..- attr(*, "class")= chr "qr"
##  $ df.residual  : int 25
##  $ xlevels      : Named list()
##  $ call         : language lm(formula = price ~ year + mileage + cc + automatic, data = usedcars)
##  $ terms        :Classes 'terms', 'formula'  language price ~ year + mileage + cc + automatic
##   .. ..- attr(*, "variables")= language list(price, year, mileage, cc, automatic)
##   .. ..- attr(*, "factors")= int [1:5, 1:4] 0 1 0 0 0 0 0 1 0 0 ...
##   .. .. ..- attr(*, "dimnames")=List of 2
##   .. .. .. ..$ : chr [1:5] "price" "year" "mileage" "cc" ...
##   .. .. .. ..$ : chr [1:4] "year" "mileage" "cc" "automatic"
##   .. ..- attr(*, "term.labels")= chr [1:4] "year" "mileage" "cc" "automatic"
##   .. ..- attr(*, "order")= int [1:4] 1 1 1 1
##   .. ..- attr(*, "intercept")= int 1
##   .. ..- attr(*, "response")= int 1
##   .. ..- attr(*, ".Environment")=<environment: R_GlobalEnv> 
##   .. ..- attr(*, "predvars")= language list(price, year, mileage, cc, automatic)
##   .. ..- attr(*, "dataClasses")= Named chr [1:5] "numeric" "numeric" "numeric" "numeric" ...
##   .. .. ..- attr(*, "names")= chr [1:5] "price" "year" "mileage" "cc" ...
##  $ model        :'data.frame':   30 obs. of  5 variables:
##   ..$ price    : int [1:30] 790 1380 270 1190 590 1120 815 450 1290 420 ...
##   ..$ year     : int [1:30] 78 39 109 20 70 58 53 68 15 96 ...
##   ..$ mileage  : int [1:30] 133462 33000 120000 69727 112000 39106 95935 120000 20215 140000 ...
##   ..$ cc       : int [1:30] 1998 2000 1800 1999 2000 1998 1800 1800 1798 1800 ...
##   ..$ automatic: int [1:30] 1 1 0 1 0 1 1 0 1 0 ...
##   ..- attr(*, "terms")=Classes 'terms', 'formula'  language price ~ year + mileage + cc + automatic
##   .. .. ..- attr(*, "variables")= language list(price, year, mileage, cc, automatic)
##   .. .. ..- attr(*, "factors")= int [1:5, 1:4] 0 1 0 0 0 0 0 1 0 0 ...
##   .. .. .. ..- attr(*, "dimnames")=List of 2
##   .. .. .. .. ..$ : chr [1:5] "price" "year" "mileage" "cc" ...
##   .. .. .. .. ..$ : chr [1:4] "year" "mileage" "cc" "automatic"
##   .. .. ..- attr(*, "term.labels")= chr [1:4] "year" "mileage" "cc" "automatic"
##   .. .. ..- attr(*, "order")= int [1:4] 1 1 1 1
##   .. .. ..- attr(*, "intercept")= int 1
##   .. .. ..- attr(*, "response")= int 1
##   .. .. ..- attr(*, ".Environment")=<environment: R_GlobalEnv> 
##   .. .. ..- attr(*, "predvars")= language list(price, year, mileage, cc, automatic)
##   .. .. ..- attr(*, "dataClasses")= Named chr [1:5] "numeric" "numeric" "numeric" "numeric" ...
##   .. .. .. ..- attr(*, "names")= chr [1:5] "price" "year" "mileage" "cc" ...
##  - attr(*, "class")= chr "lm"
\end{verbatim}

\hypertarget{uxd68cuxadc0uxacc4uxc218uxc758-uxcd94uxc815uxacfc-uxacb0uxc815uxacc4uxc218}{%
\section{회귀계수의 추정과 결정계수}\label{uxd68cuxadc0uxacc4uxc218uxc758-uxcd94uxc815uxacfc-uxacb0uxc815uxacc4uxc218}}

함수 \texttt{summary} 는 각 계수의 추정값과 가설 \(H_0: \beta_i=0\)에 대한 t-검정 결과를 보여준다.
또한 결정계수 \(R^2\)도 구해준다.

\begin{Shaded}
\begin{Highlighting}[]
\FunctionTok{summary}\NormalTok{(fit0)}
\end{Highlighting}
\end{Shaded}

\begin{verbatim}
## 
## Call:
## lm(formula = price ~ year + mileage + cc + automatic, data = usedcars)
## 
## Residuals:
##     Min      1Q  Median      3Q     Max 
## -177.35  -63.91   -0.99   70.34  212.69 
## 
## Coefficients:
##               Estimate Std. Error t value Pr(>|t|)    
## (Intercept)  5.253e+02  3.998e+02   1.314 0.200823    
## year        -5.800e+00  9.283e-01  -6.247 1.55e-06 ***
## mileage     -2.263e-03  7.211e-04  -3.138 0.004324 ** 
## cc           3.888e-01  2.022e-01   1.923 0.065958 .  
## automatic    1.653e+02  3.986e+01   4.147 0.000339 ***
## ---
## Signif. codes:  0 '***' 0.001 '**' 0.01 '*' 0.05 '.' 0.1 ' ' 1
## 
## Residual standard error: 101.1 on 25 degrees of freedom
## Multiple R-squared:  0.9045, Adjusted R-squared:  0.8892 
## F-statistic: 59.21 on 4 and 25 DF,  p-value: 2.184e-12
\end{verbatim}

각 회귀 계수에 대한 신뢰구간은 함수 \texttt{confint}로 구할 수 있다.

\begin{Shaded}
\begin{Highlighting}[]
\FunctionTok{confint}\NormalTok{(fit0)}
\end{Highlighting}
\end{Shaded}

\begin{verbatim}
##                     2.5 %        97.5 %
## (Intercept) -2.981256e+02  1.348699e+03
## year        -7.711605e+00 -3.887669e+00
## mileage     -3.748021e-03 -7.776672e-04
## cc          -2.763072e-02  8.052054e-01
## automatic    8.322275e+01  2.474025e+02
\end{verbatim}

공동 신뢰영역은 패키지 \texttt{ellipse} 에 있는 함수 \texttt{ellipse}를 이용해서 다음과 같이 그릴 수 있다.

\begin{Shaded}
\begin{Highlighting}[]
\FunctionTok{plot}\NormalTok{(ellipse}\SpecialCharTok{::}\FunctionTok{ellipse}\NormalTok{(fit0, }\AttributeTok{level =} \FloatTok{0.90}\NormalTok{), }\AttributeTok{type =} \StringTok{\textquotesingle{}l\textquotesingle{}}\NormalTok{)}
\end{Highlighting}
\end{Shaded}

\includegraphics{lmpractice_files/figure-latex/unnamed-chunk-22-1.pdf}

\begin{Shaded}
\begin{Highlighting}[]
\FunctionTok{plot}\NormalTok{(ellipse}\SpecialCharTok{::}\FunctionTok{ellipse}\NormalTok{(fit0, }\AttributeTok{which =} \FunctionTok{c}\NormalTok{(}\StringTok{\textquotesingle{}year\textquotesingle{}}\NormalTok{, }\StringTok{\textquotesingle{}mileage\textquotesingle{}}\NormalTok{), }\AttributeTok{level =} \FloatTok{0.90}\NormalTok{), }\AttributeTok{type =} \StringTok{\textquotesingle{}l\textquotesingle{}}\NormalTok{)}
\FunctionTok{points}\NormalTok{(fit0}\SpecialCharTok{$}\NormalTok{coefficients[}\StringTok{\textquotesingle{}year\textquotesingle{}}\NormalTok{], fit0}\SpecialCharTok{$}\NormalTok{coefficients[}\StringTok{\textquotesingle{}mileage\textquotesingle{}}\NormalTok{])}
\end{Highlighting}
\end{Shaded}

\includegraphics{lmpractice_files/figure-latex/unnamed-chunk-23-1.pdf}
\#\# 분산분석

\begin{Shaded}
\begin{Highlighting}[]
\FunctionTok{anova}\NormalTok{(fit0)}
\end{Highlighting}
\end{Shaded}

\begin{verbatim}
## Analysis of Variance Table
## 
## Response: price
##           Df  Sum Sq Mean Sq  F value    Pr(>F)    
## year       1 2056608 2056608 201.2036 1.841e-13 ***
## mileage    1  135864  135864  13.2919 0.0012228 ** 
## cc         1   52409   52409   5.1273 0.0324794 *  
## automatic  1  175828  175828  17.2018 0.0003389 ***
## Residuals 25  255538   10222                       
## ---
## Signif. codes:  0 '***' 0.001 '**' 0.01 '*' 0.05 '.' 0.1 ' ' 1
\end{verbatim}

\hypertarget{uxc608uxce21uxac12}{%
\section{예측값}\label{uxc608uxce21uxac12}}

반응변수에 대한 예측값 \(\hat {\bm y} = \bm X \hat {\bm \beta}\)는 함수 \texttt{redict}를 이용한다.

\begin{Shaded}
\begin{Highlighting}[]
\FunctionTok{predict}\NormalTok{(fit0)}
\end{Highlighting}
\end{Shaded}

\begin{verbatim}
##         1         2         3         4         5         6         7         8         9        10        11        12        13        14        15        16        17        18        19        20        21        22        23        24        25        26        27        28        29        30 
##  713.0214 1167.3146  321.4025 1194.0114  643.4485 1042.5270  865.9501  559.1876 1256.9013  351.5409  946.0553  623.6355  677.3900 1236.5788  991.9617 1007.3483  709.6348 1142.6549  890.3836 1029.0340  808.9611  643.6167  611.6964  182.7813  960.9247  614.4275  924.2101  872.9584  265.3927  884.0490
\end{verbatim}

새로운 자료에 대한 예측값 \(\widehat { E(y|x)}\)은 다음과 같이 데이터프레임을 만들고 예측한다.

\begin{Shaded}
\begin{Highlighting}[]
\NormalTok{nw }\OtherTok{\textless{}{-}} \FunctionTok{data.frame}\NormalTok{(}\AttributeTok{year=}\DecValTok{60}\NormalTok{, }\AttributeTok{mileage=}\DecValTok{10000}\NormalTok{, }\AttributeTok{cc=}\DecValTok{200}\NormalTok{, }\AttributeTok{automatic=}\DecValTok{1}\NormalTok{)}
\NormalTok{nw}
\end{Highlighting}
\end{Shaded}

\begin{verbatim}
##   year mileage  cc automatic
## 1   60   10000 200         1
\end{verbatim}

\begin{Shaded}
\begin{Highlighting}[]
\FunctionTok{predict}\NormalTok{(fit0, }\AttributeTok{newdata=}\NormalTok{nw, }\AttributeTok{interval=}\StringTok{"confidence"}\NormalTok{)}
\end{Highlighting}
\end{Shaded}

\begin{verbatim}
##        fit       lwr      upr
## 1 397.7504 -342.6272 1138.128
\end{verbatim}

새로운 관측값에 대항 예측은 다음과 같이 한다.

\begin{Shaded}
\begin{Highlighting}[]
\FunctionTok{predict}\NormalTok{(fit0, }\AttributeTok{newdata=}\NormalTok{nw, }\AttributeTok{interval=}\StringTok{"prediction"}\NormalTok{)}
\end{Highlighting}
\end{Shaded}

\begin{verbatim}
##        fit       lwr      upr
## 1 397.7504 -371.3501 1166.851
\end{verbatim}

\hypertarget{uxc794uxcc28-uxbd84uxc11d}{%
\section{잔차 분석}\label{uxc794uxcc28-uxbd84uxc11d}}

\begin{Shaded}
\begin{Highlighting}[]
\FunctionTok{plot}\NormalTok{(fit0)}
\end{Highlighting}
\end{Shaded}

\includegraphics{lmpractice_files/figure-latex/unnamed-chunk-28-1.pdf} \includegraphics{lmpractice_files/figure-latex/unnamed-chunk-28-2.pdf} \includegraphics{lmpractice_files/figure-latex/unnamed-chunk-28-3.pdf} \includegraphics{lmpractice_files/figure-latex/unnamed-chunk-28-4.pdf}

\hypertarget{chapter04}{%
\chapter{모형의 진단과 수정}\label{chapter04}}

예제 3.3에 나온 중고차 가격자료를 이용한 R 실습입니다.

\hypertarget{uxc21cuxcc28uxc81cuxacf1uxd569}{%
\section{순차제곱합}\label{uxc21cuxcc28uxc81cuxacf1uxd569}}

순차제곱합은 모형에 들어가는 변수의 순서에 따라서 제곱합이 틀려진다.

다음의 예를 보면 두 모형이 같은 변수들로 적합되지만 순서가 달라지면 순차제곱합이 다르다.

\begin{Shaded}
\begin{Highlighting}[]
\NormalTok{model1 }\OtherTok{\textless{}{-}}\NormalTok{ price }\SpecialCharTok{\textasciitilde{}}\NormalTok{ year }\SpecialCharTok{+}\NormalTok{ mileage }\SpecialCharTok{+}\NormalTok{ cc }\SpecialCharTok{+}\NormalTok{ automatic}
\NormalTok{model2 }\OtherTok{\textless{}{-}}\NormalTok{ price }\SpecialCharTok{\textasciitilde{}}\NormalTok{ mileage }\SpecialCharTok{+}\NormalTok{ automatic }\SpecialCharTok{+}\NormalTok{ cc }\SpecialCharTok{+}\NormalTok{ year}
\NormalTok{fit1 }\OtherTok{\textless{}{-}} \FunctionTok{lm}\NormalTok{(model1, usedcars)}
\NormalTok{fit2 }\OtherTok{\textless{}{-}} \FunctionTok{lm}\NormalTok{(model2, usedcars)}
\FunctionTok{anova}\NormalTok{(fit1)}
\end{Highlighting}
\end{Shaded}

\begin{verbatim}
## Analysis of Variance Table
## 
## Response: price
##           Df  Sum Sq Mean Sq  F value    Pr(>F)    
## year       1 2056608 2056608 201.2036 1.841e-13 ***
## mileage    1  135864  135864  13.2919 0.0012228 ** 
## cc         1   52409   52409   5.1273 0.0324794 *  
## automatic  1  175828  175828  17.2018 0.0003389 ***
## Residuals 25  255538   10222                       
## ---
## Signif. codes:  0 '***' 0.001 '**' 0.01 '*' 0.05 '.' 0.1 ' ' 1
\end{verbatim}

\begin{Shaded}
\begin{Highlighting}[]
\FunctionTok{anova}\NormalTok{(fit2)}
\end{Highlighting}
\end{Shaded}

\begin{verbatim}
## Analysis of Variance Table
## 
## Response: price
##           Df  Sum Sq Mean Sq  F value    Pr(>F)    
## mileage    1 1637355 1637355 160.1870 2.274e-12 ***
## automatic  1  341741  341741  33.4335 5.006e-06 ***
## cc         1   42683   42683   4.1758   0.05168 .  
## year       1  398929  398929  39.0283 1.552e-06 ***
## Residuals 25  255538   10222                       
## ---
## Signif. codes:  0 '***' 0.001 '**' 0.01 '*' 0.05 '.' 0.1 ' ' 1
\end{verbatim}

하지만 회귀계수의 추정량은 동일하다.

\begin{Shaded}
\begin{Highlighting}[]
\FunctionTok{summary}\NormalTok{(fit1)}
\end{Highlighting}
\end{Shaded}

\begin{verbatim}
## 
## Call:
## lm(formula = model1, data = usedcars)
## 
## Residuals:
##     Min      1Q  Median      3Q     Max 
## -177.35  -63.91   -0.99   70.34  212.69 
## 
## Coefficients:
##               Estimate Std. Error t value Pr(>|t|)    
## (Intercept)  5.253e+02  3.998e+02   1.314 0.200823    
## year        -5.800e+00  9.283e-01  -6.247 1.55e-06 ***
## mileage     -2.263e-03  7.211e-04  -3.138 0.004324 ** 
## cc           3.888e-01  2.022e-01   1.923 0.065958 .  
## automatic    1.653e+02  3.986e+01   4.147 0.000339 ***
## ---
## Signif. codes:  0 '***' 0.001 '**' 0.01 '*' 0.05 '.' 0.1 ' ' 1
## 
## Residual standard error: 101.1 on 25 degrees of freedom
## Multiple R-squared:  0.9045, Adjusted R-squared:  0.8892 
## F-statistic: 59.21 on 4 and 25 DF,  p-value: 2.184e-12
\end{verbatim}

\begin{Shaded}
\begin{Highlighting}[]
\FunctionTok{summary}\NormalTok{(fit2)}
\end{Highlighting}
\end{Shaded}

\begin{verbatim}
## 
## Call:
## lm(formula = model2, data = usedcars)
## 
## Residuals:
##     Min      1Q  Median      3Q     Max 
## -177.35  -63.91   -0.99   70.34  212.69 
## 
## Coefficients:
##               Estimate Std. Error t value Pr(>|t|)    
## (Intercept)  5.253e+02  3.998e+02   1.314 0.200823    
## mileage     -2.263e-03  7.211e-04  -3.138 0.004324 ** 
## automatic    1.653e+02  3.986e+01   4.147 0.000339 ***
## cc           3.888e-01  2.022e-01   1.923 0.065958 .  
## year        -5.800e+00  9.283e-01  -6.247 1.55e-06 ***
## ---
## Signif. codes:  0 '***' 0.001 '**' 0.01 '*' 0.05 '.' 0.1 ' ' 1
## 
## Residual standard error: 101.1 on 25 degrees of freedom
## Multiple R-squared:  0.9045, Adjusted R-squared:  0.8892 
## F-statistic: 59.21 on 4 and 25 DF,  p-value: 2.184e-12
\end{verbatim}

\hypertarget{uxd3b8uxc81cuxacf1uxd569}{%
\section{편제곱합}\label{uxd3b8uxc81cuxacf1uxd569}}

편제곱합은 다른 변수들로 보정된 제곱합으로 순서에 관계없이 일정하다.패키지 \texttt{car} 에 있는 함수 \texttt{Anova} 를 사용하면
편제곱합을 구할 수 있다.

\begin{Shaded}
\begin{Highlighting}[]
\FunctionTok{Anova}\NormalTok{(fit1, }\AttributeTok{type=}\StringTok{"III"}\NormalTok{)}
\end{Highlighting}
\end{Shaded}

\begin{verbatim}
## Anova Table (Type III tests)
## 
## Response: price
##             Sum Sq Df F value    Pr(>F)    
## (Intercept)  17645  1  1.7262 0.2008228    
## year        398929  1 39.0283 1.552e-06 ***
## mileage     100649  1  9.8467 0.0043244 ** 
## cc           37794  1  3.6975 0.0659577 .  
## automatic   175828  1 17.2018 0.0003389 ***
## Residuals   255538 25                      
## ---
## Signif. codes:  0 '***' 0.001 '**' 0.01 '*' 0.05 '.' 0.1 ' ' 1
\end{verbatim}

\begin{Shaded}
\begin{Highlighting}[]
\FunctionTok{Anova}\NormalTok{(fit2, }\AttributeTok{type=}\StringTok{"III"}\NormalTok{)}
\end{Highlighting}
\end{Shaded}

\begin{verbatim}
## Anova Table (Type III tests)
## 
## Response: price
##             Sum Sq Df F value    Pr(>F)    
## (Intercept)  17645  1  1.7262 0.2008228    
## mileage     100649  1  9.8467 0.0043244 ** 
## automatic   175828  1 17.2018 0.0003389 ***
## cc           37794  1  3.6975 0.0659577 .  
## year        398929  1 39.0283 1.552e-06 ***
## Residuals   255538 25                      
## ---
## Signif. codes:  0 '***' 0.001 '**' 0.01 '*' 0.05 '.' 0.1 ' ' 1
\end{verbatim}

\hypertarget{uxbd80uxbd84-f-uxac80uxc815}{%
\section{부분 F 검정}\label{uxbd80uxbd84-f-uxac80uxc815}}

배기량(cc)에 대항 계수가 0인지 검정해보자.

\[ H_0: ~ \beta_k =0 \]

하나의 계수에 대한 검정은 분산분석 표의 t-검정으로도 가능하며 결과는 동일하다.

\begin{Shaded}
\begin{Highlighting}[]
\NormalTok{fullmodel }\OtherTok{\textless{}{-}}\NormalTok{ price }\SpecialCharTok{\textasciitilde{}}\NormalTok{ year }\SpecialCharTok{+}\NormalTok{ mileage }\SpecialCharTok{+}\NormalTok{ cc }\SpecialCharTok{+}\NormalTok{ automatic}
\NormalTok{reducemodel1 }\OtherTok{\textless{}{-}}\NormalTok{ price }\SpecialCharTok{\textasciitilde{}}\NormalTok{ year }\SpecialCharTok{+}\NormalTok{ mileage }\SpecialCharTok{+}\NormalTok{ automatic}
\NormalTok{fitfull }\OtherTok{\textless{}{-}} \FunctionTok{lm}\NormalTok{(fullmodel, }\AttributeTok{data=}\NormalTok{usedcars)}
\NormalTok{fitreduce1 }\OtherTok{\textless{}{-}} \FunctionTok{lm}\NormalTok{(reducemodel1, }\AttributeTok{data=}\NormalTok{usedcars)}
\FunctionTok{anova}\NormalTok{(fitreduce1, fitfull)}
\end{Highlighting}
\end{Shaded}

\begin{verbatim}
## Analysis of Variance Table
## 
## Model 1: price ~ year + mileage + automatic
## Model 2: price ~ year + mileage + cc + automatic
##   Res.Df    RSS Df Sum of Sq      F  Pr(>F)  
## 1     26 293332                              
## 2     25 255538  1     37794 3.6975 0.06596 .
## ---
## Signif. codes:  0 '***' 0.001 '**' 0.01 '*' 0.05 '.' 0.1 ' ' 1
\end{verbatim}

이제 두 개 이상
의 변수에 대하여 부분 F 검정을 해보자.

\begin{Shaded}
\begin{Highlighting}[]
\NormalTok{reducemodel2 }\OtherTok{\textless{}{-}}\NormalTok{ price }\SpecialCharTok{\textasciitilde{}}\NormalTok{ year }\SpecialCharTok{+}\NormalTok{ mileage}
\NormalTok{fitreduce2 }\OtherTok{\textless{}{-}} \FunctionTok{lm}\NormalTok{(reducemodel2, }\AttributeTok{data=}\NormalTok{usedcars)}
\FunctionTok{anova}\NormalTok{(fitreduce2, fitfull)}
\end{Highlighting}
\end{Shaded}

\begin{verbatim}
## Analysis of Variance Table
## 
## Model 1: price ~ year + mileage
## Model 2: price ~ year + mileage + cc + automatic
##   Res.Df    RSS Df Sum of Sq      F    Pr(>F)    
## 1     27 483775                                  
## 2     25 255538  2    228237 11.165 0.0003429 ***
## ---
## Signif. codes:  0 '***' 0.001 '**' 0.01 '*' 0.05 '.' 0.1 ' ' 1
\end{verbatim}

\hypertarget{uxc120uxd615-uxac00uxc124uxc5d0-uxb300uxd55c-uxac80uxc815}{%
\section{선형 가설에 대한 검정}\label{uxc120uxd615-uxac00uxc124uxc5d0-uxb300uxd55c-uxac80uxc815}}

다음과 같은 선형 가설을 생각자.

\[ H_0: \bm L \bm \beta= \bm 0\]

예제 4.4 에서 다음과 같은 가설을 고려한다.

\[ H_0: \beta_2=0, \beta_3= 2.5 \beta_4 \]

\begin{Shaded}
\begin{Highlighting}[]
\NormalTok{modreduce }\OtherTok{\textless{}{-}} \FunctionTok{lm}\NormalTok{(suneung }\SpecialCharTok{\textasciitilde{}}\NormalTok{ kor }\SpecialCharTok{+} \FunctionTok{I}\NormalTok{(}\FloatTok{2.5}\SpecialCharTok{*}\NormalTok{math }\SpecialCharTok{+}\NormalTok{ sci), }\AttributeTok{data=}\NormalTok{suneung)}
\NormalTok{modfull }\OtherTok{\textless{}{-}} \FunctionTok{lm}\NormalTok{(suneung }\SpecialCharTok{\textasciitilde{}}\NormalTok{ kor }\SpecialCharTok{+}\NormalTok{ eng }\SpecialCharTok{+}\NormalTok{ math }\SpecialCharTok{+}\NormalTok{ sci, }\AttributeTok{data=}\NormalTok{suneung)}
\FunctionTok{anova}\NormalTok{(modreduce, modfull)}
\end{Highlighting}
\end{Shaded}

\begin{verbatim}
## Analysis of Variance Table
## 
## Model 1: suneung ~ kor + I(2.5 * math + sci)
## Model 2: suneung ~ kor + eng + math + sci
##   Res.Df    RSS Df Sum of Sq      F Pr(>F)
## 1     22 3136.4                           
## 2     20 3023.5  2    112.95 0.3736  0.693
\end{verbatim}

위의 검정은 다음과 과 같이 선형행렬 \(L\)을 정의하고 함수 \texttt{car::linearHypothesis}를 이용한 결과와 같다.

\[
H_0: 
\begin{bmatrix}
0 & 0  & 1 & 0 & 0  \\
0 & 0  & 0 & 1 &-2.5 
\end{bmatrix}
\begin{bmatrix}
\beta_0 \\
\beta_1 \\
\beta_2 \\
\beta_3 \\
\beta_4 
\end{bmatrix}
=\bm 0
\]

\begin{Shaded}
\begin{Highlighting}[]
\NormalTok{L }\OtherTok{\textless{}{-}} \FunctionTok{matrix}\NormalTok{(}\FunctionTok{c}\NormalTok{(}\DecValTok{0}\NormalTok{,}\DecValTok{0}\NormalTok{,}\DecValTok{1}\NormalTok{,}\DecValTok{0}\NormalTok{,}\DecValTok{0}\NormalTok{,}\DecValTok{0}\NormalTok{,}\DecValTok{0}\NormalTok{,}\DecValTok{0}\NormalTok{,}\DecValTok{1}\NormalTok{,}\SpecialCharTok{{-}}\FloatTok{2.5}\NormalTok{),}\DecValTok{2}\NormalTok{,}\DecValTok{5}\NormalTok{, }\AttributeTok{byrow=}\ConstantTok{TRUE}\NormalTok{)}
\NormalTok{L}
\end{Highlighting}
\end{Shaded}

\begin{verbatim}
##      [,1] [,2] [,3] [,4] [,5]
## [1,]    0    0    1    0  0.0
## [2,]    0    0    0    1 -2.5
\end{verbatim}

\begin{Shaded}
\begin{Highlighting}[]
\FunctionTok{linearHypothesis}\NormalTok{(modfull, }\AttributeTok{hypothesis.matrix=}\NormalTok{L)}
\end{Highlighting}
\end{Shaded}

\begin{verbatim}
## Linear hypothesis test
## 
## Hypothesis:
## eng = 0
## math - 2.5 sci = 0
## 
## Model 1: restricted model
## Model 2: suneung ~ kor + eng + math + sci
## 
##   Res.Df    RSS Df Sum of Sq      F Pr(>F)
## 1     22 3136.4                           
## 2     20 3023.5  2    112.95 0.3736  0.693
\end{verbatim}

\hypertarget{uxbcc0uxc218uxbcc0uxd658}{%
\section{변수변환}\label{uxbcc0uxc218uxbcc0uxd658}}

로그변화을 고려해 보자.

\begin{Shaded}
\begin{Highlighting}[]
\FunctionTok{plot}\NormalTok{(y}\SpecialCharTok{\textasciitilde{}}\NormalTok{time, regbook}\SpecialCharTok{::}\NormalTok{bug)}
\end{Highlighting}
\end{Shaded}

\includegraphics{lmpractice_files/figure-latex/unnamed-chunk-36-1.pdf}

\begin{Shaded}
\begin{Highlighting}[]
\NormalTok{bug2 }\OtherTok{\textless{}{-}}\NormalTok{regbook}\SpecialCharTok{::}\NormalTok{bug}
\NormalTok{bug2}\SpecialCharTok{$}\NormalTok{logy }\OtherTok{\textless{}{-}} \FunctionTok{log}\NormalTok{(bug2}\SpecialCharTok{$}\NormalTok{y)}
\NormalTok{fitlog }\OtherTok{\textless{}{-}} \FunctionTok{lm}\NormalTok{(logy}\SpecialCharTok{\textasciitilde{}}\NormalTok{time, bug2)}
\FunctionTok{plot}\NormalTok{(logy}\SpecialCharTok{\textasciitilde{}}\NormalTok{time, bug2)}
\FunctionTok{abline}\NormalTok{(fitlog)}
\end{Highlighting}
\end{Shaded}

\includegraphics{lmpractice_files/figure-latex/unnamed-chunk-36-2.pdf}

Box-Cox 변환은 다음과 같이 수행한다. 패키지 \texttt{MASS} 의 함수 \texttt{boxcox} 를 이용한다.

\begin{itemize}
\tightlist
\item
  \texttt{foot}:발길이(mm), 양말을 벗은 상태로 측정하였고 오른쪽 발만 측정하였다.
\item
  \texttt{forearm}: 팔안쪽길이(mm), 손목부터 팔꿈치가 접히는 부분까지의 길이이다. 오른쪽 팔만 측정하였다.
\end{itemize}

\begin{Shaded}
\begin{Highlighting}[]
\FunctionTok{plot}\NormalTok{(foot }\SpecialCharTok{\textasciitilde{}}\NormalTok{ forearm, }\AttributeTok{data=}\NormalTok{aflength)}
\end{Highlighting}
\end{Shaded}

\includegraphics{lmpractice_files/figure-latex/unnamed-chunk-37-1.pdf}

\begin{Shaded}
\begin{Highlighting}[]
\FunctionTok{boxcox}\NormalTok{(}\FunctionTok{lm}\NormalTok{(foot }\SpecialCharTok{\textasciitilde{}}\NormalTok{ forearm, }\AttributeTok{data=}\NormalTok{aflength))}
\end{Highlighting}
\end{Shaded}

\includegraphics{lmpractice_files/figure-latex/unnamed-chunk-37-2.pdf}
- 예제 4.11

\begin{Shaded}
\begin{Highlighting}[]
\NormalTok{woolfm1 }\OtherTok{\textless{}{-}} \FunctionTok{lm}\NormalTok{(cycle}\SpecialCharTok{\textasciitilde{}}\NormalTok{length }\SpecialCharTok{+}\NormalTok{ amplitude }\SpecialCharTok{+}\NormalTok{ load, }\AttributeTok{data=}\NormalTok{wool)}
\FunctionTok{plot}\NormalTok{(woolfm1)}
\end{Highlighting}
\end{Shaded}

\includegraphics{lmpractice_files/figure-latex/unnamed-chunk-38-1.pdf} \includegraphics{lmpractice_files/figure-latex/unnamed-chunk-38-2.pdf} \includegraphics{lmpractice_files/figure-latex/unnamed-chunk-38-3.pdf} \includegraphics{lmpractice_files/figure-latex/unnamed-chunk-38-4.pdf}

\begin{Shaded}
\begin{Highlighting}[]
\NormalTok{wool}\SpecialCharTok{$}\NormalTok{logcycle }\OtherTok{\textless{}{-}} \FunctionTok{log}\NormalTok{(wool}\SpecialCharTok{$}\NormalTok{cycle)}
\FunctionTok{boxcox}\NormalTok{(woolfm1)}
\end{Highlighting}
\end{Shaded}

\includegraphics{lmpractice_files/figure-latex/unnamed-chunk-38-5.pdf}

\begin{Shaded}
\begin{Highlighting}[]
\NormalTok{woolfm2 }\OtherTok{\textless{}{-}} \FunctionTok{lm}\NormalTok{(logcycle}\SpecialCharTok{\textasciitilde{}}\NormalTok{length }\SpecialCharTok{+}\NormalTok{ amplitude }\SpecialCharTok{+}\NormalTok{ load, }\AttributeTok{data=}\NormalTok{wool)}
\FunctionTok{plot}\NormalTok{(woolfm2)}
\end{Highlighting}
\end{Shaded}

\includegraphics{lmpractice_files/figure-latex/unnamed-chunk-38-6.pdf} \includegraphics{lmpractice_files/figure-latex/unnamed-chunk-38-7.pdf} \includegraphics{lmpractice_files/figure-latex/unnamed-chunk-38-8.pdf} \includegraphics{lmpractice_files/figure-latex/unnamed-chunk-38-9.pdf}

\hypertarget{uxb2e4uxc911uxacf5uxc120uxc131}{%
\section{다중공선성}\label{uxb2e4uxc911uxacf5uxc120uxc131}}

\hypertarget{uxace0uxc720uxac12uxacfc-uxace0uxc720uxbca1uxd130uxc5d0-uxb300uxd55c-uxc774uxb860}{%
\subsection{고유값과 고유벡터에 대한 이론}\label{uxace0uxc720uxac12uxacfc-uxace0uxc720uxbca1uxd130uxc5d0-uxb300uxd55c-uxc774uxb860}}

대칭행렬 \(\bm A = \bm X^t\bm X\)의 고유값 \(\lambda_i\)와 그에 대응하는 고유벡터
\(\bm v_i\)는 다음을 만족하는 실수와 벡터이다.

\[ \bm A \bm  v_i = \lambda_i \bm  v_i \]

고유값 \(\lambda_i\)을 구하는 방법은 다음의 방정식을 만족하는 해를 구하는 것이다.

\[ det \left ( \bm A - \lambda_i \bm I \right ) = 0\]

여기서 \(det(\bm A)\)는 행렬 \(\bm A\)의 행렬식을 의미한다.

\(\lambda_1 \ge \lambda_2 \ge \dots \ge \lambda_{p-1}\)를 \(\bm X^t \bm X\)의 고유값이라고 하자. \(\bm X^t \bm X\)의 각 고유값에 대한 정규직교 고유벡터(orthonormal eigenvector)를 \(\bm v_1, \bm v_2,\dots,\bm v_{p-1}\)라고 하자, 즉

\[ \bm  v_i^t \bm  v_i = 1 , \quad \bm  v_i^t \bm  v_j = 0 \quad (i \ne j) \]

더 나아가 행렬 \(\bm V\)를 고유벡터를 모아놓은 행렬로 정의하자.

\[ \bm V=[\bm v_1 ~ \bm v_2 ~\dots ~ \bm v_{p-1} ] \]

이때 \((p-1) \times (p-1)\) 차원의 행렬 \(\bm V\)는 직교행렬이다.

\[ \bm V^t \bm V =\bm V  \bm V^t =I \]

이제 다음과 같이 \(\bm X^t \bm X\)를 나타낼 수 있다.

\[ \bm V^t (\bm X^t \bm X) \bm V = \text{diag}(\lambda_1 , \lambda_2 , \dots , \lambda_{p-1}) = \bm \Lambda \]

또한

\[ \bm V^t (\bm X^t \bm X)^{-1} \bm V = \text{diag} \left (\frac{1}{\lambda_1} , \frac{1}{\lambda_2} , \dots , \frac{1}{\lambda_{p-1}} \right ) = \bm \Lambda^{-1} \]

위의 식에서 알 수 있듯이 \(1/\lambda_i\)는 \((\bm X^t \bm X)^{-1}\)의 고유값이다.

행렬 \(\bm V\)가 직교행렬이기 때문에 다음과 같은 표현도 가능하다.

\[ (\bm X^t \bm X) =  \bm V \bm \Lambda \bm V^t, 
\quad (\bm X^t \bm X)^{-1} =  \bm V \bm \Lambda^{-1} \bm V^t  \]

따라서 고유값 \(\lambda_k\)이 매우 0에 가까우면 다음이 성립하고

\[ \bm v_k^t (\bm X^t \bm X) \bm v_k = (\bm X \bm v_k)^t ( \bm X \bm v_k) \approx 0   \]
위의 식은 다음과 같이 행렬 \(\bm X\)의 열들간에 선형관계가 있다는 것을 의미한다.

\[  v_{1k} \bm X_1 +  v_{2k} \bm X_2 + \dots  v_{p-1,k} \bm X_k \approx 0 \]

위에서 \(\bm v_k\)와 \(\bm X\)는 다음과 같이 표시한다.

\[ \bm X=[\bm X_1~ \bm X_2~ \dots~\bm X_{p-1}], \quad 
\bm v_k = [ v_{1k},  v_{2k}, \cdots,  v_{p-1,k}]^t \]

또한 회귀계수 벡터 \(\hat \beta\)의 공분산 행렬이 다음과 같이 주어지므로

\begin{equation} 
Cov(\hat {\bm \beta}) = \sigma^2 (\bm X^t \bm X)^{-1} = \sigma^2  \bm V \bm \Lambda^{-1} \bm V^t 
\label{eq:eq1}
\end{equation}

다음과 같은 식이 성립한다.

\begin{equation} 
var(\hat \beta_j) / \sigma^2 = \frac{v^2_{j1}}{\lambda_1} + \frac{v^2_{j2}}{\lambda_2} + \dots \frac{v^2_{j, p-1}}{\lambda_{p-1}} 
\label{eq:eq2}
\end{equation}

\hypertarget{uxace0uxc720uxac12uxacfc-uxace0uxc720uxbca1uxd130uxc5d0-uxb300uxd55c-uxc608uxc81c-uxb450-uxac1cuxc758-uxb3c5uxb9bduxbcc0uxc218}{%
\subsection{고유값과 고유벡터에 대한 예제: 두 개의 독립변수}\label{uxace0uxc720uxac12uxacfc-uxace0uxc720uxbca1uxd130uxc5d0-uxb300uxd55c-uxc608uxc81c-uxb450-uxac1cuxc758-uxb3c5uxb9bduxbcc0uxc218}}

이제 다음과 두 개의 독립변수가 있는 회귀 모형을 고려해 보자.

\[ y_i = \beta_0 + \beta_1 x_{i1} + \beta_2 x_{i2} + e_i, i=1,2,\cdots,n \]

절편을 제외한 두 개의 표준화된 독립변수들로 이루어진 행렬을 \(\bm X\)로 표시하자.

\[  \bm X = [ \bm X_1 ~ \bm X_2 ]   \]

위에서 디자인 행렬 \(\bm X\)는 원래 독립변수의 디자인 행렬 \(X\)의 열들을 표준화한 변수로 구성된 것이다..

이제 \(\bm X^t \bm X\)는 두 독립변수의 상관계수 행렬임을 알 수 있다.

\[  \bm X^t \bm X =
\begin{bmatrix}
1  &  \rho \\
\rho & 1 
\end{bmatrix} 
=\bm R, \quad
0 < \rho < 1
\]

여기서 두 독립변수 \(X_1\)과 \(X_2\)의 상관계수 \(\rho\)는 0보다 크다고 가정하자.

이제 \(\bm X^t \bm X\)의 고유값(\(\lambda_i\))과 고유벡터(\(\bm v_i\))는 다음과 같은 방정식을 만족하는 수 \(\lambda_i\)와 벡터 \(\bm v_i\) 이다.

\[ (\bm X^t \bm X) \bm v_i = \lambda_i \bm v_i, \quad \bm v_i^t \bm v_i=1  \]

일단 먼저 고유값을 구하는 방법은 \(det(\bm X^t \bm X - \lambda_i \bm I ) =0\)을 만족하는
값을 찾는 것이다. 여기서 \(det(\bm A)\)는 \(\bm A\)의 행렬식을 의미한다.

\[ 
det(\bm X^t \bm X - \lambda_i \bm I ) = det \left ( 
\begin{bmatrix}
1-\lambda_i  &  \rho \\
\rho & 1-\lambda_i 
\end{bmatrix}
 \right ) =0
\]
위의 방정식은 다음과 같이 요약할 수 있고

\[ \lambda_i^2 -2 \lambda_i + (1-\rho^2) =0 \]

해는 다음과 같이 주어진다.

\[ \lambda_1 = 1+ \rho, \quad \lambda_2 = 1 -\rho \quad (\lambda_1 \ge \lambda_2) \]

이제 각 고유값에 대한 고유벡터를 구해보자. 각 고유값 \(\lambda_i\)에 대한 고유벡터를 \(\bm v_i\) 라고 하면

\[ 
\bm v_1 = 
\begin{bmatrix} 
 v_{11} \\
 v_{21}
\end{bmatrix},
~ v^2_{11}+v^2_{21}=1
\quad \quad
\bm v_2 = 
\begin{bmatrix} 
 v_{12} \\
 v_{22}
\end{bmatrix},~
v^2_{12}+v^2_{11}=1
\]

다음과 같은 방정식을 만족해야 한다.

\[ (\bm X^t \bm X) \bm v_1 = \lambda_1 \bm v_1 , \quad  (\bm X^t \bm X) \bm v_2 = \lambda_2 \bm v_2 \]

즉,

\[ 
\begin{bmatrix}
1  &  \rho \\
\rho & 1 
\end{bmatrix}
\begin{bmatrix} 
 v_{11} \\
 v_{21}
\end{bmatrix}
=
(1+ \rho)
\begin{bmatrix} 
 v_{11} \\
 v_{21}
\end{bmatrix}
, \quad 
\begin{bmatrix}
1  &  \rho \\
\rho & 1 
\end{bmatrix}
\begin{bmatrix} 
 v_{12} \\
 v_{22}
\end{bmatrix}
=
(1- \rho)
\begin{bmatrix} 
 v_{12} \\
 v_{22}
\end{bmatrix}
\]

위의 두 방정식은 정리하면 다음과 더 단순한 방정식을 얻는다.

\[  v_{11} -  v_{21} = 0, \quad  v_{12}+ v_{22}=0 \]

이제 위의 식을 만족하고 길이가 1인 두 벡터를 찾으면 다음과 같은 두 개의 직교하고 길이가 1인 고유벡터 \(\bm v_1\)과 \(\bm v_2\)를 찾을 수 있다.

\[ 
\bm v_1 = 
\begin{bmatrix} 
 v_{11} \\
 v_{21}
\end{bmatrix}
= 
\begin{bmatrix} 
1/\sqrt{2} \\
1/\sqrt{2}
\end{bmatrix},
\quad \quad
\bm v_2 = 
\begin{bmatrix} 
 v_{12} \\
 v_{22}
\end{bmatrix}
= 
\begin{bmatrix} 
1/\sqrt{2} \\
-1/\sqrt{2}
\end{bmatrix}
\]

따라서 앞 절의 이론에서 나온 고유벡터로 구성된 행렬 \(\bm V\)와 고유값을 대각원소로 하는 행렬 \(\bm \Lambda\)는 다음과 같다.

\[ 
\bm V = [\bm v_1~ \bm v_2]
= \begin{bmatrix} 
 v_{11} &  v_{12}\\
 v_{21} &  v_{22}
\end{bmatrix}
= 
\begin{bmatrix} 
1/\sqrt{2} &  1/\sqrt{2}\\
1/\sqrt{2} &  -1/\sqrt{2}
\end{bmatrix},
\quad \quad
\bm \Lambda =
 \begin{bmatrix} 
\lambda_1 & 0 \\
0 & \lambda_2
\end{bmatrix}
= 
 \begin{bmatrix} 
1+\rho & 0 \\
0 & 1-\rho
\end{bmatrix}
\]

이제 다음이 성립함을 확인할 수 있다.

\[ \bm V^t (\bm X^t \bm X) \bm V =  \bm \Lambda, \quad (\bm X^t \bm X)^{-1} =  \bm V \bm \Lambda^{-1} \bm V^t   \]

즉,

\begin{align*}
\bm V^t (\bm X^t \bm X) \bm V 
& =  
\begin{bmatrix} 
1/\sqrt{2} &  1/\sqrt{2}\\
1/\sqrt{2} &  -1/\sqrt{2}
\end{bmatrix}
\begin{bmatrix} 
1 & \rho \\
\rho & 1
\end{bmatrix}
\begin{bmatrix} 
1/\sqrt{2} &  1/\sqrt{2}\\
1/\sqrt{2} &  -1/\sqrt{2}
\end{bmatrix} \\
& =
 \begin{bmatrix} 
1+\rho & 0 \\
0 & 1-\rho
\end{bmatrix} \\
&=
\bm \Lambda
\end{align*}

또한 다음도 성립함을 확인할 수 있다.

\[  (\bm X^t \bm X)^{-1} =  \bm V \bm \Lambda^{-1} \bm V^t   \]

즉,

\begin{align*}
(\bm X^t \bm X)^{-1} & =  \bm V \bm \Lambda^{-1} \bm V^t \\   
 &  = 
 \begin{bmatrix} 
 v_{11} &  v_{12}\\
 v_{21} &  v_{22}
\end{bmatrix}
 \begin{bmatrix} 
\frac{1}{\lambda_1} & 0 \\
0 & \frac{1}{\lambda_2}
\end{bmatrix} 
\begin{bmatrix} 
 v_{11} &  v_{21}\\
 v_{12} &  v_{22}
\end{bmatrix} \\
&= 
\begin{bmatrix} 
1/\sqrt{2} &  1/\sqrt{2}\\
1/\sqrt{2} &  -1/\sqrt{2}
\end{bmatrix}
 \begin{bmatrix} 
\frac{1}{1+\rho} & 0 \\
0 & \frac{1}{1-\rho}
\end{bmatrix}
\begin{bmatrix} 
1/\sqrt{2} &  1/\sqrt{2}\\
1/\sqrt{2} &  -1/\sqrt{2}
\end{bmatrix} \\
&=
 \begin{bmatrix} 
 v_{11}^2 \frac{1}{\lambda_1} +   v_{12}^2 \frac{1}{\lambda_2} &
 v_{11}  v_{21} \frac{1}{\lambda_1} +  v_{12}  v_{22} \frac{1}{\lambda_2} \\
 v_{11}  v_{21} \frac{1}{\lambda_1} +  v_{12}  v_{22} \frac{1}{\lambda_2}   & 
 v_{21}^2 \frac{1}{\lambda_1} +   v_{22}^2 \frac{1}{\lambda_2}
\end{bmatrix} \\
&= 
\begin{bmatrix} 
(\frac{1}{\sqrt{2}})^2 \frac{1}{1+\rho} +  (\frac{1}{\sqrt{2}})^2 \frac{1}{1-\rho} &
(\frac{1}{\sqrt{2}})^2 \frac{1}{1+\rho} + (\frac{1}{\sqrt{2}}) (-\frac{1}{\sqrt{2}}) \frac{1}{1-\rho} \\
(\frac{1}{\sqrt{2}})^2 \frac{1}{1+\rho} + (\frac{1}{\sqrt{2}}) (-\frac{1}{\sqrt{2}}) \frac{1}{1-\rho}  & 
(\frac{1}{\sqrt{2}})^2 \frac{1}{1+\rho} +  (-\frac{1}{\sqrt{2}})^2 \frac{1}{1-\rho} 
\end{bmatrix} \\
& =
\frac{1}{1-\rho^2}
\begin{bmatrix}
1  & -\rho \\
-\rho & 1
\end{bmatrix}
\end{align*}

앞 절에서 나온 회귀계수 추정량의 분산 공식 \eqref{eq:eq1} 과 \eqref{eq:eq2} 를 적용하면 다음과 같은 식을 얻을 수 있다.

\begin{align*}
Var(\hat \beta_j)/\sigma^2 
& = \frac{v^2_{j1}}{\lambda_1} + \frac{v^2_{j2}}{\lambda_2} \\
& = \frac{1}{2} \left ( \frac{1}{1+\rho} + \frac{1}{1-\rho} \right ) \\
& = \frac{1}{1-\rho^2}
\end{align*}

위의 분산 공식에서 제일 작은 두 번째 고유값 \(\lambda_2 = 1- \rho\)가 0에 가까우면 분산이 매우 커지는 것을 알 수 있다. 이 고유값은 상관계수 \(\rho\)가 1에 가까울 수록 0에 가까워 진다.

\hypertarget{uxc911uxace0uxcc28-uxc608uxc81c}{%
\subsection{중고차 예제}\label{uxc911uxace0uxcc28-uxc608uxc81c}}

\begin{Shaded}
\begin{Highlighting}[]
\NormalTok{usedcars2 }\OtherTok{\textless{}{-}}\NormalTok{ usedcars }\SpecialCharTok{\%\textgreater{}\%}  \FunctionTok{mutate}\NormalTok{(}\AttributeTok{ccmile =}\NormalTok{ cc }\SpecialCharTok{+}\NormalTok{ mileage)}
\NormalTok{fitcoll1 }\OtherTok{\textless{}{-}} \FunctionTok{lm}\NormalTok{(price }\SpecialCharTok{\textasciitilde{}}\NormalTok{ year }\SpecialCharTok{+}\NormalTok{ mileage }\SpecialCharTok{+}\NormalTok{ cc }\SpecialCharTok{+}\NormalTok{ automatic }\SpecialCharTok{+}\NormalTok{ ccmile, usedcars2)}
\FunctionTok{summary}\NormalTok{(fitcoll1)}
\end{Highlighting}
\end{Shaded}

\begin{verbatim}
## 
## Call:
## lm(formula = price ~ year + mileage + cc + automatic + ccmile, 
##     data = usedcars2)
## 
## Residuals:
##     Min      1Q  Median      3Q     Max 
## -177.35  -63.91   -0.99   70.34  212.69 
## 
## Coefficients: (1 not defined because of singularities)
##               Estimate Std. Error t value Pr(>|t|)    
## (Intercept)  5.253e+02  3.998e+02   1.314 0.200823    
## year        -5.800e+00  9.283e-01  -6.247 1.55e-06 ***
## mileage     -2.263e-03  7.211e-04  -3.138 0.004324 ** 
## cc           3.888e-01  2.022e-01   1.923 0.065958 .  
## automatic    1.653e+02  3.986e+01   4.147 0.000339 ***
## ccmile              NA         NA      NA       NA    
## ---
## Signif. codes:  0 '***' 0.001 '**' 0.01 '*' 0.05 '.' 0.1 ' ' 1
## 
## Residual standard error: 101.1 on 25 degrees of freedom
## Multiple R-squared:  0.9045, Adjusted R-squared:  0.8892 
## F-statistic: 59.21 on 4 and 25 DF,  p-value: 2.184e-12
\end{verbatim}

\hypertarget{uxc608uxc81c-4.14}{%
\subsection{예제 4.14}\label{uxc608uxc81c-4.14}}

모형을 적합해 보자.

\begin{Shaded}
\begin{Highlighting}[]
\NormalTok{hald.lm }\OtherTok{\textless{}{-}} \FunctionTok{lm}\NormalTok{(y}\SpecialCharTok{\textasciitilde{}}\NormalTok{ ., }\AttributeTok{data=}\NormalTok{hald)}
\FunctionTok{summary}\NormalTok{(hald.lm)}
\end{Highlighting}
\end{Shaded}

\begin{verbatim}
## 
## Call:
## lm(formula = y ~ ., data = hald)
## 
## Residuals:
##     Min      1Q  Median      3Q     Max 
## -3.1750 -1.6709  0.2508  1.3783  3.9254 
## 
## Coefficients:
##             Estimate Std. Error t value Pr(>|t|)  
## (Intercept)  62.4054    70.0710   0.891   0.3991  
## x1            1.5511     0.7448   2.083   0.0708 .
## x2            0.5102     0.7238   0.705   0.5009  
## x3            0.1019     0.7547   0.135   0.8959  
## x4           -0.1441     0.7091  -0.203   0.8441  
## ---
## Signif. codes:  0 '***' 0.001 '**' 0.01 '*' 0.05 '.' 0.1 ' ' 1
## 
## Residual standard error: 2.446 on 8 degrees of freedom
## Multiple R-squared:  0.9824, Adjusted R-squared:  0.9736 
## F-statistic: 111.5 on 4 and 8 DF,  p-value: 4.756e-07
\end{verbatim}

상관계수 행렬의 고유값을 계산해 보자.

\begin{Shaded}
\begin{Highlighting}[]
\NormalTok{R }\OtherTok{\textless{}{-}} \FunctionTok{cor}\NormalTok{(hald[}\DecValTok{2}\SpecialCharTok{:}\DecValTok{5}\NormalTok{])}
\NormalTok{R}
\end{Highlighting}
\end{Shaded}

\begin{verbatim}
##            x1         x2         x3         x4
## x1  1.0000000  0.2285795 -0.8241338 -0.2454451
## x2  0.2285795  1.0000000 -0.1392424 -0.9729550
## x3 -0.8241338 -0.1392424  1.0000000  0.0295370
## x4 -0.2454451 -0.9729550  0.0295370  1.0000000
\end{verbatim}

\begin{Shaded}
\begin{Highlighting}[]
\FunctionTok{solve}\NormalTok{(R)}
\end{Highlighting}
\end{Shaded}

\begin{verbatim}
##          x1        x2        x3       x4
## x1 38.49621  94.11969  41.88410  99.7858
## x2 94.11969 254.42317 105.09139 267.5394
## x3 41.88410 105.09139  46.86839 111.1451
## x4 99.78580 267.53942 111.14509 282.5129
\end{verbatim}

\begin{Shaded}
\begin{Highlighting}[]
\FunctionTok{diag}\NormalTok{(}\FunctionTok{solve}\NormalTok{(R))}
\end{Highlighting}
\end{Shaded}

\begin{verbatim}
##        x1        x2        x3        x4 
##  38.49621 254.42317  46.86839 282.51286
\end{verbatim}

\begin{Shaded}
\begin{Highlighting}[]
\NormalTok{eigenval }\OtherTok{\textless{}{-}} \FunctionTok{eigen}\NormalTok{(R)}\SpecialCharTok{$}\NormalTok{values}
\NormalTok{eigenval}
\end{Highlighting}
\end{Shaded}

\begin{verbatim}
## [1] 2.235704035 1.576066070 0.186606149 0.001623746
\end{verbatim}

\begin{Shaded}
\begin{Highlighting}[]
\FunctionTok{sqrt}\NormalTok{(}\FunctionTok{max}\NormalTok{(eigenval)}\SpecialCharTok{/}\NormalTok{eigenval)}
\end{Highlighting}
\end{Shaded}

\begin{verbatim}
## [1]  1.000000  1.191022  3.461339 37.106342
\end{verbatim}

VIF를 구해보자.

\begin{Shaded}
\begin{Highlighting}[]
\NormalTok{car}\SpecialCharTok{::}\FunctionTok{vif}\NormalTok{(hald.lm)}
\end{Highlighting}
\end{Shaded}

\begin{verbatim}
##        x1        x2        x3        x4 
##  38.49621 254.42317  46.86839 282.51286
\end{verbatim}

\begin{Shaded}
\begin{Highlighting}[]
\FunctionTok{summary}\NormalTok{(regbook}\SpecialCharTok{::}\FunctionTok{vif}\NormalTok{(hald.lm))}
\end{Highlighting}
\end{Shaded}

\begin{verbatim}
## 
## VIF:
##     x1     x2     x3     x4 
##  38.50 254.42  46.87 282.51 
## 
## Variance Proportion:
##   Eigenvalues Cond.Index          x1           x2          x3           x4
## 1 2.235704035   1.000000 0.002632084 0.0005589686 0.001481988 0.0004753347
## 2 1.576066070   1.191022 0.004269804 0.0004272931 0.004954638 0.0004572915
## 3 0.186606149   3.461339 0.063519491 0.0020822791 0.046495910 0.0007243995
## 4 0.001623746  37.106342 0.929578621 0.9969314592 0.947067464 0.9983429744
\end{verbatim}

\(x_2\)를 제외하고 분석해 보자.

\begin{Shaded}
\begin{Highlighting}[]
\NormalTok{hald.lm2 }\OtherTok{\textless{}{-}} \FunctionTok{lm}\NormalTok{(y}\SpecialCharTok{\textasciitilde{}}\NormalTok{ x1 }\SpecialCharTok{+}\NormalTok{ x3 }\SpecialCharTok{+}\NormalTok{ x4, }\AttributeTok{data=}\NormalTok{hald)}
\FunctionTok{summary}\NormalTok{(hald.lm2)}
\end{Highlighting}
\end{Shaded}

\begin{verbatim}
## 
## Call:
## lm(formula = y ~ x1 + x3 + x4, data = hald)
## 
## Residuals:
##     Min      1Q  Median      3Q     Max 
## -2.9323 -1.8090  0.4806  1.1398  3.7771 
## 
## Coefficients:
##              Estimate Std. Error t value Pr(>|t|)    
## (Intercept) 111.68441    4.56248  24.479 1.52e-09 ***
## x1            1.05185    0.22368   4.702  0.00112 ** 
## x3           -0.41004    0.19923  -2.058  0.06969 .  
## x4           -0.64280    0.04454 -14.431 1.58e-07 ***
## ---
## Signif. codes:  0 '***' 0.001 '**' 0.01 '*' 0.05 '.' 0.1 ' ' 1
## 
## Residual standard error: 2.377 on 9 degrees of freedom
## Multiple R-squared:  0.9813, Adjusted R-squared:  0.975 
## F-statistic: 157.3 on 3 and 9 DF,  p-value: 4.312e-08
\end{verbatim}

\begin{Shaded}
\begin{Highlighting}[]
\FunctionTok{summary}\NormalTok{(regbook}\SpecialCharTok{::}\FunctionTok{vif}\NormalTok{(hald.lm2))}
\end{Highlighting}
\end{Shaded}

\begin{verbatim}
## 
## VIF:
##    x1    x3    x4 
## 3.678 3.460 1.181 
## 
## Variance Proportion:
##   Eigenvalues Cond.Index           x1         x3         x4
## 1   1.8683737   1.000000 0.0720157120 0.07053018 0.02229687
## 2   0.9838532   1.378056 0.0002285765 0.02382939 0.79011946
## 3   0.1477731   3.555775 0.9277557115 0.90564042 0.18758367
\end{verbatim}

\hypertarget{compute}{%
\chapter{최소제곱의 계산법}\label{compute}}

이제 선형모형에서 회귀게수를 구하는 계산 방법에 대하여 알아보자. 최소제곱법(동시에 최대 가능도 추정법)에 의한 회귀게수의 추정치를 구하려면 다음과 같은 정규 방정식(normal equation)을 풀어야 한다.

\begin{equation}
 {\bm X}^t \bm X \bm \beta = \bm X^t \bm y
\label{eq:comp-lse}
\end{equation}

중고차 자료를 이용한다.

\begin{Shaded}
\begin{Highlighting}[]
\FunctionTok{head}\NormalTok{(usedcars)}
\end{Highlighting}
\end{Shaded}

\begin{verbatim}
##   price year mileage   cc automatic
## 1   790   78  133462 1998         1
## 2  1380   39   33000 2000         1
## 3   270  109  120000 1800         0
## 4  1190   20   69727 1999         1
## 5   590   70  112000 2000         0
## 6  1120   58   39106 1998         1
\end{verbatim}

\begin{Shaded}
\begin{Highlighting}[]
\NormalTok{fit0 }\OtherTok{\textless{}{-}} \FunctionTok{lm}\NormalTok{(price }\SpecialCharTok{\textasciitilde{}}\NormalTok{ year }\SpecialCharTok{+}\NormalTok{ mileage }\SpecialCharTok{+}\NormalTok{ cc }\SpecialCharTok{+}\NormalTok{ automatic, }\AttributeTok{data=}\NormalTok{usedcars)}
\NormalTok{lmbeta }\OtherTok{\textless{}{-}}\NormalTok{ fit0}\SpecialCharTok{$}\NormalTok{coefficients}
\NormalTok{lmbeta}
\end{Highlighting}
\end{Shaded}

\begin{verbatim}
##   (Intercept)          year       mileage            cc     automatic 
## 525.286960604  -5.799637101  -0.002262844   0.388787346 165.312632517
\end{verbatim}

계획행렬은 다음과 같이 구할 수 있다.

\begin{Shaded}
\begin{Highlighting}[]
\NormalTok{X }\OtherTok{\textless{}{-}} \FunctionTok{model.matrix}\NormalTok{(fit0)}
\NormalTok{y }\OtherTok{\textless{}{-}} \FunctionTok{as.matrix}\NormalTok{(usedcars}\SpecialCharTok{$}\NormalTok{price)}
\NormalTok{X}
\end{Highlighting}
\end{Shaded}

\begin{verbatim}
##    (Intercept) year mileage   cc automatic
## 1            1   78  133462 1998         1
## 2            1   39   33000 2000         1
## 3            1  109  120000 1800         0
## 4            1   20   69727 1999         1
## 5            1   70  112000 2000         0
## 6            1   58   39106 1998         1
## 7            1   53   95935 1800         1
## 8            1   68  120000 1800         0
## 9            1   15   20215 1798         1
## 10           1   96  140000 1800         0
## 11           1   63   68924 1998         1
## 12           1   82   90000 2000         0
## 13           1   76   81279 1998         0
## 14           1   17   24070 1798         1
## 15           1   38   40000 2000         0
## 16           1   46   56887 1832         1
## 17           1   95   91216 1997         1
## 18           1   37   48680 1998         1
## 19           1   68    8000 2000         0
## 20           1   41   60634 1835         1
## 21           1   69  114131 1998         1
## 22           1   71   75000 1800         0
## 23           1   99  124417 1998         1
## 24           1  129  130000 1800         0
## 25           1   57   77559 1997         1
## 26           1  107   75216 1838         1
## 27           1   45   52000 2000         0
## 28           1   80   58000 2000         1
## 29           1  113  134500 1800         0
## 30           1   41   80000 2000         0
## attr(,"assign")
## [1] 0 1 2 3 4
\end{verbatim}

\begin{Shaded}
\begin{Highlighting}[]
\NormalTok{y}
\end{Highlighting}
\end{Shaded}

\begin{verbatim}
##       [,1]
##  [1,]  790
##  [2,] 1380
##  [3,]  270
##  [4,] 1190
##  [5,]  590
##  [6,] 1120
##  [7,]  815
##  [8,]  450
##  [9,] 1290
## [10,]  420
## [11,]  945
## [12,]  770
## [13,]  610
## [14,] 1350
## [15,] 1020
## [16,]  830
## [17,]  670
## [18,]  990
## [19,]  800
## [20,] 1100
## [21,]  740
## [22,]  570
## [23,]  660
## [24,]  300
## [25,]  960
## [26,]  650
## [27,] 1000
## [28,]  700
## [29,]  280
## [30,]  879
\end{verbatim}

\begin{Shaded}
\begin{Highlighting}[]
\NormalTok{A }\OtherTok{\textless{}{-}} \FunctionTok{t}\NormalTok{(X) }\SpecialCharTok{\%*\%}\NormalTok{ X}
\NormalTok{Xty }\OtherTok{\textless{}{-}} \FunctionTok{t}\NormalTok{(X) }\SpecialCharTok{\%*\%}\NormalTok{ y }
\NormalTok{A}
\end{Highlighting}
\end{Shaded}

\begin{verbatim}
##             (Intercept)      year      mileage         cc automatic
## (Intercept)          30      1980      2373958      57680        17
## year               1980    155858    179424383    3792473       974
## mileage         2373958 179424383 228412852144 4542340762   1191179
## cc                57680   3792473   4542340762  111163348     32882
## automatic            17       974      1191179      32882        17
\end{verbatim}

\begin{Shaded}
\begin{Highlighting}[]
\NormalTok{Xty}
\end{Highlighting}
\end{Shaded}

\begin{verbatim}
##                   [,1]
## (Intercept)      24139
## year           1365619
## mileage     1652471805
## cc            46679690
## automatic        16180
\end{verbatim}

\hypertarget{uxcd10uxb808uxc2a4uxd0a4}{%
\section{촐레스키}\label{uxcd10uxb808uxc2a4uxd0a4}}

\begin{Shaded}
\begin{Highlighting}[]
\NormalTok{U }\OtherTok{\textless{}{-}} \FunctionTok{chol}\NormalTok{(A) }\CommentTok{\# chol give upper diagonal}
\NormalTok{U}
\end{Highlighting}
\end{Shaded}

\begin{verbatim}
##             (Intercept)     year  mileage          cc  automatic
## (Intercept)    5.477226 361.4969 433423.4 10530.87904  3.1037612
## year           0.000000 158.6758 143331.0   -90.79521 -0.9327196
## mileage        0.000000   0.0000 141468.0   -63.44464 -0.1440343
## cc             0.000000   0.0000      0.0   501.66291  0.2050022
## automatic      0.000000   0.0000      0.0     0.00000  2.5365191
\end{verbatim}

\begin{Shaded}
\begin{Highlighting}[]
\NormalTok{bstar }\OtherTok{\textless{}{-}} \FunctionTok{solve}\NormalTok{(}\FunctionTok{t}\NormalTok{(U) ,Xty)}
\NormalTok{hatb\_chol }\OtherTok{\textless{}{-}} \FunctionTok{solve}\NormalTok{(U, bstar)}
\NormalTok{hatb\_chol}
\end{Highlighting}
\end{Shaded}

\begin{verbatim}
##                      [,1]
## (Intercept) 525.286960604
## year         -5.799637101
## mileage      -0.002262844
## cc            0.388787346
## automatic   165.312632517
\end{verbatim}

\hypertarget{qr}{%
\section{QR}\label{qr}}

\begin{Shaded}
\begin{Highlighting}[]
\NormalTok{QRans }\OtherTok{\textless{}{-}} \FunctionTok{qr}\NormalTok{(X)}
\NormalTok{Q1 }\OtherTok{\textless{}{-}} \FunctionTok{qr.Q}\NormalTok{(QRans)}
\NormalTok{Q1}
\end{Highlighting}
\end{Shaded}

\begin{verbatim}
##             [,1]        [,2]        [,3]       [,4]        [,5]
##  [1,] -0.1825742  0.07562591  0.30742310 -0.2027340  0.19971845
##  [2,] -0.1825742 -0.17015830 -0.15369537 -0.1039196  0.09114151
##  [3,] -0.1825742  0.27099286  0.01432404  0.1936620 -0.10728957
##  [4,] -0.1825742 -0.28989933  0.22723602 -0.1284304  0.06676071
##  [5,] -0.1825742  0.02520864  0.20679510 -0.1848696 -0.21733215
##  [6,] -0.1825742 -0.05041728 -0.23185156 -0.1117203  0.13010376
##  [7,] -0.1825742 -0.08192807  0.20178344  0.2338289  0.17106774
##  [8,] -0.1825742  0.01260432  0.27611529  0.2073189 -0.18633389
##  [9,] -0.1825742 -0.32141013 -0.09082550  0.3181650  0.07320675
## [10,] -0.1825742  0.18906478  0.23870575  0.1801128 -0.12576957
## [11,] -0.1825742 -0.01890648 -0.05300173 -0.1400423  0.14955765
## [12,] -0.1825742  0.10083455 -0.02533894 -0.1691993 -0.20143834
## [13,] -0.1825742  0.06302159 -0.04867448 -0.1554176 -0.21555404
## [14,] -0.1825742 -0.30880581 -0.07634583  0.3140525  0.07833142
## [15,] -0.1825742 -0.17646046 -0.09782906 -0.1098444 -0.30272348
## [16,] -0.1825742 -0.12604319 -0.02954052  0.2072806  0.13956469
## [17,] -0.1825742  0.18276262 -0.09975034 -0.1686365  0.21874911
## [18,] -0.1825742 -0.18276262 -0.03008727 -0.1132842  0.09276884
## [19,] -0.1825742  0.01260432 -0.51558321 -0.0912301 -0.25541870
## [20,] -0.1825742 -0.15755399  0.02887180  0.1996162  0.13067512
## [21,] -0.1825742  0.01890648  0.22824373 -0.1824548  0.17600462
## [22,] -0.1825742  0.03151080 -0.06113331  0.2465485 -0.19536153
## [23,] -0.1825742  0.20797126  0.10939817 -0.2016431  0.23722748
## [24,] -0.1825742  0.39703604 -0.04269165  0.1780603 -0.06543995
## [25,] -0.1825742 -0.05671943  0.04634772 -0.1437698  0.14099343
## [26,] -0.1825742  0.25838854 -0.28947196  0.1586158  0.26223342
## [27,] -0.1825742 -0.13234535 -0.05770029 -0.1229037 -0.28527840
## [28,] -0.1825742  0.08823023 -0.23876821 -0.1399259  0.17841437
## [29,] -0.1825742  0.29620149  0.09128011  0.1793670 -0.09480537
## [30,] -0.1825742 -0.15755399  0.16576495 -0.1466026 -0.28377408
\end{verbatim}

\begin{Shaded}
\begin{Highlighting}[]
\NormalTok{R }\OtherTok{\textless{}{-}} \FunctionTok{qr.R}\NormalTok{(QRans)}
\NormalTok{R}
\end{Highlighting}
\end{Shaded}

\begin{verbatim}
##   (Intercept)      year   mileage           cc  automatic
## 1   -5.477226 -361.4969 -433423.4 -10530.87904 -3.1037612
## 2    0.000000  158.6758  143331.0    -90.79521 -0.9327196
## 3    0.000000    0.0000  141468.0    -63.44464 -0.1440343
## 4    0.000000    0.0000       0.0   -501.66291 -0.2050022
## 5    0.000000    0.0000       0.0      0.00000  2.5365191
\end{verbatim}

\begin{Shaded}
\begin{Highlighting}[]
\NormalTok{Q1 }\SpecialCharTok{\%*\%}\NormalTok{ R}
\end{Highlighting}
\end{Shaded}

\begin{verbatim}
##       (Intercept) year mileage   cc    automatic
##  [1,]           1   78  133462 1998 1.000000e+00
##  [2,]           1   39   33000 2000 1.000000e+00
##  [3,]           1  109  120000 1800 3.330669e-16
##  [4,]           1   20   69727 1999 1.000000e+00
##  [5,]           1   70  112000 2000 4.440892e-16
##  [6,]           1   58   39106 1998 1.000000e+00
##  [7,]           1   53   95935 1800 1.000000e+00
##  [8,]           1   68  120000 1800 2.220446e-16
##  [9,]           1   15   20215 1798 1.000000e+00
## [10,]           1   96  140000 1800 3.330669e-16
## [11,]           1   63   68924 1998 1.000000e+00
## [12,]           1   82   90000 2000 4.440892e-16
## [13,]           1   76   81279 1998 2.220446e-16
## [14,]           1   17   24070 1798 1.000000e+00
## [15,]           1   38   40000 2000 2.220446e-16
## [16,]           1   46   56887 1832 1.000000e+00
## [17,]           1   95   91216 1997 1.000000e+00
## [18,]           1   37   48680 1998 1.000000e+00
## [19,]           1   68    8000 2000 2.220446e-16
## [20,]           1   41   60634 1835 1.000000e+00
## [21,]           1   69  114131 1998 1.000000e+00
## [22,]           1   71   75000 1800 4.440892e-16
## [23,]           1   99  124417 1998 1.000000e+00
## [24,]           1  129  130000 1800 3.608225e-16
## [25,]           1   57   77559 1997 1.000000e+00
## [26,]           1  107   75216 1838 1.000000e+00
## [27,]           1   45   52000 2000 2.220446e-16
## [28,]           1   80   58000 2000 1.000000e+00
## [29,]           1  113  134500 1800 3.608225e-16
## [30,]           1   41   80000 2000 1.110223e-16
\end{verbatim}

\begin{Shaded}
\begin{Highlighting}[]
\NormalTok{c }\OtherTok{\textless{}{-}} \FunctionTok{t}\NormalTok{(Q1) }\SpecialCharTok{\%*\%}\NormalTok{ y}
\NormalTok{hatb\_qr }\OtherTok{\textless{}{-}} \FunctionTok{solve}\NormalTok{(R, c)}
\NormalTok{hatb\_qr}
\end{Highlighting}
\end{Shaded}

\begin{verbatim}
##                      [,1]
## (Intercept) 525.286960604
## year         -5.799637101
## mileage      -0.002262844
## cc            0.388787346
## automatic   165.312632517
\end{verbatim}

\hypertarget{svd}{%
\section{SVD}\label{svd}}

\begin{Shaded}
\begin{Highlighting}[]
\NormalTok{SVDans }\OtherTok{\textless{}{-}} \FunctionTok{svd}\NormalTok{(X)}
\NormalTok{U1 }\OtherTok{\textless{}{-}}\NormalTok{ SVDans}\SpecialCharTok{$}\NormalTok{u}
\NormalTok{U1}
\end{Highlighting}
\end{Shaded}

\begin{verbatim}
##              [,1]        [,2]        [,3]        [,4]        [,5]
##  [1,] -0.27922535 -0.14384816 -0.17683730 -0.19605332  0.21490162
##  [2,] -0.06910432  0.29439322 -0.01255121 -0.08990468  0.07302296
##  [3,] -0.25106075 -0.12847284  0.18834231  0.10431768 -0.17349316
##  [4,] -0.14592052  0.13404317 -0.37005122 -0.06506012  0.09762905
##  [5,] -0.23433662 -0.04988011 -0.13888170  0.22047822  0.18122523
##  [6,] -0.08187526  0.26738497  0.12638464 -0.12857541  0.09130066
##  [7,] -0.20072757 -0.02370994 -0.18962224 -0.17516194 -0.23918651
##  [8,] -0.25106068 -0.12856953 -0.17837344  0.18285052 -0.20644692
##  [9,] -0.04235545  0.30581174 -0.14242453 -0.07935719 -0.35996963
## [10,] -0.29289168 -0.21567979 -0.03012250  0.12303554 -0.16115036
## [11,] -0.14424103  0.13742563  0.01875236 -0.14737052  0.12929933
## [12,] -0.18832260  0.04604211  0.08085774  0.20429408  0.16579954
## [13,] -0.17008212  0.08360304  0.07194430  0.21809284  0.14708697
## [14,] -0.05041837  0.28901322 -0.14423291 -0.08437980 -0.35396906
## [15,] -0.08374516  0.26387907 -0.05721816  0.30403022  0.07345566
## [16,] -0.11905814  0.15348681 -0.05580187 -0.14344493 -0.22608197
## [17,] -0.19086582  0.04011533  0.19116595 -0.21575072  0.17844689
## [18,] -0.10189970  0.22560416 -0.11036330 -0.09131098  0.08533222
## [19,] -0.01681569  0.40343187  0.37461298  0.25645964  0.06110169
## [20,] -0.12689529  0.13779975 -0.11995752 -0.13444132 -0.21990301
## [21,] -0.23879363 -0.05960914 -0.15856502 -0.17283842  0.18563870
## [22,] -0.15694105  0.06758419  0.07838486  0.19101551 -0.25531665
## [23,] -0.26030733 -0.10437316  0.05720774 -0.23348128  0.22147290
## [24,] -0.27197626 -0.17201383  0.31613351  0.06291757 -0.14602238
## [25,] -0.16230149  0.09955405 -0.07893713 -0.13874226  0.13234117
## [26,] -0.15739446  0.07505237  0.39557026 -0.26478314 -0.14431045
## [27,] -0.10884374  0.21158980 -0.05592161  0.28691249  0.09275684
## [28,] -0.12139308  0.18551955  0.22642819 -0.17616602  0.13446615
## [29,] -0.28138819 -0.19166622  0.15003237  0.09217341 -0.15375428
## [30,] -0.16740706  0.08953357 -0.23476347  0.28591832  0.12145000
\end{verbatim}

\begin{Shaded}
\begin{Highlighting}[]
\NormalTok{R1 }\OtherTok{\textless{}{-}} \FunctionTok{diag}\NormalTok{(SVDans}\SpecialCharTok{$}\NormalTok{d)}
\NormalTok{R1}
\end{Highlighting}
\end{Shaded}

\begin{verbatim}
##          [,1]     [,2]     [,3]     [,4]      [,5]
## [1,] 478020.2    0.000   0.0000 0.000000 0.0000000
## [2,]      0.0 4563.562   0.0000 0.000000 0.0000000
## [3,]      0.0    0.000 111.7954 0.000000 0.0000000
## [4,]      0.0    0.000   0.0000 2.536856 0.0000000
## [5,]      0.0    0.000   0.0000 0.000000 0.2528772
\end{verbatim}

\begin{Shaded}
\begin{Highlighting}[]
\NormalTok{V }\OtherTok{\textless{}{-}}\NormalTok{ SVDans}\SpecialCharTok{$}\NormalTok{v}
\NormalTok{V}
\end{Highlighting}
\end{Shaded}

\begin{verbatim}
##               [,1]          [,2]          [,3]          [,4]          [,5]
## [1,] -1.039213e-05  0.0005024682  0.0001895614 -1.705262e-03 -9.999984e-01
## [2,] -7.853907e-04  0.0107615439  0.9999299572 -4.859183e-03  2.032501e-04
## [3,] -9.998020e-01 -0.0198917212 -0.0005712134 -7.842434e-07  2.881731e-07
## [4,] -1.988441e-02  0.9997439979 -0.0107728592  4.945024e-04  4.996616e-04
## [5,] -5.214794e-06  0.0004412482 -0.0048645579 -9.999866e-01  1.704542e-03
\end{verbatim}

\begin{Shaded}
\begin{Highlighting}[]
\NormalTok{c }\OtherTok{\textless{}{-}} \FunctionTok{t}\NormalTok{(U1) }\SpecialCharTok{\%*\%}\NormalTok{ y}
\NormalTok{hatb\_svd }\OtherTok{\textless{}{-}}\NormalTok{ V }\SpecialCharTok{\%*\%} \FunctionTok{solve}\NormalTok{(R1, c)}
\NormalTok{hatb\_svd}
\end{Highlighting}
\end{Shaded}

\begin{verbatim}
##               [,1]
## [1,] 525.286960605
## [2,]  -5.799637101
## [3,]  -0.002262844
## [4,]   0.388787346
## [5,] 165.312632517
\end{verbatim}

\hypertarget{uxacb0uxacfc}{%
\section{결과}\label{uxacb0uxacfc}}

\begin{Shaded}
\begin{Highlighting}[]
\NormalTok{resbeta }\OtherTok{\textless{}{-}} \FunctionTok{data.frame}\NormalTok{(lmbeta, }\AttributeTok{chol =}\NormalTok{hatb\_chol, }\AttributeTok{qr=}\NormalTok{hatb\_qr, }\AttributeTok{svd =}\NormalTok{ hatb\_svd)}
\NormalTok{resbeta}
\end{Highlighting}
\end{Shaded}

\begin{verbatim}
##                    lmbeta          chol            qr           svd
## (Intercept) 525.286960604 525.286960604 525.286960604 525.286960605
## year         -5.799637101  -5.799637101  -5.799637101  -5.799637101
## mileage      -0.002262844  -0.002262844  -0.002262844  -0.002262844
## cc            0.388787346   0.388787346   0.388787346   0.388787346
## automatic   165.312632517 165.312632517 165.312632517 165.312632517
\end{verbatim}

\hypertarget{uxcc38uxace0}{%
\section{참고}\label{uxcc38uxace0}}

QR분해와 SVD 분해를 다음과 같이 하면 전체 차원에 대하여 구할 수 있다.

\begin{Shaded}
\begin{Highlighting}[]
\NormalTok{Q1 }\OtherTok{\textless{}{-}} \FunctionTok{qr.Q}\NormalTok{(QRans)}
\FunctionTok{dim}\NormalTok{(Q1)}
\end{Highlighting}
\end{Shaded}

\begin{verbatim}
## [1] 30  5
\end{verbatim}

\begin{Shaded}
\begin{Highlighting}[]
\NormalTok{Q }\OtherTok{\textless{}{-}} \FunctionTok{qr.Q}\NormalTok{(QRans, }\AttributeTok{complete =} \ConstantTok{TRUE}\NormalTok{)}
\FunctionTok{dim}\NormalTok{(Q)}
\end{Highlighting}
\end{Shaded}

\begin{verbatim}
## [1] 30 30
\end{verbatim}

\begin{Shaded}
\begin{Highlighting}[]
\NormalTok{RR }\OtherTok{\textless{}{-}} \FunctionTok{qr.R}\NormalTok{(QRans,}\AttributeTok{complete =} \ConstantTok{TRUE}\NormalTok{) }
\NormalTok{RR}
\end{Highlighting}
\end{Shaded}

\begin{verbatim}
##    (Intercept)      year   mileage           cc  automatic
## 1    -5.477226 -361.4969 -433423.4 -10530.87904 -3.1037612
## 2     0.000000  158.6758  143331.0    -90.79521 -0.9327196
## 3     0.000000    0.0000  141468.0    -63.44464 -0.1440343
## 4     0.000000    0.0000       0.0   -501.66291 -0.2050022
## 5     0.000000    0.0000       0.0      0.00000  2.5365191
## 6     0.000000    0.0000       0.0      0.00000  0.0000000
## 7     0.000000    0.0000       0.0      0.00000  0.0000000
## 8     0.000000    0.0000       0.0      0.00000  0.0000000
## 9     0.000000    0.0000       0.0      0.00000  0.0000000
## 10    0.000000    0.0000       0.0      0.00000  0.0000000
## 11    0.000000    0.0000       0.0      0.00000  0.0000000
## 12    0.000000    0.0000       0.0      0.00000  0.0000000
## 13    0.000000    0.0000       0.0      0.00000  0.0000000
## 14    0.000000    0.0000       0.0      0.00000  0.0000000
## 15    0.000000    0.0000       0.0      0.00000  0.0000000
## 16    0.000000    0.0000       0.0      0.00000  0.0000000
## 17    0.000000    0.0000       0.0      0.00000  0.0000000
## 18    0.000000    0.0000       0.0      0.00000  0.0000000
## 19    0.000000    0.0000       0.0      0.00000  0.0000000
## 20    0.000000    0.0000       0.0      0.00000  0.0000000
## 21    0.000000    0.0000       0.0      0.00000  0.0000000
## 22    0.000000    0.0000       0.0      0.00000  0.0000000
## 23    0.000000    0.0000       0.0      0.00000  0.0000000
## 24    0.000000    0.0000       0.0      0.00000  0.0000000
## 25    0.000000    0.0000       0.0      0.00000  0.0000000
## 26    0.000000    0.0000       0.0      0.00000  0.0000000
## 27    0.000000    0.0000       0.0      0.00000  0.0000000
## 28    0.000000    0.0000       0.0      0.00000  0.0000000
## 29    0.000000    0.0000       0.0      0.00000  0.0000000
## 30    0.000000    0.0000       0.0      0.00000  0.0000000
## attr(,"assign")
## [1] 0 1 2 3 4
\end{verbatim}

\begin{Shaded}
\begin{Highlighting}[]
\FunctionTok{svd}\NormalTok{(X, }\AttributeTok{nu=}\FunctionTok{dim}\NormalTok{(X)[}\DecValTok{1}\NormalTok{], }\AttributeTok{nv=}\FunctionTok{dim}\NormalTok{(X)[}\DecValTok{2}\NormalTok{])}
\end{Highlighting}
\end{Shaded}

\begin{verbatim}
## $d
## [1] 4.780202e+05 4.563562e+03 1.117954e+02 2.536856e+00 2.528772e-01
## 
## $u
##              [,1]        [,2]        [,3]        [,4]        [,5]         [,6]         [,7]         [,8]         [,9]        [,10]         [,11]        [,12]         [,13]        [,14]        [,15]         [,16]        [,17]        [,18]       [,19]        [,20]        [,21]        [,22]
##  [1,] -0.27922535 -0.14384816 -0.17683730 -0.19605332  0.21490162 -0.075429520 -0.299217629 -0.217825866 -0.091015083 -0.248937500 -0.1707889060 -0.073609738 -0.0519965469 -0.101539469  0.038822890 -0.1712774263 -0.204525951 -0.136176758  0.18829708 -0.190418431 -0.317469568 -0.059858007
##  [2,] -0.06910432  0.29439322 -0.01255121 -0.08990468  0.07302296 -0.168760218 -0.102057561  0.022372896 -0.425673128  0.173848733 -0.0782182361  0.054238210  0.0105208534 -0.408983685 -0.228721140 -0.2138290393  0.093265422 -0.222920465 -0.19567393 -0.222381022  0.053233000 -0.075440197
##  [3,] -0.25106075 -0.12847284  0.18834231  0.10431768 -0.17349316 -0.220397812  0.167948520  0.134456509  0.009452520  0.043980201 -0.0844529246 -0.164972950 -0.1710641960  0.017051311 -0.130419084 -0.0103257941 -0.191017539 -0.010131919 -0.53698163  0.047905865  0.133302028 -0.148434710
##  [4,] -0.14592052  0.13404317 -0.37005122 -0.06506012  0.09762905 -0.140220867  0.189560341  0.247685524  0.246304277  0.235419887 -0.1620418086 -0.105061658 -0.0953506991  0.243613356 -0.071896921  0.1578472745 -0.173754333 -0.149401319 -0.04507774  0.148915906 -0.195276497  0.281630998
##  [5,] -0.23433662 -0.04988011 -0.13888170  0.22047822  0.18122523  0.136780832  0.074061790 -0.269233703  0.075180510 -0.197636410  0.0968507485 -0.225195781 -0.2309089498  0.075421206 -0.300405753  0.1130107856  0.180171792  0.033726849 -0.11638496  0.084445259  0.030004064 -0.166877796
##  [6,] -0.08187526  0.26738497  0.12638464 -0.12857541  0.09130066  0.888944052  0.026238750  0.077218036 -0.011931879  0.052673854 -0.0765800355 -0.038201708 -0.0390695435 -0.010092638 -0.027127813 -0.0250624873 -0.105247607 -0.056265299 -0.13252768 -0.010503352 -0.021254365  0.004791949
##  [7,] -0.20072757 -0.02370994 -0.18962224 -0.17516194 -0.23918651  0.044176807  0.820988040 -0.125492873 -0.142244590 -0.099634572  0.0092709871  0.076576589  0.0764253319 -0.144051404  0.063907884 -0.1135803726  0.039029819 -0.012221652  0.17206857 -0.127284779 -0.046809226 -0.051732317
##  [8,] -0.25106068 -0.12856953 -0.17837344  0.18285052 -0.20644692  0.103392532 -0.147156067  0.796827470 -0.089286352 -0.166159639  0.0510036358  0.014508584  0.0154544978 -0.092096820 -0.001908755 -0.0623928016  0.094028135  0.020394212  0.15784730 -0.083814061 -0.033049919 -0.093236819
##  [9,] -0.04235545  0.30581174 -0.14242453 -0.07935719 -0.35996963 -0.029475729 -0.150064556 -0.102563812  0.743782144 -0.049229780 -0.0025298335  0.050236346  0.0343367254 -0.250797667 -0.045901626 -0.1615689625  0.057284793 -0.053164511 -0.02422286 -0.163944801  0.036061637 -0.128141958
## [10,] -0.29289168 -0.21567979 -0.03012250  0.12303554 -0.16115036  0.096801407 -0.118661529 -0.170050357 -0.016652582  0.832752020  0.0396812681  0.002359084  0.0106636650 -0.021779286  0.038726243 -0.0313036051  0.043708705  0.040750950  0.15902352 -0.046622674 -0.049555006 -0.066254755
## [11,] -0.14424103  0.13742563  0.01875236 -0.14737052  0.12929933 -0.067328056 -0.005833423  0.045850556  0.008622881  0.025549419  0.9363463427 -0.016910609 -0.0133790314  0.007847630  0.009826160 -0.0168651952 -0.086804703 -0.045267771 -0.02678442 -0.011826906 -0.056623021  0.030564233
## [12,] -0.18832260  0.04604211  0.08085774  0.20429408  0.16579954 -0.019568300  0.039528261 -0.017671950  0.071629597 -0.034218298 -0.0273673333  0.913468018 -0.0814880189  0.069999869 -0.057102849  0.0390739408 -0.045833289 -0.011694160 -0.06496977  0.039658339 -0.038465820 -0.010743618
## [13,] -0.17008212  0.08360304  0.07194430  0.21809284  0.14708697 -0.025231884  0.038543823 -0.019590633  0.053421980 -0.030348731 -0.0269678577 -0.087300908  0.9155936077  0.052670658 -0.070526309  0.0310653527 -0.038956567 -0.016989992 -0.08242967  0.032107321 -0.028984550 -0.021075780
## [14,] -0.05041837  0.28901322 -0.14423291 -0.08437980 -0.35396906 -0.025795182 -0.151464623 -0.103644582 -0.250328567 -0.051797137 -0.0019566484  0.051537402  0.0364486347  0.754722711 -0.040253647 -0.1591037220  0.056226244 -0.051029824 -0.01438257 -0.162024606  0.031885434 -0.124277518
## [15,] -0.08374516  0.26387907 -0.05721816  0.30403022  0.07345566 -0.040984407  0.021044974 -0.043341144 -0.047875750 -0.013452981 -0.0201520408 -0.084040092 -0.0927428770 -0.044308390  0.858028775 -0.0116525076  0.013832358 -0.048892772 -0.14149083 -0.013149521  0.010411274 -0.067970580
## [16,] -0.11905814  0.15348681 -0.05580187 -0.14344493 -0.22608197 -0.020624418 -0.119401009 -0.067470273 -0.148456596 -0.049796792 -0.0152783729  0.039706106  0.0345320686 -0.146991738  0.010383855  0.8896079577  0.004364348 -0.031829025  0.02486154 -0.111412197 -0.008076644 -0.069510381
## [17,] -0.19086582  0.04011533  0.19116595 -0.21575072  0.17844689 -0.075571514  0.026952884  0.083946624  0.090788806  0.024428242 -0.0766657296 -0.031004850 -0.0191554057  0.087418893  0.055777964  0.0181585781  0.855461286 -0.022006222 -0.02739966  0.030385346 -0.074555011  0.060445011
## [18,] -0.10189970  0.22560416 -0.11036330 -0.09131098  0.08533222 -0.064069304 -0.029924572  0.017470070 -0.059306083  0.028276149 -0.0540926109 -0.007123968 -0.0106220547 -0.057790292 -0.029798835 -0.0459023297 -0.041827409  0.935360372 -0.03460176 -0.045854202 -0.039230810  0.004174156
## [19,] -0.01681569  0.40343187  0.37461298  0.25645964  0.06110169 -0.139456496  0.127220201  0.068039944  0.014506948  0.040200908 -0.0627123636 -0.144814857 -0.1519140488  0.020072825 -0.158887600  0.0162717550 -0.095915904 -0.042069170  0.64494175  0.042912010  0.058725389 -0.081716148
## [20,] -0.12689529  0.13779975 -0.11995752 -0.13444132 -0.21990301 -0.007694886 -0.133113852 -0.082098650 -0.157269746 -0.055739804 -0.0098373797  0.047529106  0.0419781377 -0.156020184  0.010954551 -0.1138566761  0.019904449 -0.033876969  0.05313915  0.881116681 -0.014868069 -0.066971864
## [21,] -0.23879363 -0.05960914 -0.15856502 -0.17283842  0.18563870  0.001271196 -0.057664721 -0.005178806  0.036012242 -0.016670762 -0.0427739483  0.017063660  0.0269825173  0.031309241  0.064835669 -0.0060783660 -0.054823370 -0.029426828  0.13800551 -0.016275436  0.889291778  0.069015782
## [22,] -0.15694105  0.06758419  0.07838486  0.19101551 -0.25531665  0.022030308 -0.077357170 -0.132659275 -0.095126984 -0.117905688  0.0229412181 -0.028932069 -0.0327395146 -0.094232504 -0.050896659 -0.0638138081  0.039370956  0.009351300 -0.03158050 -0.065523138  0.023415328  0.872017421
## [23,] -0.26030733 -0.10437316  0.05720774 -0.23348128  0.22147290 -0.024854387 -0.011358958  0.046136792  0.110614281 -0.006404093 -0.0612510937 -0.005913515  0.0105797854  0.104363947  0.095765463  0.0261083181 -0.120381393 -0.010741627  0.09446326  0.026953841 -0.114628279  0.089029418
## [24,] -0.27197626 -0.17201383  0.31613351  0.06291757 -0.14602238  0.031596425 -0.037547503 -0.083326704  0.059250404 -0.134084965  0.0072170880 -0.041967989 -0.0289371178  0.054208947  0.048631767  0.0019532394 -0.050679568  0.055325736  0.02542448  0.007397732 -0.027538013 -0.061256356
## [25,] -0.16230149  0.09955405 -0.07893713 -0.13874226  0.13234117 -0.043553659 -0.030811852  0.019839009 -0.003888602  0.012068218 -0.0536190487 -0.002461830  0.0009861557 -0.005245894  0.015889574 -0.0226745308 -0.062154901 -0.045850322  0.02522850 -0.024181144 -0.068813780  0.034592987
## [26,] -0.15739446  0.07505237  0.39557026 -0.26478314 -0.14431045 -0.079707454 -0.018110705  0.043908193  0.006571716 -0.023524708 -0.0558191557 -0.011101889 -0.0030582439  0.005076707  0.069440994 -0.0425078154 -0.133826293  0.004507466 -0.07594886 -0.019945805 -0.014250424 -0.027852336
## [27,] -0.10884374  0.21158980 -0.05592161  0.28691249  0.09275684 -0.030655990  0.018257852 -0.045003807 -0.027694525 -0.020907361 -0.0189959175 -0.080820476 -0.0868607223 -0.025293202 -0.123892048 -0.0031746097  0.008567727 -0.041840025 -0.11305091 -0.005975918 -0.002361812 -0.055636585
## [28,] -0.12139308  0.18551955  0.22642819 -0.17616602  0.13446615 -0.111306860  0.045283446  0.099847161  0.047417929  0.049470647 -0.0841645220 -0.045497653 -0.0401586851  0.047160485  0.008094979  0.0005847946 -0.142190402 -0.039689245 -0.11944411  0.018680907 -0.038616156  0.029540043
## [29,] -0.28138819 -0.19166622  0.15003237  0.09217341 -0.15375428  0.062614745 -0.076382123 -0.124879527  0.022348255 -0.149806442  0.0227588237 -0.020787570 -0.0100891989  0.017294688  0.043434782 -0.0142064185 -0.005279735  0.048161420  0.08879857 -0.018688228 -0.037782450 -0.063958036
## [30,] -0.16740706  0.08953357 -0.23476347  0.28591832  0.12145000  0.023078639 -0.029501029 -0.093566823 -0.029154022 -0.052412613  0.0001956554 -0.051502994 -0.0549671264 -0.029009302 -0.094782511 -0.0045055288  0.048002930 -0.036092427  0.01085265 -0.020652987 -0.038120513 -0.034856238
##              [,23]        [,24]        [,25]        [,26]        [,27]        [,28]        [,29]        [,30]
##  [1,] -0.312678690 -0.170738743 -0.208445441 -0.152001015  0.007094190 -0.110524798 -0.207460937 -0.094040290
##  [2,]  0.188573696  0.271161638 -0.079685473  0.056694999 -0.173918575 -0.041522577  0.223136609 -0.120911884
##  [3,] -0.028209133 -0.257202205  0.009923558 -0.354388789 -0.112472497 -0.273283470 -0.113193875  0.079272760
##  [4,] -0.200084366  0.246073102 -0.167194615  0.138709378 -0.080198854 -0.156001679  0.241168550 -0.101525663
##  [5,]  0.129094912 -0.045826854  0.055918168  0.317526152 -0.296565683  0.185928511 -0.118731568 -0.368769353
##  [6,] -0.064048599 -0.025956681 -0.051971447 -0.117223676 -0.022900157 -0.126072269  0.011651273  0.026291040
##  [7,] -0.002324873 -0.018605216 -0.016371254 -0.015367741  0.059831516  0.060994937 -0.057367411  0.009671275
##  [8,]  0.031653573 -0.047073787  0.013482178  0.078261424 -0.008390478  0.126419177 -0.104029041 -0.083040245
##  [9,]  0.085809021 -0.008655873 -0.006145459 -0.057551866 -0.027921655  0.016067057 -0.028559388 -0.015058434
## [10,] -0.022090550 -0.090243946  0.010299223  0.032713508  0.024059346  0.092854611 -0.126814187 -0.044083402
## [11,] -0.081473171 -0.007713686 -0.055502522 -0.069357584  0.006697690 -0.082809478  0.008316961  0.018179729
## [12,] -0.054137466 -0.047523075 -0.025866615  0.005646443 -0.062531426 -0.033835578 -0.041007954 -0.065862964
## [13,] -0.040589671 -0.043264694 -0.024261177  0.005588562 -0.073166463 -0.034282740 -0.037004866 -0.071411959
## [14,]  0.081246531 -0.009340264 -0.006646706 -0.055628263 -0.023403262  0.018223223 -0.030118999 -0.013668790
## [15,]  0.035856064  0.001952783 -0.018768501  0.037597766 -0.130366592 -0.015943924 -0.005722413 -0.116419191
## [16,]  0.009929471 -0.031768599 -0.018495804 -0.070945060  0.015386319 -0.005456017 -0.040538619  0.015436318
## [17,] -0.142372245 -0.057913243 -0.058865030 -0.122795648  0.043441119 -0.121887549 -0.018116686  0.063152602
## [18,] -0.031069111  0.030719093 -0.052358122 -0.030631686 -0.024951159 -0.054781266  0.029427792 -0.017034276
## [19,] -0.005793140 -0.097129792 -0.016147841 -0.125198700 -0.144053743 -0.152174388 -0.031630174 -0.044006917
## [20,]  0.016281370 -0.015913061 -0.019387604 -0.046664371  0.015717509  0.013245940 -0.035132055  0.003899332
## [21,] -0.105193556  0.023497508 -0.060803742  0.008191747  0.050763081 -0.013205651  0.004579539  0.003035130
## [22,]  0.039954495 -0.098312838  0.018462592 -0.027202645 -0.047667964  0.034350829 -0.107789516 -0.051802062
## [23,]  0.840546459 -0.033773729 -0.062956288 -0.064383367  0.075425459 -0.070078747 -0.020144923  0.051171441
## [24,] -0.075419967  0.825682351  0.010337396 -0.092595920  0.031857572 -0.009706517 -0.154607546  0.021103036
## [25,] -0.073455625  0.014009307  0.943932424 -0.032207212  0.011632508 -0.051026975  0.013183390  0.002540507
## [26,] -0.101753433 -0.148956351 -0.023848931  0.777033426  0.062165324 -0.128002983 -0.088618063  0.114040063
## [27,]  0.020332452 -0.001918332 -0.020354394  0.041145977  0.884094223 -0.011198292 -0.011222817 -0.110698609
## [28,] -0.107932877 -0.056125202 -0.055253989 -0.145872062  0.005138948  0.852838832 -0.005407034  0.054821899
## [29,] -0.049459353 -0.133899215  0.010358224 -0.032583360  0.027760570  0.039366095  0.858858270 -0.010302448
## [30,]  0.018807785  0.034759603 -0.023382813  0.113489580 -0.092556865  0.051505685 -0.007104314  0.866021355
## 
## $v
##               [,1]          [,2]          [,3]          [,4]          [,5]
## [1,] -1.039213e-05  0.0005024682  0.0001895614 -1.705262e-03 -9.999984e-01
## [2,] -7.853907e-04  0.0107615439  0.9999299572 -4.859183e-03  2.032501e-04
## [3,] -9.998020e-01 -0.0198917212 -0.0005712134 -7.842434e-07  2.881731e-07
## [4,] -1.988441e-02  0.9997439979 -0.0107728592  4.945024e-04  4.996616e-04
## [5,] -5.214794e-06  0.0004412482 -0.0048645579 -9.999866e-01  1.704542e-03
\end{verbatim}

\hypertarget{chapter05}{%
\chapter{자료에 대한 진단}\label{chapter05}}

\hypertarget{uxc794uxcc28-uxadf8uxb9bc}{%
\section{잔차 그림}\label{uxc794uxcc28-uxadf8uxb9bc}}

회귀 분석에서 이상점과 영향점에 대한 분석을 할 때 여러 가지 잔차 그림(residual plot)은 매우 유용하다.

\begin{itemize}
\item
  Studentized 잔차 \(r_i^*\) 와 예측값 \(\hat y_i\) 그림 (residual vs.~fitted value)
\item
  Studentized 잔차 \(r_i^*\) 와 독립변수 값 \(x_{ji}\) 그림 (residual vs.~predictor)
\end{itemize}

\hypertarget{uxbd80uxbd84uxd68cuxadc0uxadf8uxb9bc-partial-regression-plot-added-variable-plot}{%
\subsection{부분회귀그림 (Partial regression plot; added variable plot)}\label{uxbd80uxbd84uxd68cuxadc0uxadf8uxb9bc-partial-regression-plot-added-variable-plot}}

하나의 독립변수 \(x_j\)와 종속변수 \(y\)의 관계를 다른 독립변수의 영향을 제거하고 검토하는 그림이다.

다음과 같이 독립변수 \(x_j\)를 제외한 \(y\)에 대한 회귀식을 고려하고

\[ 
y_i=\beta_0 + \beta_1 x_{i1} + \beta_2 x_{i2} + \dots + \beta_{j-1} x_{i,j-1
} +\beta_{j+1} x_{i,j+1}+  \beta_{p-1} x_{i, p-1} + e_i 
\]

그 잔차를 \(r_{y| {\bm X}_{-j}}\) 라고 하자. 또한 설명변수 \(x_j\)를 반응변수로 하는 대한 회귀식을 고려하고

\[
x_{ij}=\beta'_0 + \beta'_1 x_{i1} + \beta'_2 x_{i2} + \dots + \beta'_{j-1} x_{i,j-1} +\beta'_{j+1} x_{i,j+1}+  \beta'_{p-1} x_{i,p-1} + e'_i 
\]

그 잔차를 \(r_{x_j|\bm X_{-j}}\)라고 하자. 위에서 구한 두 잔차 \(r_{y|\bm X_{-j}}\)과 \(r_{x_j|\bm X_{-j}}\)에 대한 그림을 부분회귀그림이라고 하며 만약에 설명변수 \(x_j\)가 유의하면 두 잔차의 관계는 다음과 같다.

\[
r_{y|\bm X_{-j}} = \beta_j r_{x_j|\bm X_{-j}} + e 
\]

\hypertarget{uxc790uxb8cc-usedcars-uxc5d0-uxb300uxd55c-uxc794uxcc28-uxbd84uxc11d-uxc608uxc81c-3.8-uxc608uxc81c-5.3}{%
\section{\texorpdfstring{자료 \texttt{usedcars} 에 대한 잔차 분석 (예제 3.8, 예제 5.3)}{자료 usedcars 에 대한 잔차 분석 (예제 3.8, 예제 5.3)}}\label{uxc790uxb8cc-usedcars-uxc5d0-uxb300uxd55c-uxc794uxcc28-uxbd84uxc11d-uxc608uxc81c-3.8-uxc608uxc81c-5.3}}

\begin{Shaded}
\begin{Highlighting}[]
\NormalTok{usedcars.lm }\OtherTok{\textless{}{-}} \FunctionTok{lm}\NormalTok{(price }\SpecialCharTok{\textasciitilde{}}\NormalTok{ year }\SpecialCharTok{+}\NormalTok{ mileage }\SpecialCharTok{+}\NormalTok{ cc }\SpecialCharTok{+}\NormalTok{ automatic, usedcars)}
\FunctionTok{summary}\NormalTok{(usedcars.lm)}
\end{Highlighting}
\end{Shaded}

\begin{verbatim}
## 
## Call:
## lm(formula = price ~ year + mileage + cc + automatic, data = usedcars)
## 
## Residuals:
##     Min      1Q  Median      3Q     Max 
## -177.35  -63.91   -0.99   70.34  212.69 
## 
## Coefficients:
##               Estimate Std. Error t value Pr(>|t|)    
## (Intercept)  5.253e+02  3.998e+02   1.314 0.200823    
## year        -5.800e+00  9.283e-01  -6.247 1.55e-06 ***
## mileage     -2.263e-03  7.211e-04  -3.138 0.004324 ** 
## cc           3.888e-01  2.022e-01   1.923 0.065958 .  
## automatic    1.653e+02  3.986e+01   4.147 0.000339 ***
## ---
## Signif. codes:  0 '***' 0.001 '**' 0.01 '*' 0.05 '.' 0.1 ' ' 1
## 
## Residual standard error: 101.1 on 25 degrees of freedom
## Multiple R-squared:  0.9045, Adjusted R-squared:  0.8892 
## F-statistic: 59.21 on 4 and 25 DF,  p-value: 2.184e-12
\end{verbatim}

\hypertarget{uxc794uxcc28uxadf8uxb9bc}{%
\subsection{잔차그림}\label{uxc794uxcc28uxadf8uxb9bc}}

\begin{Shaded}
\begin{Highlighting}[]
\FunctionTok{plot}\NormalTok{(usedcars.lm)}
\end{Highlighting}
\end{Shaded}

\includegraphics{lmpractice_files/figure-latex/unnamed-chunk-54-1.pdf} \includegraphics{lmpractice_files/figure-latex/unnamed-chunk-54-2.pdf} \includegraphics{lmpractice_files/figure-latex/unnamed-chunk-54-3.pdf} \includegraphics{lmpractice_files/figure-latex/unnamed-chunk-54-4.pdf}

\hypertarget{uxc794uxcc28}{%
\subsection{잔차}\label{uxc794uxcc28}}

\begin{Shaded}
\begin{Highlighting}[]
\NormalTok{resid\_inter }\OtherTok{\textless{}{-}} \FunctionTok{rstandard}\NormalTok{(usedcars.lm)  }\CommentTok{\# internal studentized residual}
\NormalTok{resid\_exter }\OtherTok{\textless{}{-}} \FunctionTok{rstudent}\NormalTok{(usedcars.lm)   }\CommentTok{\# external studentized residual}
\NormalTok{hatval }\OtherTok{\textless{}{-}} \FunctionTok{hatvalues}\NormalTok{(usedcars.lm)       }\CommentTok{\# leverage}
\FunctionTok{data.frame}\NormalTok{(resid\_inter , resid\_exter, hatval)}
\end{Highlighting}
\end{Shaded}

\begin{verbatim}
##     resid_inter  resid_exter     hatval
## 1   0.859118727  0.854468915 0.21455013
## 2   2.223678962  2.432560486 0.10501551
## 3  -0.553417436 -0.545588388 0.15599165
## 4  -0.044084943 -0.043195925 0.18996253
## 5  -0.576180936 -0.568325835 0.15814304
## 6   0.816421059  0.810807796 0.11903880
## 7  -0.551399974 -0.543574935 0.16470220
## 8  -1.198080391 -1.209098031 0.18743332
## 9   0.378399054  0.371820157 0.25147525
## 10  0.745189938  0.738380670 0.17431785
## 11 -0.010873657 -0.010653990 0.07847932
## 12  1.537452127  1.583088042 0.11334881
## 13 -0.706665346 -0.699408397 0.11029244
## 14  1.286252868  1.304157036 0.23928783
## 15  0.305838319  0.300221293 0.17774944
## 16 -1.862061057 -1.965848316 0.11253640
## 17 -0.425962246 -0.418878889 0.15297508
## 18 -1.582023043 -1.634007936 0.08908012
## 19 -1.128902196 -1.135412108 0.37287989
## 20  0.746526192  0.739734875 0.11591279
## 21 -0.739861832 -0.732982630 0.15005336
## 22 -0.783820912 -0.777598695 0.13701582
## 23  0.529387292  0.521623446 0.18549015
## 24  1.320232140  1.341155689 0.22878139
## 25 -0.009531759 -0.009339195 0.07924745
## 26  0.414031097  0.407063963 0.27781733
## 27  0.813419294  0.807745473 0.15066704
## 28 -1.855054061 -1.957267375 0.14953912
## 29  0.158642701  0.155515766 0.17056129
## 30 -0.055408868 -0.054292716 0.18765466
\end{verbatim}

\hypertarget{uxc601uxd5a5uxc810-uxce21uxb3c4}{%
\subsection{영향점 측도}\label{uxc601uxd5a5uxc810-uxce21uxb3c4}}

\begin{Shaded}
\begin{Highlighting}[]
\CommentTok{\# DFBETAS for each model variable, DFFITS, covariance ratios, }
\CommentTok{\# Cook\textquotesingle{}s distances and the diagonal elements of the hat matrix}
\CommentTok{\# Cases which are influential with respect to any of these measures }
\CommentTok{\# are marked with an asterisk.}
\FunctionTok{influence.measures}\NormalTok{(usedcars.lm)}
\end{Highlighting}
\end{Shaded}

\begin{verbatim}
## Influence measures of
##   lm(formula = price ~ year + mileage + cc + automatic, data = usedcars) :
## 
##      dfb.1_  dfb.year  dfb.milg   dfb.cc dfb.atmt    dffit cov.r   cook.d    hat inf
## 1  -0.20716 -0.108544  0.327125  0.17932  0.19256  0.44658 1.344 4.03e-02 0.2146    
## 2  -0.18773  0.033416 -0.347354  0.24746  0.23435  0.83326 0.455 1.16e-01 0.1050    
## 3  -0.10302 -0.086990  0.008925  0.10950  0.06372 -0.23455 1.366 1.13e-02 0.1560    
## 4   0.00469  0.016237 -0.011729 -0.00589 -0.00320 -0.02092 1.513 9.12e-05 0.1900    
## 5   0.11228  0.102465 -0.135083 -0.12498  0.13462 -0.24632 1.363 1.25e-02 0.1581    
## 6  -0.07885  0.136480 -0.181209  0.08714  0.11239  0.29805 1.216 1.80e-02 0.1190    
## 7  -0.14228  0.100579 -0.106201  0.14682 -0.10174 -0.24137 1.381 1.20e-02 0.1647    
## 8  -0.27687  0.308498 -0.320654  0.25704  0.24993 -0.58070 1.123 6.62e-02 0.1874    
## 9   0.15471 -0.066016 -0.054370 -0.13883  0.03146  0.21551 1.592 9.62e-03 0.2515    
## 10  0.13093 -0.056206  0.169173 -0.13765 -0.10220  0.33927 1.328 2.34e-02 0.1743    
## 11  0.00143 -0.000670  0.000312 -0.00142 -0.00166 -0.00311 1.331 2.01e-06 0.0785    
## 12 -0.27881  0.085189 -0.022194  0.31082 -0.33867  0.56603 0.842 6.04e-02 0.1133    
## 13  0.10909 -0.027792  0.028700 -0.12774  0.15983 -0.24625 1.246 1.24e-02 0.1103    
## 14  0.52930 -0.229975 -0.166587 -0.47750  0.11713  0.73144 1.145 1.04e-01 0.2393    
## 15 -0.02434 -0.037299 -0.032166  0.04432 -0.10023  0.13959 1.464 4.04e-03 0.1777    
## 16 -0.47183  0.092199  0.101866  0.45460 -0.29124 -0.70004 0.655 8.79e-02 0.1125    
## 17  0.08120 -0.112348  0.030779 -0.06848 -0.09956 -0.17801 1.396 6.55e-03 0.1530    
## 18  0.14607  0.138551  0.019412 -0.18052 -0.15882 -0.51098 0.795 4.90e-02 0.0891    
## 19  0.08767 -0.454297  0.733107 -0.15988  0.36621 -0.87551 1.505 1.52e-01 0.3729    
## 20  0.17302 -0.084980  0.007572 -0.16482  0.10281  0.26785 1.239 1.46e-02 0.1159    
## 21  0.14757  0.081228 -0.204480 -0.13332 -0.13993 -0.30798 1.292 1.93e-02 0.1501    
## 22 -0.21369 -0.011866  0.084126  0.19253  0.16353 -0.30984 1.255 1.95e-02 0.1370    
## 23 -0.12798  0.071546  0.083597  0.10512  0.13711  0.24893 1.423 1.28e-02 0.1855    
## 24  0.22299  0.430646 -0.103308 -0.26300 -0.09994  0.73047 1.108 1.03e-01 0.2288    
## 25  0.00129  0.000354 -0.000686 -0.00128 -0.00137 -0.00274 1.332 1.56e-06 0.0792    
## 26  0.06915  0.204983 -0.141142 -0.08585  0.12561  0.25248 1.641 1.32e-02 0.2778   *
## 27 -0.08134 -0.093035 -0.048156  0.12751 -0.25004  0.34021 1.263 2.35e-02 0.1507    
## 28  0.28532 -0.571184  0.447521 -0.26551 -0.37866 -0.82073 0.688 1.21e-01 0.1495    
## 29  0.02625  0.019365  0.010861 -0.02922 -0.01619  0.07052 1.471 1.04e-03 0.1706    
## 30  0.00732  0.016726 -0.010214 -0.01018  0.01709 -0.02609 1.509 1.42e-04 0.1877
\end{verbatim}

\hypertarget{uxbd80uxbd84uxd68cuxadc0uxadf8uxb9bc}{%
\subsection{부분회귀그림}\label{uxbd80uxbd84uxd68cuxadc0uxadf8uxb9bc}}

\begin{Shaded}
\begin{Highlighting}[]
\CommentTok{\#added variable plot}
\FunctionTok{avPlots}\NormalTok{(usedcars.lm)}
\end{Highlighting}
\end{Shaded}

\includegraphics{lmpractice_files/figure-latex/unnamed-chunk-57-1.pdf}

\hypertarget{uxc790uxb8cc-houseprice-uxc5d0-uxb300uxd55c-uxc794uxcc28-uxbd84uxc11d-uxc5f0uxc2b5uxbb38uxc81c-5.9}{%
\section{\texorpdfstring{자료 \texttt{houseprice} 에 대한 잔차 분석 (연습문제 5.9)}{자료 houseprice 에 대한 잔차 분석 (연습문제 5.9)}}\label{uxc790uxb8cc-houseprice-uxc5d0-uxb300uxd55c-uxc794uxcc28-uxbd84uxc11d-uxc5f0uxc2b5uxbb38uxc81c-5.9}}

\begin{itemize}
\tightlist
\item
  \texttt{price} : 주택 판매가격(천만원)
\item
  \texttt{tax} : 세금(만원)
\item
  \texttt{ground} : 대지평수(평)
\item
  \texttt{floor} : 건물평수(평)
\item
  \texttt{year} : 주택연령(년)
\end{itemize}

\begin{Shaded}
\begin{Highlighting}[]
\NormalTok{house.lm }\OtherTok{\textless{}{-}} \FunctionTok{lm}\NormalTok{(price }\SpecialCharTok{\textasciitilde{}}\NormalTok{ tax }\SpecialCharTok{+}\NormalTok{ ground }\SpecialCharTok{+}\NormalTok{ floor }\SpecialCharTok{+}\NormalTok{ year, houseprice)}
\FunctionTok{summary}\NormalTok{(house.lm )}
\end{Highlighting}
\end{Shaded}

\begin{verbatim}
## 
## Call:
## lm(formula = price ~ tax + ground + floor + year, data = houseprice)
## 
## Residuals:
##     Min      1Q  Median      3Q     Max 
## -3.4891 -1.3574  0.1337  1.0686  3.4938 
## 
## Coefficients:
##             Estimate Std. Error t value Pr(>|t|)    
## (Intercept)  1.21874    2.04661   0.595  0.55759    
## tax          0.05195    0.01383   3.756  0.00109 ** 
## ground       0.01159    0.02534   0.458  0.65169    
## floor        0.34941    0.07268   4.807 8.41e-05 ***
## year        -0.21894    0.33149  -0.660  0.51582    
## ---
## Signif. codes:  0 '***' 0.001 '**' 0.01 '*' 0.05 '.' 0.1 ' ' 1
## 
## Residual standard error: 2.039 on 22 degrees of freedom
## Multiple R-squared:  0.9313, Adjusted R-squared:  0.9188 
## F-statistic: 74.53 on 4 and 22 DF,  p-value: 1.817e-12
\end{verbatim}

\hypertarget{uxc794uxcc28uxadf8uxb9bc-1}{%
\subsection{잔차그림}\label{uxc794uxcc28uxadf8uxb9bc-1}}

\begin{Shaded}
\begin{Highlighting}[]
\FunctionTok{plot}\NormalTok{(house.lm)}
\end{Highlighting}
\end{Shaded}

\includegraphics{lmpractice_files/figure-latex/unnamed-chunk-59-1.pdf} \includegraphics{lmpractice_files/figure-latex/unnamed-chunk-59-2.pdf} \includegraphics{lmpractice_files/figure-latex/unnamed-chunk-59-3.pdf} \includegraphics{lmpractice_files/figure-latex/unnamed-chunk-59-4.pdf}

\hypertarget{uxc794uxcc28-1}{%
\subsection{잔차}\label{uxc794uxcc28-1}}

\begin{Shaded}
\begin{Highlighting}[]
\NormalTok{resid\_inter }\OtherTok{\textless{}{-}} \FunctionTok{rstandard}\NormalTok{(house.lm)  }\CommentTok{\# internal studentized residual}
\NormalTok{resid\_exter }\OtherTok{\textless{}{-}} \FunctionTok{rstudent}\NormalTok{(house.lm)   }\CommentTok{\# external studentized residual}
\NormalTok{hatval }\OtherTok{\textless{}{-}} \FunctionTok{hatvalues}\NormalTok{(house.lm)       }\CommentTok{\# leverage}
\FunctionTok{data.frame}\NormalTok{(resid\_inter , resid\_exter, hatval)}
\end{Highlighting}
\end{Shaded}

\begin{verbatim}
##    resid_inter resid_exter     hatval
## 1   0.08485161  0.08291431 0.09320631
## 2  -0.67688158 -0.66831473 0.21913894
## 3   0.22562538  0.22069338 0.19724713
## 4  -0.46948020 -0.46100124 0.11507835
## 5   0.52919554  0.52035100 0.06445355
## 6   1.84331217  1.95851264 0.13574846
## 7  -0.06384170 -0.06237966 0.09690865
## 8  -1.99395355 -2.15227270 0.26336419
## 9   0.83013951  0.82406251 0.58244898
## 10  1.91549041  2.05020730 0.38378669
## 11  1.05080043  1.05341670 0.06658106
## 12 -0.94527321 -0.94288627 0.18220181
## 13  0.50227319  0.49356319 0.10480032
## 14  0.06861447  0.06704409 0.08643549
## 15  0.93871235  0.93606793 0.08186844
## 16 -1.29008121 -1.31098352 0.12001347
## 17  1.06516817  1.06859785 0.10343469
## 18  0.45890058  0.45051112 0.12193265
## 19  0.23755461  0.23239110 0.10805628
## 20  1.38973242  1.42161443 0.04635041
## 21 -1.09213214 -1.09717908 0.27411620
## 22 -0.50097821 -0.49227596 0.26244274
## 23  0.02037349  0.01990526 0.09076853
## 24 -1.56186889 -1.61831671 0.18011950
## 25 -1.49330686 -1.53905802 0.27174557
## 26 -0.39509397 -0.38738692 0.05564326
## 27 -1.32177835 -1.34593675 0.69210833
\end{verbatim}

\hypertarget{uxc601uxd5a5uxc810-uxce21uxb3c4-1}{%
\subsection{영향점 측도}\label{uxc601uxd5a5uxc810-uxce21uxb3c4-1}}

\begin{Shaded}
\begin{Highlighting}[]
\CommentTok{\# DFBETAS for each model variable, DFFITS, covariance ratios, }
\CommentTok{\# Cook\textquotesingle{}s distances and the diagonal elements of the hat matrix}
\CommentTok{\# Cases which are influential with respect to any of these measures }
\CommentTok{\# are marked with an asterisk.}
\FunctionTok{influence.measures}\NormalTok{(house.lm)}
\end{Highlighting}
\end{Shaded}

\begin{verbatim}
## Influence measures of
##   lm(formula = price ~ tax + ground + floor + year, data = houseprice) :
## 
##       dfb.1_   dfb.tax dfb.grnd dfb.flor  dfb.year    dffit cov.r   cook.d    hat inf
## 1   0.014217  0.002017 -0.01275 -0.00268 -0.000135  0.02658 1.389 1.48e-04 0.0932    
## 2   0.042404  0.079930  0.14075 -0.15424 -0.158274 -0.35404 1.455 2.57e-02 0.2191    
## 3   0.069922 -0.027157 -0.08571  0.04958 -0.034094  0.10940 1.554 2.50e-03 0.1972    
## 4  -0.007662  0.042042  0.03758 -0.03751 -0.069086 -0.16624 1.356 5.73e-03 0.1151    
## 5   0.063837 -0.019261 -0.03297 -0.00264  0.006118  0.13658 1.265 3.86e-03 0.0645    
## 6  -0.054546 -0.098819  0.13310 -0.13410  0.454340  0.77620 0.631 1.07e-01 0.1357    
## 7   0.006212 -0.003046 -0.00704  0.00782 -0.014422 -0.02043 1.396 8.75e-05 0.0969    
## 8  -0.005790  0.732398 -1.01363 -0.05094  0.080104 -1.28691 0.632 2.84e-01 0.2634   *
## 9  -0.485623  0.145037 -0.28210  0.47003  0.181974  0.97327 2.578 1.92e-01 0.5824   *
## 10 -0.441587 -0.024860  0.39734  0.44403 -0.383840  1.61800 0.822 4.57e-01 0.3838   *
## 11  0.135462 -0.079420  0.10301 -0.05113 -0.073622  0.28134 1.045 1.58e-02 0.0666    
## 12 -0.290579  0.312281  0.18500 -0.33279  0.268032 -0.44505 1.254 3.98e-02 0.1822    
## 13  0.080276  0.061427  0.01756 -0.11076 -0.027110  0.16887 1.331 5.91e-03 0.1048    
## 14  0.006784 -0.000403  0.01053 -0.01017 -0.001943  0.02062 1.380 8.91e-05 0.0864    
## 15  0.028693  0.027810  0.05281 -0.12180  0.117515  0.27952 1.120 1.57e-02 0.0819    
## 16  0.148557 -0.215769  0.20142  0.00669 -0.261136 -0.48414 0.968 4.54e-02 0.1200    
## 17  0.246376 -0.173013  0.01794  0.10015 -0.265665  0.36296 1.080 2.62e-02 0.1034    
## 18  0.103552  0.009435  0.01415 -0.03784 -0.110898  0.16788 1.370 5.85e-03 0.1219    
## 19  0.014602  0.036836  0.02175 -0.04407 -0.019199  0.08089 1.397 1.37e-03 0.1081    
## 20  0.091734 -0.009786  0.01097 -0.05796  0.042615  0.31341 0.836 1.88e-02 0.0464    
## 21 -0.450615  0.123028  0.53573 -0.37690  0.450929 -0.67424 1.316 9.01e-02 0.2741    
## 22  0.205044 -0.042537 -0.22449  0.10230 -0.199701 -0.29365 1.615 1.79e-02 0.2624    
## 23 -0.000357 -0.003982  0.00133  0.00316  0.000992  0.00629 1.388 8.29e-06 0.0908    
## 24  0.498955 -0.067078 -0.55352  0.21412 -0.497936 -0.75852 0.855 1.07e-01 0.1801    
## 25 -0.628635  0.164535 -0.05777 -0.02051  0.792697 -0.94014 1.015 1.66e-01 0.2717    
## 26 -0.002774 -0.019178  0.01071 -0.01033  0.016113 -0.09403 1.289 1.84e-03 0.0556    
## 27  0.109605 -1.912785  0.46868  1.41138 -0.355827 -2.01796 2.710 7.85e-01 0.6921   *
\end{verbatim}

\begin{Shaded}
\begin{Highlighting}[]
\FunctionTok{data.frame}\NormalTok{(}\FunctionTok{influence.measures}\NormalTok{(house.lm)}\SpecialCharTok{$}\NormalTok{infmat) }\SpecialCharTok{\%\textgreater{}\%} \FunctionTok{arrange}\NormalTok{(}\FunctionTok{desc}\NormalTok{(cook.d))}
\end{Highlighting}
\end{Shaded}

\begin{verbatim}
##           dfb.1_       dfb.tax     dfb.grnd     dfb.flor      dfb.year        dffit     cov.r       cook.d        hat
## 27  0.1096047217 -1.9127853141  0.468676877  1.411381453 -0.3558265703 -2.017960765 2.7098290 7.854588e-01 0.69210833
## 10 -0.4415870543 -0.0248604761  0.397338451  0.444034941 -0.3838396389  1.617995073 0.8224149 4.570343e-01 0.38378669
## 8  -0.0057902619  0.7323984486 -1.013634156 -0.050938912  0.0801037723 -1.286913158 0.6323020 2.842916e-01 0.26336419
## 9  -0.4856234858  0.1450371174 -0.282098491  0.470027818  0.1819741325  0.973272210 2.5775061 1.922563e-01 0.58244898
## 25 -0.6286351847  0.1645351185 -0.057767986 -0.020507197  0.7926966789 -0.940144619 1.0154477 1.664207e-01 0.27174557
## 24  0.4989545517 -0.0670775894 -0.553522590  0.214118018 -0.4979362714 -0.758522754 0.8551849 1.071838e-01 0.18011950
## 6  -0.0545461054 -0.0988187998  0.133102135 -0.134103045  0.4543397926  0.776200207 0.6310808 1.067389e-01 0.13574846
## 21 -0.4506145915  0.1230280032  0.535727211 -0.376896955  0.4509288359 -0.674235034 1.3155567 9.008406e-02 0.27411620
## 16  0.1485568463 -0.2157694362  0.201415137  0.006694886 -0.2611364532 -0.484143630 0.9676585 4.539605e-02 0.12001347
## 12 -0.2905789777  0.3122807716  0.185004213 -0.332794386  0.2680322860 -0.445053871 1.2541060 3.981541e-02 0.18220181
## 17  0.2463756161 -0.1730128116  0.017939895  0.100146357 -0.2656652976  0.362958055 1.0800825 2.617885e-02 0.10343469
## 2   0.0424037098  0.0799299689  0.140753289 -0.154237840 -0.1582742772 -0.354041305 1.4545976 2.571587e-02 0.21913894
## 20  0.0917343695 -0.0097856304  0.010973417 -0.057957406  0.0426148630  0.313410964 0.8358047 1.877401e-02 0.04635041
## 22  0.2050435817 -0.0425371361 -0.224492803  0.102295021 -0.1997011456 -0.293648675 1.6154977 1.786103e-02 0.26244274
## 11  0.1354618103 -0.0794200881  0.103010169 -0.051125306 -0.0736223445  0.281343734 1.0450182 1.575232e-02 0.06658106
## 15  0.0286930080  0.0278098038  0.052808877 -0.121803476  0.1175145249  0.279520178 1.1203321 1.571472e-02 0.08186844
## 13  0.0802759117  0.0614266653  0.017558647 -0.110759188 -0.0271098481  0.168874509 1.3306151 5.906805e-03 0.10480032
## 18  0.1035515927  0.0094345541  0.014153127 -0.037842744 -0.1108978787  0.167881022 1.3696294 5.848701e-03 0.12193265
## 4  -0.0076624989  0.0420420176  0.037576080 -0.037512885 -0.0690856639 -0.166244194 1.3559605 5.732622e-03 0.11507835
## 5   0.0638374271 -0.0192611164 -0.032971284 -0.002640161  0.0061177215  0.136580011 1.2651222 3.858725e-03 0.06445355
## 3   0.0699218586 -0.0271569355 -0.085706412  0.049581510 -0.0340940269  0.109396575 1.5538338 2.501697e-03 0.19724713
## 26 -0.0027742220 -0.0191783392  0.010712437 -0.010330051  0.0161132923 -0.094033628 1.2894913 1.839532e-03 0.05564326
## 19  0.0146019324  0.0368361881  0.021746080 -0.044073810 -0.0191988608  0.080886448 1.3966972 1.367318e-03 0.10805628
## 1   0.0142171388  0.0020173997 -0.012754491 -0.002679822 -0.0001347719  0.026582627 1.3893053 1.480086e-04 0.09320631
## 14  0.0067843423 -0.0004028513  0.010527346 -0.010173078 -0.0019433376  0.020622293 1.3797899 8.908700e-05 0.08643549
## 7   0.0062121502 -0.0030464468 -0.007037304  0.007822060 -0.0144221961 -0.020434236 1.3959920 8.747214e-05 0.09690865
## 23 -0.0003574067 -0.0039823274  0.001330353  0.003161260  0.0009924586  0.006289241 1.3877189 8.287463e-06 0.09076853
\end{verbatim}

\hypertarget{uxbd80uxbd84uxd68cuxadc0uxadf8uxb9bc-1}{%
\subsection{부분회귀그림}\label{uxbd80uxbd84uxd68cuxadc0uxadf8uxb9bc-1}}

\begin{Shaded}
\begin{Highlighting}[]
\CommentTok{\#added variable plot}
\FunctionTok{avPlots}\NormalTok{(house.lm)}
\end{Highlighting}
\end{Shaded}

\includegraphics{lmpractice_files/figure-latex/unnamed-chunk-63-1.pdf}

  \bibliography{book.bib,packages.bib}

\end{document}
