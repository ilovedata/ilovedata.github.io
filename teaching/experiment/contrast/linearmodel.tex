% Options for packages loaded elsewhere
\PassOptionsToPackage{unicode}{hyperref}
\PassOptionsToPackage{hyphens}{url}
%
\documentclass[
]{book}
\usepackage{lmodern}
\usepackage{amsmath}
\usepackage{ifxetex,ifluatex}
\ifnum 0\ifxetex 1\fi\ifluatex 1\fi=0 % if pdftex
  \usepackage[T1]{fontenc}
  \usepackage[utf8]{inputenc}
  \usepackage{textcomp} % provide euro and other symbols
  \usepackage{amssymb}
\else % if luatex or xetex
  \usepackage{unicode-math}
  \defaultfontfeatures{Scale=MatchLowercase}
  \defaultfontfeatures[\rmfamily]{Ligatures=TeX,Scale=1}
\fi
% Use upquote if available, for straight quotes in verbatim environments
\IfFileExists{upquote.sty}{\usepackage{upquote}}{}
\IfFileExists{microtype.sty}{% use microtype if available
  \usepackage[]{microtype}
  \UseMicrotypeSet[protrusion]{basicmath} % disable protrusion for tt fonts
}{}
\makeatletter
\@ifundefined{KOMAClassName}{% if non-KOMA class
  \IfFileExists{parskip.sty}{%
    \usepackage{parskip}
  }{% else
    \setlength{\parindent}{0pt}
    \setlength{\parskip}{6pt plus 2pt minus 1pt}}
}{% if KOMA class
  \KOMAoptions{parskip=half}}
\makeatother
\usepackage{xcolor}
\IfFileExists{xurl.sty}{\usepackage{xurl}}{} % add URL line breaks if available
\IfFileExists{bookmark.sty}{\usepackage{bookmark}}{\usepackage{hyperref}}
\hypersetup{
  pdftitle={일원배치법과 선형모형},
  pdfauthor={서울시립대 통계학과},
  hidelinks,
  pdfcreator={LaTeX via pandoc}}
\urlstyle{same} % disable monospaced font for URLs
\usepackage{color}
\usepackage{fancyvrb}
\newcommand{\VerbBar}{|}
\newcommand{\VERB}{\Verb[commandchars=\\\{\}]}
\DefineVerbatimEnvironment{Highlighting}{Verbatim}{commandchars=\\\{\}}
% Add ',fontsize=\small' for more characters per line
\usepackage{framed}
\definecolor{shadecolor}{RGB}{248,248,248}
\newenvironment{Shaded}{\begin{snugshade}}{\end{snugshade}}
\newcommand{\AlertTok}[1]{\textcolor[rgb]{0.94,0.16,0.16}{#1}}
\newcommand{\AnnotationTok}[1]{\textcolor[rgb]{0.56,0.35,0.01}{\textbf{\textit{#1}}}}
\newcommand{\AttributeTok}[1]{\textcolor[rgb]{0.77,0.63,0.00}{#1}}
\newcommand{\BaseNTok}[1]{\textcolor[rgb]{0.00,0.00,0.81}{#1}}
\newcommand{\BuiltInTok}[1]{#1}
\newcommand{\CharTok}[1]{\textcolor[rgb]{0.31,0.60,0.02}{#1}}
\newcommand{\CommentTok}[1]{\textcolor[rgb]{0.56,0.35,0.01}{\textit{#1}}}
\newcommand{\CommentVarTok}[1]{\textcolor[rgb]{0.56,0.35,0.01}{\textbf{\textit{#1}}}}
\newcommand{\ConstantTok}[1]{\textcolor[rgb]{0.00,0.00,0.00}{#1}}
\newcommand{\ControlFlowTok}[1]{\textcolor[rgb]{0.13,0.29,0.53}{\textbf{#1}}}
\newcommand{\DataTypeTok}[1]{\textcolor[rgb]{0.13,0.29,0.53}{#1}}
\newcommand{\DecValTok}[1]{\textcolor[rgb]{0.00,0.00,0.81}{#1}}
\newcommand{\DocumentationTok}[1]{\textcolor[rgb]{0.56,0.35,0.01}{\textbf{\textit{#1}}}}
\newcommand{\ErrorTok}[1]{\textcolor[rgb]{0.64,0.00,0.00}{\textbf{#1}}}
\newcommand{\ExtensionTok}[1]{#1}
\newcommand{\FloatTok}[1]{\textcolor[rgb]{0.00,0.00,0.81}{#1}}
\newcommand{\FunctionTok}[1]{\textcolor[rgb]{0.00,0.00,0.00}{#1}}
\newcommand{\ImportTok}[1]{#1}
\newcommand{\InformationTok}[1]{\textcolor[rgb]{0.56,0.35,0.01}{\textbf{\textit{#1}}}}
\newcommand{\KeywordTok}[1]{\textcolor[rgb]{0.13,0.29,0.53}{\textbf{#1}}}
\newcommand{\NormalTok}[1]{#1}
\newcommand{\OperatorTok}[1]{\textcolor[rgb]{0.81,0.36,0.00}{\textbf{#1}}}
\newcommand{\OtherTok}[1]{\textcolor[rgb]{0.56,0.35,0.01}{#1}}
\newcommand{\PreprocessorTok}[1]{\textcolor[rgb]{0.56,0.35,0.01}{\textit{#1}}}
\newcommand{\RegionMarkerTok}[1]{#1}
\newcommand{\SpecialCharTok}[1]{\textcolor[rgb]{0.00,0.00,0.00}{#1}}
\newcommand{\SpecialStringTok}[1]{\textcolor[rgb]{0.31,0.60,0.02}{#1}}
\newcommand{\StringTok}[1]{\textcolor[rgb]{0.31,0.60,0.02}{#1}}
\newcommand{\VariableTok}[1]{\textcolor[rgb]{0.00,0.00,0.00}{#1}}
\newcommand{\VerbatimStringTok}[1]{\textcolor[rgb]{0.31,0.60,0.02}{#1}}
\newcommand{\WarningTok}[1]{\textcolor[rgb]{0.56,0.35,0.01}{\textbf{\textit{#1}}}}
\usepackage{longtable,booktabs}
\usepackage{calc} % for calculating minipage widths
% Correct order of tables after \paragraph or \subparagraph
\usepackage{etoolbox}
\makeatletter
\patchcmd\longtable{\par}{\if@noskipsec\mbox{}\fi\par}{}{}
\makeatother
% Allow footnotes in longtable head/foot
\IfFileExists{footnotehyper.sty}{\usepackage{footnotehyper}}{\usepackage{footnote}}
\makesavenoteenv{longtable}
\usepackage{graphicx}
\makeatletter
\def\maxwidth{\ifdim\Gin@nat@width>\linewidth\linewidth\else\Gin@nat@width\fi}
\def\maxheight{\ifdim\Gin@nat@height>\textheight\textheight\else\Gin@nat@height\fi}
\makeatother
% Scale images if necessary, so that they will not overflow the page
% margins by default, and it is still possible to overwrite the defaults
% using explicit options in \includegraphics[width, height, ...]{}
\setkeys{Gin}{width=\maxwidth,height=\maxheight,keepaspectratio}
% Set default figure placement to htbp
\makeatletter
\def\fps@figure{htbp}
\makeatother
\setlength{\emergencystretch}{3em} % prevent overfull lines
\providecommand{\tightlist}{%
  \setlength{\itemsep}{0pt}\setlength{\parskip}{0pt}}
\setcounter{secnumdepth}{5}
\usepackage[onehalfspacing]{setspace}

\usepackage[hangul]{kotex}
\usepackage{fullpage}
\newcommand{\pardiff}[2]{\frac{\partial #1}{\partial #2 }}
\newcommand{\pardiffl}[2]{{\partial #1}/{\partial #2 }}
\newcommand{\pardiffd}[2]{\frac{\partial^2 #1}{\partial #2^t \partial #2 }}
\newcommand{\pardiffdd}[3]{\frac{\partial^2 #1}{\partial #2 \partial #3 }}

\newcommand{\bm}[1]{\boldsymbol{\mathbf{#1}}}

\usepackage{booktabs}
\usepackage{longtable}
\usepackage[bf,singlelinecheck=off]{caption}

%\setmainfont[UprightFeatures={SmallCapsFont=AlegreyaSC-Regular}]{Alegreya}

\usepackage{framed,color}
\definecolor{shadecolor}{RGB}{248,248,248}

\renewcommand{\textfraction}{0.05}
\renewcommand{\topfraction}{0.8}
\renewcommand{\bottomfraction}{0.8}
\renewcommand{\floatpagefraction}{0.75}

\renewenvironment{quote}{\begin{VF}}{\end{VF}}
\let\oldhref\href
\renewcommand{\href}[2]{#2\footnote{\url{#1}}}

\makeatletter
\newenvironment{kframe}{%
\medskip{}
\setlength{\fboxsep}{.8em}
 \def\at@end@of@kframe{}%
 \ifinner\ifhmode%
  \def\at@end@of@kframe{\end{minipage}}%
  \begin{minipage}{\columnwidth}%
 \fi\fi%
 \def\FrameCommand##1{\hskip\@totalleftmargin \hskip-\fboxsep
 \colorbox{shadecolor}{##1}\hskip-\fboxsep
     % There is no \\@totalrightmargin, so:
     \hskip-\linewidth \hskip-\@totalleftmargin \hskip\columnwidth}%
 \MakeFramed {\advance\hsize-\width
   \@totalleftmargin\z@ \linewidth\hsize
   \@setminipage}}%
 {\par\unskip\endMakeFramed%
 \at@end@of@kframe}
\makeatother

\makeatletter

\@ifundefined{Shaded}{
}{\renewenvironment{Shaded}{\begin{kframe}}{\end{kframe}}}
\makeatother

\newenvironment{rmdblock}[1]
  {
  \begin{itemize}
  \renewcommand{\labelitemi}{
    \raisebox{-.7\height}[0pt][0pt]{
      {\setkeys{Gin}{width=3em,keepaspectratio}\includegraphics{images/#1}}
    }
  }
  \setlength{\fboxsep}{1em}
  \begin{kframe}
  \item
  }
  {
  \end{kframe}
  \end{itemize}
  }
  
\newenvironment{rmdnote}
  {\begin{rmdblock}{note}}
  {\end{rmdblock}}
  
\newenvironment{rmdcaution}
  {\begin{rmdblock}{caution}}
  {\end{rmdblock}}
  
\newenvironment{rmdimportant}
  {\begin{rmdblock}{important}}
  {\end{rmdblock}}
  
\newenvironment{rmdtip}
  {\begin{rmdblock}{tip}}
  {\end{rmdblock}}
  
\newenvironment{rmdwarning}
  {\begin{rmdblock}{warning}}
  {\end{rmdblock}}
  


%\usepackage{makeidx}
%\makeindex

\urlstyle{tt}

\usepackage{amsthm}
\makeatletter
 \def\thm@space@setup{%
   \thm@preskip=8pt plus 2pt minus 4pt
   \thm@postskip=\thm@preskip
}
\makeatother

\frontmatter

\ifluatex
  \usepackage{selnolig}  % disable illegal ligatures
\fi
\usepackage[]{natbib}
\bibliographystyle{apalike}

\title{일원배치법과 선형모형}
\author{서울시립대 통계학과}
\date{2021-04-06}

\begin{document}
\maketitle

{
\setcounter{tocdepth}{1}
\tableofcontents
}
\hypertarget{uxc11cuxbb38}{%
\chapter*{서문}\label{uxc11cuxbb38}}


일원배치 실험계획법의 목적은 서로 다른 처리의 효과가 같은지 다른지 알아보는 것이다. 따라서 분삽분석표를 이용하여 다음과 같은 가설을 검정한다.

\[ H_0 : \alpha_1 = \alpha_2 = \cdots = \alpha_a \]

만약 위의 귀무가설을 기각하지 못했다면 처리의 효과들이 모두 같으므로 더 이상의 추론은 소용이 없다.
하지만 만약 귀무가설을 기각하게 되면 처리 효과들이 어떻게 다른지 추론해 보아야 한다.

교과서 3.5 절과 강의노트에서 각 처리에 대한 반응값 의 평균 \(\mu+\alpha_i\)과 각 처리간의 차이 \(\alpha_i - \alpha_j\)에 대한 신뢰구간과 가설 검정을 다루었다.

이 강의에서는 일원배치에서 처리 효과를 비교하는 통계적 방법들에 대하여 더욱 자세하게 알아보려고 한다.

\begin{rmdcaution}
교과서에서는 반응 변수를 \(x\) 로 표현하였는데 이 강의에서는 \(y\) 로 표시할 것이다.
\end{rmdcaution}

\begin{center}\rule{0.5\linewidth}{0.5pt}\end{center}

이 노트는 분산분석 후에 여러 개의 수준에 대한 비교를 할 때 통계적 방법에 대한 이론과 예제에 대한 강의자료입니다.
이 노트에 있는 R 프로그램을 실행하려면 다음과 같은 패키지들이 필요하다.

\begin{verbatim}
library(dplyr)
library(tidyr)
library(ggplot2)
library(agricolae)
library(emmeans)
\end{verbatim}

\mainmatter

\hypertarget{onewaylse}{%
\chapter{일원배치 모형과 최소제곱법}\label{onewaylse}}

\hypertarget{uxcd5cuxc18cuxc81cuxacf1uxbc95uxacfc-uxc81cuxc57duxc870uxac74}{%
\section{최소제곱법과 제약조건}\label{uxcd5cuxc18cuxc81cuxacf1uxbc95uxacfc-uxc81cuxc57duxc870uxac74}}

이제 일원배치법에 대한 통계적 모형에서 모수에 대한 추정을 생각해 보자.

\begin{equation}
y_{ij} = \mu + \alpha_i + e_{ij} 
\label{eq:oneway}
\end{equation}

추정해야할 모수는 전체 평균 \(\mu\)와 각 그룹의 처리 효과 \(\alpha_i\) 그리고 분산 \(\sigma_E^2\)이다. 전체 평균과 그룹의 효과는 오차제곱합(Sum of Square Error; SSE)을 최소로 하는 모수를 추정하는 최소제곱법(Least Square method; LS)으로 구할 수 있다.

\begin{equation} 
 \min_{\mu, \alpha_1, \dots \alpha_a} \sum_{i=1}^a \sum_{j=1}^r 
(y_{ij} - \mu -\alpha_i)^2 =\min_{\mu, \alpha_1, \dots \alpha_a} SSE 
\label{eq:lsesse}
\end{equation}

위의 오차제곱합이 모든 모수에 대하여 미분가능한 이차식으므로 최소제곱 추정량은 제곱합을 모수에 대하여 미분하고 0 으로 놓아 방정식을 풀어서 얻을 수 있다.

오차제곱합을 모수 \(\mu\)와 \(\alpha_1,\alpha_2,\dots,\alpha_a\) 로 미분하여 0 으로 놓은 방정식은 다음과 같다.

\begin{align*}
& \pardiff{}{\mu} SSE = -2 \sum_{i=1}^a \sum_{j=1}^r (y_{ij} - \mu -\alpha_i) = 0 \\
& \pardiff{}{\alpha_i} SSE = -2 \sum_{j=1}^r (y_{ij} - \mu -\alpha_i) = 0 , \quad i=1,2,\dots, a 
\end{align*}

위의 방정식을 정리하면 다음과 같은 \(a+1\)개의 방정식을 얻는다.

\begin{align}
   \mu +\frac{ \sum_{i=1}^a \alpha_i}{a} & = \bar {\bar y}\\
   \mu + \alpha_1  & =  \bar {y}_{1.} \\
   \mu + \alpha_2  & =  \bar {y}_{2.} \\
         \cdots & \cdots \\
   \mu + \alpha_a  & =  \bar {y}_{a.} \\
\label{eq:normaleq1}   
\end{align}

위의 방정식에서 첫 번째 방정식은 다른 \(a\)개의 방정식을 모두 합한 방정식과 같다. 따라서 모수는 \(a+1\)개이지만 실제 방정식의 개수는 \(a\)개이므로
유일한 해가 얻어지지 않는다. 따라서 유일한 해를 구하려면 하나의 제약조건이 필요하며 일반적으로 다음과 같은 두 개의 조건 중 하나를 사용한다.

\hypertarget{set-to-zero-condition}{%
\subsection{set-to-zero condition}\label{set-to-zero-condition}}

첫 번째 효과 \(\alpha_1\)를 0으로 놓는 조건을 주는 것이다 (\(\alpha_1=0\)). set-to-zero 조건 하에서는 다음과 같은 추정량이 얻어진다.

\begin{equation}
\hat \mu = \bar {y}_{1.}, \quad \hat \alpha_1=0, ~~  \hat \alpha_i = \bar {y}_{i.} -\bar {y}_{1.},~~i=2,\dots,a
\label{eq:setzeroest}
\end{equation}

\hypertarget{sum-to-zero-condition}{%
\subsection{sum-to-zero condition}\label{sum-to-zero-condition}}

처리들의 효과의 합은 0이라는 조건을 주는 것이다 ( \(\sum_{i=1}^a \alpha_i=0\)). sum-to-zero 조건에서는 계수의 추정치가 다음과 같이 주어진다.

\begin{equation}
\hat \mu = \bar {\bar {y}}, \quad \hat \alpha_i = \bar {y}_{i.} -\bar {\bar {y}},~~i=1,2,\dots,a 
\label{eq:sumzeroest}
\end{equation}

여기서 유의할 점은 \textbf{개별 모수들의 추정량은 조건에 따라서 달라지지만 집단의 평균을 나타내는 모수 \(\mu+ \alpha_i\) 에 대한 추정량은 언제나 같다}.

\[ \widehat{\mu+ \alpha_i} = \hat \mu + \hat {\alpha}_i =  \bar {y}_{i.} \]

만약에 자료를 아래와 같은 평균 모형으로 나타낼 경우에는 각 평균 \(\mu_i\) 는 각 그룹의 표본 평균으로 추정된다.

\[ y_{ij} = \mu_i + e_{ij} \]

평균 모형에서 각 그룹의 모평균에 대한 최소제곱 추정량은 \(\hat \mu_i = \bar {y}_{i.}\) 이며 이는 주효과 모형에서의 추정량과 동일하다.

또한 모형에 관계없이 오차항의 분산 \(\sigma_E^2\) 에 대한 추정량은 다음과 같이 주어진다.

\begin{equation*} 
\hat \sigma_E^2 = \frac{ \sum_i \sum_j (y_{ij} - \hat \mu -\hat \alpha_i )^2}{a(r-1)}
\end{equation*}

\hypertarget{uxc120uxd615uxbaa8uxd615uxacfc-uxc81cuxc57d-uxc870uxac74}{%
\section{선형모형과 제약 조건}\label{uxc120uxd615uxbaa8uxd615uxacfc-uxc81cuxc57d-uxc870uxac74}}

일원배치 모형 \eqref{eq:oneway}를 다음과 같은 벡터를 이용한 선형모형(linear model, regression model) 형태로 나타내고자 한다.

\begin{equation}
\bm y = \bm X \bm \beta +\bm e
\label{eq:lm}
\end{equation}

위의 선형모형식의 요소 \(\bm y\), \(\bm X\), \(\bm \beta\), \(\bm e\)는 다음과 같은 벡터와 행렬로 표현된다.

\begin{equation}
\begin{bmatrix}
y_{11} \\
y_{12} \\
\vdots \\
y_{1r} \\
y_{21} \\
y_{22} \\
\vdots \\
y_{2r} \\
\vdots \\
y_{a1} \\
y_{a2} \\
\vdots \\
y_{ar} \\
\end{bmatrix} 
 =
\begin{bmatrix}
1 & 1 & 0 & . & . & 0 \\
1 & 1 & 0 & . & . & 0 \\
1 & \vdots & \vdots & \vdots & \vdots & \vdots \\
1 & 1 & 0 & . & . & 0 \\
1 & 0 & 1 & . & . & 0 \\
1 & 0 & 1 & . & . & 0 \\
1 & \vdots & \vdots & \vdots & \vdots & \vdots \\
1 & 0 & 1 & . & . & 0 \\
\vdots & \vdots & \vdots & \vdots & \vdots & \vdots \\
1 & 0 & 0 & . & . & 1 \\
1 & 0 & 0 & . & . & 1 \\
1 & \vdots & \vdots & \vdots & \vdots & \vdots \\
1 & 0 & 0 & . & . & 1 \\
\end{bmatrix}
\begin{bmatrix}
\mu \\
\alpha_{1} \\
\alpha_{2} \\
\vdots \\
\alpha_{a} \\
\end{bmatrix} +
\begin{bmatrix}
e_{11} \\
e_{12} \\
\vdots \\
e_{1r} \\
e_{21} \\
e_{22} \\
\vdots \\
e_{2r} \\
\vdots \\
e_{a1} \\
e_{a2} \\
\vdots \\
e_{ar} \\
\end{bmatrix}
\label{eq:lm2}
\end{equation}

이제 위에서 논의한 최소제곱법을 선형 모형 \eqref{eq:lm} 에 적용하면 다음과 같이 표현할 수 있다.

\begin{equation} 
 \min_{\mu, \alpha_1, \dots \alpha_a} \sum_{i=1}^a \sum_{j=1}^r 
(y_{ij} - \mu -\alpha_i)^2 = \min_{\bm \beta } ( \bm y -  \bm X \bm \beta )^t( \bm y -  \bm X \bm \beta ) 
 \label{eq:rsq2}
 \end{equation}

최소제곱법의 기준을 만족하는 계수 \(\bm \beta\)는 다음과 같은 정규방정식(normal equation)의 해(solution)이다.

\begin{equation}
\bm X^t \bm X \bm \beta = \bm X^t \bm y
\label{eq:normaleq2}
\end{equation}

정규방정식 \eqref{eq:normaleq2} 을 일워배치의 선형모형식 \eqref{eq:lm2} 에 나타난 \(\bm y\), \(\bm X\)로 이용하여 나타내면 다음과 같다.

\begin{equation}
\begin{bmatrix}
ar   & r & r & \cdot & \cdot & r \\
r & r &  0  & \cdot & \cdot & 0 \\
r & 0   & r  & \cdot & \cdot & 0 \\
\cdot & \cdot   & \cdot  & \cdot & \cdot & \cdot \\
\cdot & \cdot   & \cdot  & \cdot & \cdot & \cdot \\
r & 0   &  0   & \cdot & \cdot & r \\
\end{bmatrix}
\begin{bmatrix}
\mu \\
\alpha_{1} \\
\alpha_{2} \\
\cdot \\
\cdot \\
\alpha_{a} \\
\end{bmatrix}
=
\begin{bmatrix}
ar \bar {\bar y} \\
r {\bar y}_{1.}\\
r \bar y_{2.}\\
\cdot \\
\cdot \\
r \bar y_{a.}
\end{bmatrix}
\label{eq:normaleq3}
\end{equation}

정규방정식 \eqref{eq:normaleq3} 는 위에서 구한 최소제곱법에서 유도된 방정식 \eqref{eq:normaleq1} 과 같다.

여기서 유의할 점은 선형모형식 \eqref{eq:lm2} 의 계획행렬 \(\bm X\) 가 완전 계수(full rank) 행렬이 아니다.
계획행렬 \(\bm X\)의 첫 번째 열은 다른 열을 합한 것과 같다.
또한 정규 방정식 \eqref{eq:normaleq3}에서 \(\bm X^t \bm X\) 행렬도 완전계수 행렬이 아니다.
따라서 \(\bm X^t \bm X\) 행렬의 역행렬은 존재하지 않는다.

이러한 이유로 모수에 대한 유일한 추정량이 존재하지 않기 때문에 앞에서 언급한 제약 조건을 고려해야 정규방정식을 풀 수 있다.

\hypertarget{set-to-zero-uxc870uxac74uxc5d0uxc11cuxc758-uxbaa8uxd615uxacfc-uxcd5cuxc18cuxc81cuxacf1-uxcd94uxc815uxb7c9}{%
\subsection{Set-to-zero 조건에서의 모형과 최소제곱 추정량}\label{set-to-zero-uxc870uxac74uxc5d0uxc11cuxc758-uxbaa8uxd615uxacfc-uxcd5cuxc18cuxc81cuxacf1-uxcd94uxc815uxb7c9}}

만약 Set-to-zero 조건을 가정한다면 모수에서 \(\alpha_1\)을 제외하고 선형모형식 \eqref{eq:lm2}를 다음과 같이 다시 표현할 수 있다.\\
효과 \(\alpha_1\)을 0 으로 놓는다는 것은 \(\alpha_1\)을 추정할 필요가 없으므로 모수벡터 \(\bm \beta\) 에서 \(\alpha_1\)를 빼고
게획행렬에서도 대응하는 열을 제거하는 것이다.

\begin{equation}
\begin{bmatrix}
y_{11} \\
y_{12} \\
\vdots \\
y_{1r} \\
y_{21} \\
y_{22} \\
\vdots \\
y_{2r} \\
\vdots \\
y_{a1} \\
y_{a2} \\
\vdots \\
y_{ar} \\
\end{bmatrix} 
 =
\begin{bmatrix}
1 &  0 & . & . & 0 \\
1 &  0 & . & . & 0 \\
1 &  \vdots & \vdots & \vdots & \vdots \\
1 &  0 & . & . & 0 \\
1 &  1 & . & . & 0 \\
1 &  1 & . & . & 0 \\
1 &  \vdots & \vdots & \vdots & \vdots \\
1 &  1 & . & . & 0 \\
\vdots &  \vdots & \vdots & \vdots & \vdots \\
1 &  0 & . & . & 1 \\
1 &  0 & . & . & 1 \\
1 &  \vdots & \vdots & \vdots & \vdots \\
1 &  0 & . & . & 1 \\
\end{bmatrix}
\begin{bmatrix}
\mu \\
\alpha_{2} \\
\alpha_{3} \\
\vdots \\
\alpha_{a} \\
\end{bmatrix} +
\begin{bmatrix}
e_{11} \\
e_{12} \\
\vdots \\
e_{1r} \\
e_{21} \\
e_{22} \\
\vdots \\
e_{2r} \\
\vdots \\
e_{a1} \\
e_{a2} \\
\vdots \\
e_{ar} \\
\end{bmatrix}
\label{eq:lm-zero}
\end{equation}

이제 수정된 모형식 \eqref{eq:lm-zero} 에 최소제곱법을 적용하여 정규방정식을 구하면 다음과 같은 방정식을 얻는다.

\begin{equation}
\begin{bmatrix}
ar   & r & r & \cdot & \cdot & r \\
r & r &  0  & \cdot & \cdot & 0 \\
r & 0   & r  & \cdot & \cdot & 0 \\
\cdot & \cdot   & \cdot  & \cdot & \cdot & \cdot \\
\cdot & \cdot   & \cdot  & \cdot & \cdot & \cdot \\
r & 0   &  0   & \cdot & \cdot & r \\
\end{bmatrix}
\begin{bmatrix}
\mu \\
\alpha_{2} \\
\alpha_{3} \\
\cdot \\
\cdot \\
\alpha_{a} \\
\end{bmatrix}
=
\begin{bmatrix}
ar \bar {\bar y} \\
r {\bar y}_{2.}\\
r \bar y_{3.}\\
\cdot \\
\cdot \\
r \bar y_{a.}
\end{bmatrix}
\label{eq:normaleq-zero}
\end{equation}

위의 정규방정 \eqref{eq:normaleq-zero} 를 풀면 위에서 언급한 sum-to-zero 조건에서 구해지는 모수의 추정량 \eqref{eq:setzeroest}를 얻을 수 있다.

\hypertarget{sum-to-zero-uxc870uxac74uxc5d0uxc11cuxc758-uxbaa8uxd615uxacfc-uxcd5cuxc18cuxc81cuxacf1-uxcd94uxc815uxb7c9}{%
\subsection{Sum-to-zero 조건에서의 모형과 최소제곱 추정량}\label{sum-to-zero-uxc870uxac74uxc5d0uxc11cuxc758-uxbaa8uxd615uxacfc-uxcd5cuxc18cuxc81cuxacf1-uxcd94uxc815uxb7c9}}

이제 Sum-to-zero 조건에서 모수의 추정에 대해 알아보자. 조건 \(\sum_{i=1}^a \alpha_i =0\) 조건을 마지막 모수 \(\alpha_a\)에 대하여 표현하면 다음과 같다.

\[ \alpha_a = -\alpha_1 - \alpha_2 - \dots - \alpha_{a-1} \]

따라서 마지막 처리 \(\alpha_a\) 에 대한 관측값에 대한 모형은 다음과 같아 쓸 수 있다.

\[ y_{aj} = \mu + \alpha_a + e_{aj} = \mu +( -\alpha_1 - \alpha_2 - \dots - \alpha_{a-1}) + e_{ij} \]

이러한 결과를 모형방정식에 반영한다. 즉, 모수벡터 \(\bm \beta\) 에서 \(\alpha_a\)를 제거하고 게획행렬에 위의 마지막 처리에 대한 효과식을 반영하면 다음과 같은 선형모형식을 얻는다.

\begin{equation}
\begin{bmatrix}
y_{11} \\
y_{12} \\
\vdots \\
y_{1r} \\
y_{21} \\
y_{22} \\
\vdots \\
y_{2r} \\
\vdots \\
y_{a1} \\
y_{a2} \\
\vdots \\
y_{ar} \\
\end{bmatrix} 
 =
\begin{bmatrix}
1 & 1 & 0 & . & . & 0 \\
1 & 1 & 0 & . & . & 0 \\
1 & \vdots & \vdots & \vdots & \vdots & \vdots \\
1 & 1 & 0 & . & . & 0 \\
1 & 0 & 1 & . & . & 0 \\
1 & 0 & 1 & . & . & 0 \\
1 & \vdots & \vdots & \vdots & \vdots & \vdots \\
1 & 0 & 1 & . & . & 0 \\
\vdots & \vdots & \vdots & \vdots & \vdots & \vdots \\
1 & 0 & 0 & . & . & 1 \\
1 & 0 & 0 & . & . & 1 \\
1 & \vdots & \vdots & \vdots & \vdots & \vdots \\
1 & 0 & 0 & . & . & 1 \\
1 & 0 & 0 & . & . & 1 \\
1 & -1 & -1 & . & . & -1 \\
1 & -1 & -1 & . & . & -1 \\
1 & \vdots & \vdots & \vdots & \vdots & \vdots \\
1 & -1 & -1 & . & . & -1 \\
1 & -1 & -1 & . & . & -1 \\
\end{bmatrix}
\begin{bmatrix}
\mu \\
\alpha_{1} \\
\alpha_{2} \\
\vdots \\
\alpha_{a-1} \\
\end{bmatrix} +
\begin{bmatrix}
e_{11} \\
e_{12} \\
\vdots \\
e_{1r} \\
e_{21} \\
e_{22} \\
\vdots \\
e_{2r} \\
\vdots \\
e_{a1} \\
e_{a2} \\
\vdots \\
e_{ar} \\
\end{bmatrix}
\label{eq:lm-sum}
\end{equation}

이제 수정된 모형식 \eqref{eq:lm-sum} 에 최소제곱법을 적용하여 정규방정식을 구하면 다음과 같은 방정식을 얻는다.

\begin{equation}
\begin{bmatrix}
ar   & 0 & 0 & \cdot & \cdot & 0 \\
0 & 2r &  r  & \cdot & \cdot & r \\
0 & r   & 2r  & \cdot & \cdot & r \\
\cdot & \cdot   & \cdot  & \cdot & \cdot & \cdot \\
\cdot & \cdot   & \cdot  & \cdot & \cdot & \cdot \\
0 & r   &  r   & \cdot & \cdot & 2r \\
\end{bmatrix}
\begin{bmatrix}
\mu \\
\alpha_{2} \\
\alpha_{3} \\
\cdot \\
\cdot \\
\alpha_{a-1} \\
\end{bmatrix}
=
\begin{bmatrix}
ar \bar {\bar y} \\
r {\bar y}_{2.}-r {\bar y}_{a.} \\
r \bar y_{3.}-r {\bar y}_{a.}\\
\cdot \\
\cdot \\
r \bar y_{a-1,.} -r {\bar y}_{a.}
\end{bmatrix}
\label{eq:normaleq-sum}
\end{equation}

위의 정규방정 \eqref{eq:normaleq-sum} 를 풀면 위에서 언급한 sum-to-zero 조건에서 구해지는 모수의 추정량 \eqref{eq:sumzeroest}를 얻을 수 있다.

\hypertarget{estimable}{%
\chapter{추정 가능한 함수}\label{estimable}}

\hypertarget{uxc77cuxc6d0uxbc30uxce58uxbc95uxc5d0-uxcd94uxc815uxac00uxb2a5uxd55c-uxbaa8uxc218}{%
\section{일원배치법에 추정가능한 모수}\label{uxc77cuxc6d0uxbc30uxce58uxbc95uxc5d0-uxcd94uxc815uxac00uxb2a5uxd55c-uxbaa8uxc218}}

앞 절에서 보았듯이 일원배치법을 선형 모형식으로 표현하는 경우 평균에 대한 모수는 모두 \(a+1\) 개가 있다.

\[ \mu, \alpha_1, \alpha_2, \cdots, \alpha_a \]

하지만 모형식에서 계획행렬 \(\bm X\)가 완전 계수 행렬이 아니기 때문에 1개의 제약 조건을 가정하고 모수를 추정하였다.
하지만 제약 조건이 달라지면 각 모수의 추정량이 달라지기 때문에 각 모수는 유일한 값으로 추정이 불가능하다.

이렇게 각 모수들은 제약 조건에 따라서 유일하게 추정이 불가능하지만 앞 절에서 보았듯이 \(\mu + \alpha_i\) 에 대한 추정량은 제약조건에 관계없이
표본 평균 \(\bar y_{i.}\)으로 동일하게 추정되어 진다.

그러면 어떤 모수들은 유일하게 추정이 불가능하고 어떤 모수들이 유일하게 추정이 가능할까?

이제 제약조건이 달라도 유일하게 추정이 가능한 모수들의 형태를 살펴보자.

\hypertarget{uxcd94uxc815uxac00uxb2a5uxd55c-uxbaa8uxc218uxc758-uxd568uxc218}{%
\section{추정가능한 모수의 함수}\label{uxcd94uxc815uxac00uxb2a5uxd55c-uxbaa8uxc218uxc758-uxd568uxc218}}

선형모형 \(\bm y =\bm X \bm \beta + \bm e\) 에서 계획행렬 \(\bm X\)의 계수가 완전하지 않으면 모수 벡터 \(\bm \beta\)는 유일한 값으로 추정할 수 없다.

이제 모수들의 선형결합 \(\psi = \bm c^t \bm \beta\)를 고려하자.

예를 들어 일원배치 모형에서는 다음과 같은 모수들의 선형결합을 고려하는 것이다.

\[ \psi = \bm c^t \bm \beta = c_0 \mu + c_1 \alpha_1 + c_2 \alpha_2 + \cdots + c_a \alpha_a \]

위에서 본 것처럼 하나의 모수 \(\alpha_1\)에 대한 유일한 추정은 불가능하다.

\[  \alpha_1 = (0) \mu + (1) \alpha_1 + (0) \alpha_2 + \cdots + (0) \alpha_a \]

하지만 모수의 조합 \(\mu+ \alpha_2\) 은 유일한 추정이 가능하다.

\[  \mu + \alpha_1 = (1) \mu + (1) \alpha_1 + (0) \alpha_2 + \cdots + (0) \alpha_a \]

이제 문제는 선형조합 \(\psi= \bm c^t \bm \beta\) 에서 계수들 \(c_0, c_1, \dots, c_a\)가 어떤 값을 가지는 경우 유일한 추정이 가능한 지 알아내는 것이다.

이제 \(\psi = \bm c^t \bm \beta\) 에 대한 유일한 추정량 \(\hat \psi\) 이 있다고 가정하자. 선형 모형에서 추정량 \(\hat \psi\)의 형태는 관측값에 대한 선형함수가 되어야 한다. 따라서 추정량을 \(\hat \psi = \bm a^t \bm y\) 로 나타낼 수 있다. 이제 추정량 \(\hat \psi\)의 기대값은 \(\psi=\bm c^t \bm \beta\)이어야 하므로 다음이 성립해야 한다.

\[ E(\hat \psi| \bm X) = E(\bm a^t \bm y| \bm X) = \bm a^t E(\bm y| \bm X) = \bm a^t \bm X \bm \beta = \bm c^t \bm \beta \]

위의 식에서 가장 마지막 두 항의 관계를 보면 다음이 성립해야 한다.

\begin{equation}
\bm a^t \bm X = \bm c^t  \quad \text{ equivalently }\quad \bm c = \bm X^t \bm a
\label{eq:estimable}
\end{equation}

즉 추정가능한 모수의 조합 \(\psi = \bm c^t \bm \beta\)에서 \textbf{계수 벡터 \(\bm c\) 는 계획행렬에 있는 행들의 선형 조합}으로 표시되어야 한다는 것이다. 이렇게 유일하게 추정이 가능한
모수의 조합을 \textbf{추정가능한 함수(estimable function)}이라고 한다.

\hypertarget{uxc608uxc81c}{%
\section{예제}\label{uxc608uxc81c}}

2개의 수준이 있고 반복이 2번 있는 일원배치 \((a=2,r=2)\) 에 대한 선형모형 \eqref{eq:lm2}을 생각해보자. 이 경우 계획행렬 \(\bm X\) 과 모수벡터 \(\bm \beta\) 는 다음과 같다.

\begin{equation}
\bm X = 
\begin{bmatrix}
1 & 1 & 0  \\
1 & 1 & 0  \\
1 & 0 & 1  \\
1 & 0 & 1  
\end{bmatrix}
\quad 
\bm \beta = 
\begin{bmatrix}
\mu \\
\alpha_1 \\
\alpha_2 
\end{bmatrix}
\end{equation}

이제 유일하게 추정 가능한 모수 조합은 어떤 형태일까?

\[ \psi = \bm c^t \bm \beta = c_0 \mu + c_1 \alpha_1 + c_2 \alpha_2 \]

다음으로 임의의 벡터 \(\bm a\) 에 대하여 \(\bm X^t \bm a\)의 형태를 보자.

\begin{align*}
\bm X^t \bm a & = 
\begin{bmatrix}
1 & 1 & 1 & 1  \\
1 & 1 & 0 & 0  \\
0 & 0 & 1 & 1  
\end{bmatrix}
\begin{bmatrix}
a_1 \\
a_2 \\
a_3 \\
a_4 
\end{bmatrix} \\
& = 
(a_1 + a_2)
\begin{bmatrix}
1 \\
1 \\
0 
\end{bmatrix}
+ 
(a_3 + a_4)
\begin{bmatrix}
1 \\
0 \\
1 
\end{bmatrix}
\end{align*}

위의 식에서 유의할 점은 벡터 \(\bm a\)는 임의로 주어진 벡터이다.

따라서 유일하게 추정 가능한 모수의 선형조합 \(\psi = \bm c^t \bm \beta\) 에 대한 계수 벡터 \(\bm c^t =[ c_0 ~ c_1 ~ c_2]\) 는 계획행렬 \(\bm X\)의 유일한 행들의 선형 조합으로 구성되어야 한다.

\begin{equation}
\bm c =
\begin{bmatrix}
c_0 \\
c_1 \\
c_2 
\end{bmatrix}
= 
a_1
\begin{bmatrix}
1 \\
1 \\
0 
\end{bmatrix}
+ 
a_2
\begin{bmatrix}
1 \\
0 \\
1 
\end{bmatrix}
\end{equation}

\begin{itemize}
\tightlist
\item
  처리의 효과를 나타내는 모수 \(\alpha_i\)는 추정이 불가능하다.
\end{itemize}

예를 들어 첫 번째 처리에 대한 효과 모수 \(\alpha_1\) 를 선형조합으로 나타내면

\[ \alpha_1 = c_0 \mu + c_1 \alpha_1 + c_2 \alpha_2 = (0) \mu + (1) \alpha_1 + (0) \alpha_2 \]

따라서 \(\bm c^t = [0~1~0]\)을 만들수 있는 계수 \(a_1\)과 \(a_2\)를 찾아야 하는데 불가능하다. 따라서 모수 \(\alpha_1\) 은 추정 불가능하다.

\begin{equation*}
\bm c =
\begin{bmatrix}
0 \\
1 \\
0 
\end{bmatrix}
= 
a_1
\begin{bmatrix}
1 \\
1 \\
0 
\end{bmatrix}
+ 
a_2
\begin{bmatrix}
1 \\
0 \\
1 
\end{bmatrix}
\end{equation*}

\begin{itemize}
\tightlist
\item
  처리의 평균을 나타내는 모수의 조합 \(\mu + \alpha_i\)는 추정이 가능하다.
\end{itemize}

예를 들어 모수 조합 \(\mu + \alpha_1\) 를 선형조합으로 나타내면

\[ \mu + \alpha_1 = c_0 \mu + c_1 \alpha_1 + c_2 \alpha_2 = (1) \mu + (1) \alpha_1 + (0) \alpha_2 \]

따라서 \(\bm c^t = [1~1~0]\)을 만들수 있는 계수 \(a_1=1\)과 \(a_2=0\) 이므로 추정이 가능하다.

\begin{equation*}
\bm c =
\begin{bmatrix}
1 \\
1 \\
0 
\end{bmatrix}
= 
(1)
\begin{bmatrix}
1 \\
1 \\
0 
\end{bmatrix}
+ 
(0)
\begin{bmatrix}
1 \\
0 \\
1 
\end{bmatrix}
\end{equation*}

\begin{itemize}
\tightlist
\item
  처리 효과의 차이를 나타내는 모수의 조합 \(\alpha_1-\alpha_2\)는 추정이 가능하다.
\end{itemize}

\[ \alpha_1 -\alpha_2= c_0 \mu + c_1 \alpha_1 + c_2 \alpha_2 = (0) \mu + (1) \alpha_1 + (-1) \alpha_2 \]

따라서 \(\bm c^t = [0~1~-1]\)을 만들수 있는 계수 \(a_1=1\)과 \(a_2=-1\) 이므로 추정이 가능하다.

\begin{equation*}
\bm c =
\begin{bmatrix}
0 \\
1 \\
-1 
\end{bmatrix}
= 
(1)
\begin{bmatrix}
1 \\
1 \\
0 
\end{bmatrix}
+ 
(-1)
\begin{bmatrix}
1 \\
0 \\
1 
\end{bmatrix}
\end{equation*}

\hypertarget{rprogram}{%
\chapter{일원배치에서의 추정: R 실습}\label{rprogram}}

\hypertarget{uxc608uxc81c-3.1}{%
\section{예제 3.1}\label{uxc608uxc81c-3.1}}

4개의 서로 다른 원단업체에서 직물을 공급받고 있다. 공급한 직물의 긁힘에
대한 저항력을 알아보기 위하여 각 업체마다 4개의 제품을 랜덤하게 선택하여
(\(a=4\), \(r=4\)) 일원배치법에 의하여 마모도 검사을 실시하였다.

\hypertarget{uxc790uxb8ccuxc758-uxc0dduxc131}{%
\section{자료의 생성}\label{uxc790uxb8ccuxc758-uxc0dduxc131}}

\begin{Shaded}
\begin{Highlighting}[]
\NormalTok{company}\OtherTok{\textless{}{-}} \FunctionTok{as.factor}\NormalTok{(}\FunctionTok{rep}\NormalTok{(}\FunctionTok{c}\NormalTok{(}\DecValTok{1}\SpecialCharTok{:}\DecValTok{4}\NormalTok{), }\AttributeTok{each=}\DecValTok{4}\NormalTok{))}
\NormalTok{response}\OtherTok{\textless{}{-}} \FunctionTok{c}\NormalTok{(}\FloatTok{1.93}\NormalTok{, }\FloatTok{2.38}\NormalTok{, }\FloatTok{2.20}\NormalTok{, }\FloatTok{2.25}\NormalTok{,}
             \FloatTok{2.55}\NormalTok{, }\FloatTok{2.72}\NormalTok{, }\FloatTok{2.75}\NormalTok{, }\FloatTok{2.70}\NormalTok{,}
             \FloatTok{2.40}\NormalTok{, }\FloatTok{2.68}\NormalTok{, }\FloatTok{2.32}\NormalTok{, }\FloatTok{2.28}\NormalTok{,}
             \FloatTok{2.33}\NormalTok{, }\FloatTok{2.38}\NormalTok{, }\FloatTok{2.28}\NormalTok{, }\FloatTok{2.25}\NormalTok{)}
\NormalTok{df31}\OtherTok{\textless{}{-}} \FunctionTok{data.frame}\NormalTok{(}\AttributeTok{company=}\NormalTok{company, }\AttributeTok{response=}\NormalTok{ response)}
\NormalTok{df31}
\end{Highlighting}
\end{Shaded}

\begin{verbatim}
##    company response
## 1        1     1.93
## 2        1     2.38
## 3        1     2.20
## 4        1     2.25
## 5        2     2.55
## 6        2     2.72
## 7        2     2.75
## 8        2     2.70
## 9        3     2.40
## 10       3     2.68
## 11       3     2.32
## 12       3     2.28
## 13       4     2.33
## 14       4     2.38
## 15       4     2.28
## 16       4     2.25
\end{verbatim}

각 수준에 대한 표보 평균을 구해보자.

\begin{Shaded}
\begin{Highlighting}[]
\NormalTok{df31s }\OtherTok{\textless{}{-}}\NormalTok{ df31 }\SpecialCharTok{\%\textgreater{}\%} \FunctionTok{group\_by}\NormalTok{(company)  }\SpecialCharTok{\%\textgreater{}\%}  \FunctionTok{summarise}\NormalTok{(}\AttributeTok{mean=}\FunctionTok{mean}\NormalTok{(response), }\AttributeTok{median=} \FunctionTok{median}\NormalTok{(response), }\AttributeTok{sd=}\FunctionTok{sd}\NormalTok{(response), }\AttributeTok{min=}\FunctionTok{min}\NormalTok{(response), }\AttributeTok{max=}\FunctionTok{max}\NormalTok{(response))}
\NormalTok{df31s}
\end{Highlighting}
\end{Shaded}

\begin{verbatim}
## # A tibble: 4 x 6
##   company  mean median     sd   min   max
## * <fct>   <dbl>  <dbl>  <dbl> <dbl> <dbl>
## 1 1        2.19   2.22 0.189   1.93  2.38
## 2 2        2.68   2.71 0.0891  2.55  2.75
## 3 3        2.42   2.36 0.180   2.28  2.68
## 4 4        2.31   2.30 0.0572  2.25  2.38
\end{verbatim}

\hypertarget{uxc120uxd615uxbaa8uxd615uxc758-uxc801uxd569set-to-zero}{%
\section{선형모형의 적합(set-to-zero)}\label{uxc120uxd615uxbaa8uxd615uxc758-uxc801uxd569set-to-zero}}

이제 자료를 다음과 같은 선형 모형으로 적합해 보자. 선형 모형의 적합은
\texttt{lm()} 함수를 사용한다.

\[ y_{ij} = \mu + \alpha_i + e_{ij}  \]

여기서 선형식의 모수와 \texttt{R}의 변수는 다음과 같은 관계를 가진다,

\begin{longtable}[]{@{}rr@{}}
\toprule
선형식의 모수 & \texttt{R}의 변수\tabularnewline
\midrule
\endhead
\(\mu\) & \texttt{(Intercept)}\tabularnewline
\(\alpha_1\) & \texttt{company1}\tabularnewline
\(\alpha_2\) & \texttt{company2}\tabularnewline
\(\alpha_3\) & \texttt{company3}\tabularnewline
\(\alpha_4\) & \texttt{company4}\tabularnewline
\bottomrule
\end{longtable}

\begin{Shaded}
\begin{Highlighting}[]
\NormalTok{fit1 }\OtherTok{\textless{}{-}} \FunctionTok{lm}\NormalTok{(response}\SpecialCharTok{\textasciitilde{}}\NormalTok{company,}\AttributeTok{data=}\NormalTok{df31)}
\FunctionTok{summary}\NormalTok{(fit1)}
\end{Highlighting}
\end{Shaded}

\begin{verbatim}
## 
## Call:
## lm(formula = response ~ company, data = df31)
## 
## Residuals:
##     Min      1Q  Median      3Q     Max 
## -0.2600 -0.0700  0.0150  0.0625  0.2600 
## 
## Coefficients:
##             Estimate Std. Error t value Pr(>|t|)    
## (Intercept)   2.1900     0.0705   31.06  7.8e-13 ***
## company2      0.4900     0.0997    4.91  0.00036 ***
## company3      0.2300     0.0997    2.31  0.03971 *  
## company4      0.1200     0.0997    1.20  0.25198    
## ---
## Signif. codes:  0 '***' 0.001 '**' 0.01 '*' 0.05 '.' 0.1 ' ' 1
## 
## Residual standard error: 0.141 on 12 degrees of freedom
## Multiple R-squared:  0.687,  Adjusted R-squared:  0.609 
## F-statistic: 8.78 on 3 and 12 DF,  p-value: 0.00235
\end{verbatim}

위에서 적합한 결과를 보면 평균 \(\mu\)와 4개의 처리 \(\alpha_1\),
\(\alpha_2\), \(\alpha_3\), \(\alpha_4\) 가 모형에 있지만 모수의 추정량은
평균(\texttt{intercept})과 3개의 모수(\texttt{company2}, \texttt{company3}, \texttt{company4})만
추정량이 주어진다.

\texttt{R} 에서 옵션을 지정하지 않고 함수 \texttt{lm()}으로 선형모형을 적합하는 경우 set-to-zero 조건을
적용하며 자료에 나타난 처리의 수준들 중 순위가 가장 낮은 수준의 효과를
0으로 지정한다 (\texttt{company1}=0 ). set-to-zero 조건을 강제로 지정하려면 다음과 같은 명령문을 먼저 실행한다.

\begin{verbatim}
options(contrasts=c("contr.treatment", "contr.poly"))
\end{verbatim}

위의 결과를 보면 \texttt{(Intercept)}에 대한 추정량이 첫 번째 처리 \texttt{company1}의
평균과 같은 것을 알 수 있다.

set-to-zero 조건에서의 계획행렬은 다음과 같이 볼 수 있다.

\begin{Shaded}
\begin{Highlighting}[]
\FunctionTok{model.matrix}\NormalTok{(fit1)}
\end{Highlighting}
\end{Shaded}

\begin{verbatim}
##    (Intercept) company2 company3 company4
## 1            1        0        0        0
## 2            1        0        0        0
## 3            1        0        0        0
## 4            1        0        0        0
## 5            1        1        0        0
## 6            1        1        0        0
## 7            1        1        0        0
## 8            1        1        0        0
## 9            1        0        1        0
## 10           1        0        1        0
## 11           1        0        1        0
## 12           1        0        1        0
## 13           1        0        0        1
## 14           1        0        0        1
## 15           1        0        0        1
## 16           1        0        0        1
## attr(,"assign")
## [1] 0 1 1 1
## attr(,"contrasts")
## attr(,"contrasts")$company
## [1] "contr.treatment"
\end{verbatim}

이제 각 처리 평균에 대한 추정값 \(\widehat{\mu+ \alpha_i}\)을 구해보자.

\begin{Shaded}
\begin{Highlighting}[]
\FunctionTok{emmeans}\NormalTok{(fit1, }\StringTok{"company"}\NormalTok{)}
\end{Highlighting}
\end{Shaded}

\begin{verbatim}
##  company emmean     SE df lower.CL upper.CL
##  1         2.19 0.0705 12     2.04     2.34
##  2         2.68 0.0705 12     2.53     2.83
##  3         2.42 0.0705 12     2.27     2.57
##  4         2.31 0.0705 12     2.16     2.46
## 
## Confidence level used: 0.95
\end{verbatim}

이 경우 처리 평균에 대한 추정값은 산술 평균과 동일하게 나온다.

\hypertarget{uxc120uxd615uxbaa8uxd615uxc758-uxc801uxd569-sum-to-zero}{%
\section{선형모형의 적합 (sum-to-zero)}\label{uxc120uxd615uxbaa8uxd615uxc758-uxc801uxd569-sum-to-zero}}

이제 일원배치 모형에서 sum-to-zero 조건을 적용하여 모수를 추정해 보자.
sum-to-zero 조건을 적용하려면 다음과 같은 명령어를 실행해야 한다.

\begin{Shaded}
\begin{Highlighting}[]
\FunctionTok{options}\NormalTok{(}\AttributeTok{contrasts=}\FunctionTok{c}\NormalTok{(}\StringTok{"contr.sum"}\NormalTok{, }\StringTok{"contr.poly"}\NormalTok{))}
\end{Highlighting}
\end{Shaded}

이제 다시 선형모형을 적합하고 추정결과를 보자.

\begin{Shaded}
\begin{Highlighting}[]
\NormalTok{fit2 }\OtherTok{\textless{}{-}} \FunctionTok{lm}\NormalTok{(response}\SpecialCharTok{\textasciitilde{}}\NormalTok{company,}\AttributeTok{data=}\NormalTok{df31)}
\FunctionTok{summary}\NormalTok{(fit2)}
\end{Highlighting}
\end{Shaded}

\begin{verbatim}
## 
## Call:
## lm(formula = response ~ company, data = df31)
## 
## Residuals:
##     Min      1Q  Median      3Q     Max 
## -0.2600 -0.0700  0.0150  0.0625  0.2600 
## 
## Coefficients:
##             Estimate Std. Error t value Pr(>|t|)    
## (Intercept)   2.4000     0.0353   68.08  < 2e-16 ***
## company1     -0.2100     0.0611   -3.44  0.00490 ** 
## company2      0.2800     0.0611    4.59  0.00063 ***
## company3      0.0200     0.0611    0.33  0.74889    
## ---
## Signif. codes:  0 '***' 0.001 '**' 0.01 '*' 0.05 '.' 0.1 ' ' 1
## 
## Residual standard error: 0.141 on 12 degrees of freedom
## Multiple R-squared:  0.687,  Adjusted R-squared:  0.609 
## F-statistic: 8.78 on 3 and 12 DF,  p-value: 0.00235
\end{verbatim}

이제 sum-to-zero 조건에 따라서 위의 set-to-zero 결과와 모수의 추정값이
다르게 나타나는 것을 알 수 있다. 마지막 모수 \texttt{company4}(\(\alpha_4\))는
sum-to-zero 조건을 이용하여 다음과 같은 관계를 이용하여 구할 수 있다.

\[  \alpha_4 = -(\alpha_1 + \alpha_2 + \alpha_3) \]

sum-to-zero 조건에서의 계획행렬은 다음과 같이 볼 수 있다.

\begin{Shaded}
\begin{Highlighting}[]
\FunctionTok{model.matrix}\NormalTok{(fit2)}
\end{Highlighting}
\end{Shaded}

\begin{verbatim}
##    (Intercept) company1 company2 company3
## 1            1        1        0        0
## 2            1        1        0        0
## 3            1        1        0        0
## 4            1        1        0        0
## 5            1        0        1        0
## 6            1        0        1        0
## 7            1        0        1        0
## 8            1        0        1        0
## 9            1        0        0        1
## 10           1        0        0        1
## 11           1        0        0        1
## 12           1        0        0        1
## 13           1       -1       -1       -1
## 14           1       -1       -1       -1
## 15           1       -1       -1       -1
## 16           1       -1       -1       -1
## attr(,"assign")
## [1] 0 1 1 1
## attr(,"contrasts")
## attr(,"contrasts")$company
## [1] "contr.sum"
\end{verbatim}

이제 각 처리 평균에 대한 추정값 \(\widehat{\mu+ \alpha_i}\)을 구해보면 set-to-zero 조건에서의 추정값과 동일함을 알 수 있다.

\begin{Shaded}
\begin{Highlighting}[]
\FunctionTok{emmeans}\NormalTok{(fit2, }\StringTok{"company"}\NormalTok{)}
\end{Highlighting}
\end{Shaded}

\begin{verbatim}
##  company emmean     SE df lower.CL upper.CL
##  1         2.19 0.0705 12     2.04     2.34
##  2         2.68 0.0705 12     2.53     2.83
##  3         2.42 0.0705 12     2.27     2.57
##  4         2.31 0.0705 12     2.16     2.46
## 
## Confidence level used: 0.95
\end{verbatim}

\hypertarget{uxbd84uxc0b0uxbd84uxc11d}{%
\section{분산분석}\label{uxbd84uxc0b0uxbd84uxc11d}}

분산분석의 결과는 어떠한 제약 조건에서도 동일하다.

\begin{Shaded}
\begin{Highlighting}[]
\NormalTok{res1 }\OtherTok{\textless{}{-}} \FunctionTok{anova}\NormalTok{(fit1)}
\NormalTok{res1}
\end{Highlighting}
\end{Shaded}

\begin{verbatim}
## Analysis of Variance Table
## 
## Response: response
##           Df Sum Sq Mean Sq F value Pr(>F)   
## company    3  0.524  0.1747    8.78 0.0024 **
## Residuals 12  0.239  0.0199                  
## ---
## Signif. codes:  0 '***' 0.001 '**' 0.01 '*' 0.05 '.' 0.1 ' ' 1
\end{verbatim}

\begin{Shaded}
\begin{Highlighting}[]
\NormalTok{res2}\OtherTok{\textless{}{-}} \FunctionTok{anova}\NormalTok{(fit2)}
\NormalTok{res2}
\end{Highlighting}
\end{Shaded}

\begin{verbatim}
## Analysis of Variance Table
## 
## Response: response
##           Df Sum Sq Mean Sq F value Pr(>F)   
## company    3  0.524  0.1747    8.78 0.0024 **
## Residuals 12  0.239  0.0199                  
## ---
## Signif. codes:  0 '***' 0.001 '**' 0.01 '*' 0.05 '.' 0.1 ' ' 1
\end{verbatim}

  \bibliography{book.bib,packages.bib}

\end{document}
